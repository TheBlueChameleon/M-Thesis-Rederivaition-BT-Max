\documentclass[
	english,
	a4paper,
	fontsize=10pt,
	parskip=half,
	titlepage=true,
	DIV=12,
	final
]{scrreprt}


%==============================================================================%
% PACKAGES
%
% Standard text formatting
\usepackage[utf8]{inputenc}
\usepackage{babel}
\usepackage[T1]{fontenc}
\usepackage{lmodern}
\usepackage{microtype}
\usepackage{ragged2e}

\usepackage{csquotes}
\usepackage{xspace}

\usepackage{placeins}	% FloatBarrier.
\usepackage{url}
\usepackage[bf, format=plain]{caption}

\usepackage{hyperref}
\hypersetup{
    colorlinks,
    citecolor=black,
    filecolor=black,
    linkcolor=black,
    urlcolor=black
}

% gfx
\usepackage{wrapfig}
\usepackage{xcolor}

% tables
\usepackage{tabularx}
\usepackage{booktabs}
\usepackage{multicol}
\usepackage{multirow}
\usepackage{makecell}
\usepackage{color, colortbl}

% math
\usepackage{amsmath}
\usepackage{amssymb}
\usepackage{dsfont}
\let\olddiv\div
\usepackage[arrowdel]{physics}
\usepackage{mathtools}

% indexes, links, page format
\usepackage{scrlayer-scrpage}

% misc
\usepackage[super]{nth}
\usepackage[
	output-decimal-marker={.},
	input-symbols = {()},  			% do not treat "(" and ")" in any special way
	group-digits  = true  			% guess what.
]{siunitx}
\usepackage{minted}

%==============================================================================%
% GLOBAL MACROS
%

% Document properties
\newcommand{\myName}{Stefan Hartinger\xspace}
\newcommand{\myTitle}{Master Thesis Notes: Dealing with more than three modes\xspace}

\addtokomafont{labelinglabel}{\sffamily}

% Text abbreviations
\newcommand*{\ie}{i.\,e.\xspace}
\newcommand*{\eg}{e.\,g.\xspace}

% Misc Symbols
\newcommand*{\thus}{\ensuremath{\rightarrow}\xspace}
\newcommand*{\Thus}{\ensuremath{\Rightarrow}\xspace}

% Tables
\newcommand*{\tabcrlf}{\\ \hline}			% actually still allows for optional argument

% Math
\newcommand*{\numberthis}{\addtocounter{equation}{1}\tag{\theequation}}

\newcommand*{\smallfrac}  [2]{\ensuremath{{}^        {#1} \!/_        {#2}}}
\newcommand*{\smallfracrm}[2]{\ensuremath{{}^{\mathrm{#1}}\!/_{\mathrm{#2}}}}

\newcommand*{\transp}{\ensuremath{^\intercal}}

\newcommand*{\iunit}{\ensuremath{\mathrm{i}}}

\newcommand*{\setNaturals} {\ensuremath{\mathds{N}}}
\newcommand*{\setIntegers} {\ensuremath{\mathds{Z}}}
\newcommand*{\setReals}    {\ensuremath{\mathds{R}}}
\newcommand*{\setRationals}{\ensuremath{\mathds{Q}}}
\newcommand*{\setComplex}  {\ensuremath{\mathds{C}}}

\newcommand*{\Lag}{\ensuremath{\mathcal{L}}\xspace}
\newcommand*{\Ham}{\ensuremath{\mathcal{H}}\xspace}

%\newcommand*{\Poisson}[2]{\ensuremath{\left\{ {#1}, {#2} \right\}}}
% physics has \pb which is poisson bracket
% also use alias acom: anticommutator, which is exactly the same.

\newcommand*{\equalCond}{  \mathop{=}\limits^!  }

\DeclareMathOperator{\arsinh}{arsinh}
\DeclareMathOperator{\diag}{diag}

\newcommand*{\DD}[1]{\ensuremath{\text{D}\vec{#1}\;}}
\newcommand*{\set}[1]{\ensuremath{\{#1\}}}

%==============================================================================%
% GLOBAL PARAMTERS
%

\title{\myTitle}
\author{\myName}
\date{\today}

% header, footer
\clearpairofpagestyles
	\cfoot
		[\pagemark]
		{\pagemark}
	\ohead
		[\myTitle, \myName]
		{\myTitle, \myName}
\pagestyle{scrheadings}

%==============================================================================%
% THE REAL STUFF
%	
\begin{document}
\tableofcontents
\newpage

\chapter{The Exact Probability Amplitude in the $K=3$ Case}
\section{Multinomial Coefficient}
\begin{align}
	\mqty(N \\ k_1, k_2, \ldots k_M)
&\coloneqq
	\frac
		{N!}
		{\prod_{j=1}^{M} k_j!}
\end{align}

Note that, for $\sum_j k_j = N$, this will yield an integer. From this, we also get the statement for $\sum_j k_j \leq N$. This is due to $N!$ containing all possible factors, not only primes. It is not possible to \enquote{divide out} all prime factors with the restrictions on $k_j$.

Simple case: Assume $N = p_1 + p_2$, with $p_j$: prime. Then:
\begin{align}
	\mqty(N \\ p_1, p_2)
&=
	\frac{(p_1 + p_2)!}{p_1! \; p_2!}
=
	\frac
		{2 \cdot \ldots \cdot p_1 \cdot \ldots \cdot p_2 \cdot \ldots \cdot (p_1 + p_2)}
		{(2 \cdot \ldots \cdot p_1) \; (2 \cdot \ldots \cdot p_2)}
\end{align}

The prime factors $p_1, p_2$ are exactly cancelled. All other prime factors of $p_j!$ will be in $N!$, either directly or as integer multiples. It freaks me out, and I can't really proove it, but it is sound -- this will always give an integer.


\section{Starting Point: Generic Formula}
This is a transcription of eqn. (96) in Max thesis paper.

Assuming a $K \times K$ system. Should be compatible with zero occupancies (though not explicitly checked; the $2 \times 2$ implementation is compatible).

\begin{align}
\nonumber
	A_F(\vec{m}, \vec{n})
&=
	\frac
		{\sqrt{\prod_j m_j! \, n_j!}}
		{N!}
	\mqty(
		N \\
		m_1, m_2, \ldots, m_K
	) \\ 
\nonumber
&	\sum_{\nu_1^1 = 0}^{n_1}
	\sum_{\nu_2^1 = 0}^{n_1 - \nu_1^1}
	\ldots
	\sum_{\nu_{K-1}^1 = 0}^{n_1 - \nu_1^1 - \ldots - \nu_{K-2}^1} \\
\nonumber
&	\sum_{\nu_1^2 = 0}^{n_2}
	\sum_{\nu_2^2 = 0}^{n_2 - \nu_1^2}
	\ldots
	\sum_{\nu_{K-1}^2 = 0}^{n_2 - \nu_1^2 - \ldots - \nu_{K-2}^2} \\
\nonumber
&	\ldots \\
\nonumber
&	\sum_{\nu_1^K = 0}^{n_K}
	\sum_{\nu_2^K = 0}^{n_K - \nu_1^K}
	\ldots
	\sum_{\nu_{K-1}^K = 0}^{n_K - \nu_1^K - \ldots - \nu_{K-2}^K} \\
\nonumber
&\quad
	\prod_{j=1}^{K-1}
		\Theta\qty(
			m_j -
			 {\textstyle \sum_{l=1}^{K} \nu_j^l}
		)
		\Theta\qty[
			m_K - 
			{\textstyle \qty(\sum_{l=1}^{K}
				n_l - 
				\qty(
					\sum_{o=1}^{K-1} \nu_o^l
				))
			}
		] \\
\nonumber
&\qquad\qquad
		\mqty(
			m_j \\
			\nu_j^1, \nu_j^2, \ldots, \nu_j^K
		)
		\mqty(
			m_K \\
			\qty(n_1 - {
				\textstyle \sum_{o=1}^{K-1} \nu_o^1
			}),
			\qty(n_2 - {
				\textstyle \sum_{o=1}^{K-1} \nu_o^2
			}),
			\ldots,
			\qty(n_K - {
				\textstyle \sum_{o=1}^{K-1} \nu_o^K
			})
		) \\
&\qquad\qquad
	\prod_{l=1}^K
	\prod_{o=1}^{K-1}
		u_{o,l}^{\nu_o^l}
		u_{K,l}^{n_l - \sum_{p=1}^{K-1} \nu_p^l}
\end{align}


\section{Notes on Implementation}
Provide a function \texttt{bigInt multinomial(int, vector<int>)}. With this, the prefactor can be written as \texttt{sqrt(lambda) / factorial(N) * multinomial(N, m)}.

There is no way to absorb all the summations into one \texttt{for}-loop. If there were, the entire problem would fall into a different class and I'd be rid of my thesis. Yes, there are $\mathcal{O}(N^2)$ summations.



\section{Specialization: $K=3$}
\begin{align}
\nonumber
	A_F(\vec{m}, \vec{n})
&=
	\frac
		{\sqrt{\prod_j m_j! \, n_j!}}
		{N!}
	\mqty(
		N \\
		m_1, m_2, m_3
	) \\ 
\nonumber
&	\sum_{\nu_1^1 = 0}^{n_1}
	\sum_{\nu_2^1 = 0}^{n_1 - \nu_1^1}
	\sum_{\nu_1^2 = 0}^{n_2}
	\sum_{\nu_2^2 = 0}^{n_2 - \nu_1^2}
	\sum_{\nu_1^3 = 0}^{n_3}
	\sum_{\nu_2^3 = 0}^{n_3 - \nu_1^3} \\
\nonumber
&\quad
	\prod_{j=1}^{2}
		\Theta\qty(
			m_j -
			 {\textstyle \sum_{l=1}^{3} \nu_j^l}
		)
		\Theta\qty[
			m_3 - 
			{\textstyle \qty(\sum_{l=1}^3
				n_l - 
				\qty(
					\sum_{o=1}^2 \nu_o^l
				))
			}
		] \\
\nonumber
&\qquad\qquad
		\mqty(
			m_j \\
			\nu_j^1, \nu_j^2, \nu_j^3
		)
		\mqty(
			m_3 \\
			\qty(n_1 - {
				\textstyle \sum_{o=1}^2 \nu_o^1
			}),
			\qty(n_2 - {
				\textstyle \sum_{o=1}^2 \nu_o^2
			}),
			\qty(n_K - {
				\textstyle \sum_{o=1}^2 \nu_o^3
			})
		) \\
&\qquad\qquad
	\prod_{l=1}^3
	\prod_{o=1}^2
		u_{o,l}^{\nu_o^l}
		u_{3,l}^{n_l - \sum_{p=1}^2 \nu_p^l}
\end{align}


\chapter{Exact Phases in the $K=3$ case}
{\color{blue} I tried to find an analytical solution the the phase retrieval problem for three modes as a means to control the PRA results or really only select relevant data points. However, the attempts below proved unpractical. I kept it as a source for ideas / as documentation.}

\section{Failed Attempt: Linear Equation}
{\color{red} The approach below was ultimatively shown to be unusable, but formed the basis for the viable but unwieldy quadratic attempt.}

\subsection{Finding a linear equation}
Start from Fourier-Link (eqn. 1.24 in 04: redo), bring into $= 0$-form and do mod-square of equation.\\
Get (for freely choosable $l \in \{1,2,3\}$):
\begin{align}
	0
&=
	\qty[
		\sum_{j,k=1}^3
			\sqrt{m_k m_j}  \;
			u_{kl} u_{jl}^* \;
			\exp[\Big. \iunit(\theta_k - \theta_j)]
	] - n_l
	\label{eqn:FourierModSquare} \\
&=
	\qty[
		\sum_{k=1}^3
			m_k
			\underbrace{|u_{kl}|^2}_{= \smallfrac{1}{N}}
	] +
	\qty[
		\sum_{k=1}^3
		\sum_{j=k+1}^3
			\sqrt{m_k m_j}  \;
			u_{kl} u_{jl}^* \;
			\qty[
				\exp(\Big. \iunit(\theta_k - \theta_j)) +
				\exp(\Big. \iunit(\theta_j - \theta_k))
			]
	]	- n_l \\
&=
	1 + 
	\qty[
		\sum_{k=1}^3
		\sum_{j=k+1}^3
			\sqrt{m_k m_j}  \;
			u_{kl} u_{jl}^* \;
			2 \cosh(\Big. \iunit(\theta_k - \theta_j))
	]	- n_l
\end{align}
(note that $\mathbb{U}$ is symmetric -- cf \url{https://en.wikipedia.org/wiki/DFT_matrix}).

Do temporary renaming: call $x_{kj} \coloneqq 2 \cosh(\Big. \iunit(\theta_k - \theta_j))$, with $x_{kj} = x_{jk}$ (property of $\cosh$) and $x \in \setComplex$.

Find:
\begin{align}
	n_l - 1
&=
	\qty[
		\sum_{k=1}^3
		\sum_{j=k+1}^3
			\sqrt{m_k m_j}  \;
			u_{kl} u_{jl}^* \;
			x_{kj}
	]
\end{align}

I.\;e.: get \emph{three} equations ($l=1,2,3$) for \emph{three} unknowns (off-diagonal elements of $x_{jk}$). \Thus system can be solved for the three $x$'s. Together with fixing $\theta_1 = 0$ this solution should even be unique in variables $\theta_k$.

Compact form:
\begin{align}
	C_l
&=
	\sum_{k=1}^3
	\sum_{j=k+1}^3
		a^{(l)}_{kj} x_{kj}
=
	\mathbb{A}^{(l)} : \mathbb{X} \\
&=
	a^{(l)}_{12} x_{12} +
	a^{(l)}_{13} x_{13} +
	a^{(l)}_{23} x_{23}
	\label{eqn:matEqnPrecursor}
\end{align}

where $\mathbb{M}_1 : \mathbb{M}_2$ denotes the \emph{matrix scalar product} (as in math. modelling, Summer Term 2020) and $\mathbb{A}^{(l)}$ is a $3 \times 3$ upper triangular matrix where the main diagonal is zeros only. The equations (\ref{eqn:matEqnPrecursor}) (for $l = 1, 2, 3$) can hence be written as:
\begin{align}
	\vec{C}
&=
	\mathbb{B} \vec{x}
\end{align}
where
\begin{align*}
	\vec{C}
&=
	\mqty(n_l - \qty[
		\sum_{k=1}^3
			m_k
			|u_{kl}|^2
	])_{l=1,2,3}
=
	\mqty( \big. n_l - 1)_{l=1,2,3}
&
	a^{(l)}_{kj}
&=
	\sqrt{m_k m_j}  \; u_{kl} u_{jl}^*
\\
	\mathbb{B}
&\coloneqq
	\mqty(
		a^{(1)}_{12} & a^{(1)}_{13} & a^{(1)}_{23} \\
		a^{(2)}_{12} & a^{(2)}_{13} & a^{(2)}_{23} \\
		a^{(3)}_{12} & a^{(3)}_{13} & a^{(3)}_{23}
	)
&
	\vec{x}
&\coloneqq
	\mqty(x_{12} \\ x_{13} \\ x_{23})
\end{align*}

{\color{red}
$\det(\mathbb{B}) = 0$, \ie the system cannot be solved. Regard the defintion of $a_{jk}^{(l)}$: the rows of $\mathbb{B}$ are linearly dependent, so this is not only a matter of introducing some exceptions but an overall issue.
}

\subsection{Going back from cosh-variables to phases}
{\color{blue} This seems fleshed out so far and can be used, should I ever solve for the $x_{kj}$ variables. However I think it has low numerical accuracy.}

For $z \in \setComplex$, $\cosh^{-1}(z)$ has up to four solutions:

\begin{equation}
	\cosh^{-1}(z) = \log[ \sqrt{z - 1} \sqrt{z + 1} + z ]
\end{equation}
(cf. \url{https://www.wolframalpha.com/input/?i=acosh%28x+%2B+iy%29})

If $\sqrt{z}$ is a solution to $z^2 = c$, then so is $-\sqrt{z}$. The product of the two square roots can only have either of two signs, so we get to
\begin{equation}
	\cosh^{-1}(z) = \log[ {\color{blue}\pm} \sqrt{z - 1} \sqrt{z + 1} + z ]
\end{equation}

Finally, $\cosh$ is symmetric, hence we get another $\pm$:
\begin{equation}
	\cosh^{-1}(z) = {\color{blue}\pm} \log[ {\color{blue}\pm} \sqrt{z - 1} \sqrt{z + 1} + z ]
\end{equation}

This gives:
\begin{align}
	\theta_k - \theta_j = \mp \iunit \log\qty[
		\pm \frac{1}{2} \qty( 
			\sqrt{x_{kj} - 1} \sqrt{x_{kj} + 1} + x_{kj} 
	)]
\end{align}

We expect purely real phases $\theta_k, \theta_j$ which imposes a restriction on $x_{kj}$ that can be used to check correctness.

Since $\theta_1 \coloneqq 0$, we directly get $\theta_2 = -\iunit \cosh^{-1}(\smallfrac{1}{2} \; x_{12})$. The same condition allows to find four values for $\theta_3 = -\iunit \cosh^{-1}(x_{13})$.

From these, four values for $\theta_3$ should coincide with the \emph{sixteen} values found from 
\[ \theta_3 = \theta_2^{(r)} + \iunit \cosh^{-1}( \smallfrac{1}{2} \; x_{23}) \]

{\color{red} however, we usually expect \emph{six} solutions...}

\subsection{Numerical Stability}
I expect accuracy issues when dealing with trignometric functions. In particular, I only need the $(\mu, \nu) = (\exp(\iunit \theta), \exp(\iunit \chi))$ variables. Hence it is convenient, to solve not for $\theta$, but for $\mu$. {\color{red} The below section was not finished and I kept running into issues; put some effort into the equations before implementing}

Above considerations still hold when realizing:

\begin{align}
	x_{kj}
=
	\cosh(\iunit (\theta_k - \theta_j))
&=
	\frac{1}{2} \qty[ \Big.
		\exp(+\iunit \theta_k) \exp(-\iunit \theta_j) +
		\exp(+\iunit \theta_j) \exp(-\iunit \theta_k)
	] \\
&=
	\frac{1}{2} \qty[
		\mu_k \mu_j^* +
		\mu_j \mu_k^*
	]
\end{align}
giving a quadratic equation (in complex variables):

\begin{align}
	x_{kj} \mu_j
&=
	\frac{1}{2} \qty[
		\mu_k \underbrace{\abs{\mu_j}^2}_{=1} +
		\mu_j^2 \mu_k^*
	]
\\
	0
&=
	\frac{\mu_k^*}{2} \mu_j^2 +
	(\frac{\mu_k}{2} - x_{kj}) \mu_j
%=
%	\mu_j \qty(
%		\frac{\mu_k^*}{2} \mu_j +
%		\frac{\mu_k}{2} + x_{kj}
%	)
%=
%	\mu_j \qty[\big. x_{kj} + \Re(\mu_k) ]
\\
	\mu_j
&=
	\frac{1}{\mu_k^*} \qty[
		\pm (\frac{\mu_k}{2} - x_{kj}) - (\frac{\mu_k}{2} - x_{kj})
	]
\end{align}

With $\mu_1 = \mu_1^* = 1$, this again allows to find $\mu_2$:
\begin{align}
	\mu_2^+ &= 0 \\
	\mu_2^- &= \frac{-\mu_1 - 2 x_{12}}{\mu_1} = -1 - 2 x_{12}
\end{align}

and so:
\begin{align}
	\mu_3
\end{align}

\section{Quadratic Attempt}
{\color{blue} This attempt can be shown to be solveable and, I think, yields all possible solutions. I didn't finish it because it's too unwieldy and prone to numerical errors.}

\subsection{Formulating and Solving the System}
Start from eqn. (\ref{eqn:matEqnPrecursor}) and compute the \emph{square}:

\begin{align}
	C_l^2
=
	(n_l - 1)^2
&=
	\qty( \big.
	a^{(l)}_{12} x_{12} +
	a^{(l)}_{13} x_{13} +
	a^{(l)}_{23} x_{23}
	)^2 \\
&=
	\nonumber
	(a^{(l)}_{12})^2 (x_{12})^2 +
	(a^{(l)}_{13})^2 (x_{13})^2 +
	(a^{(l)}_{23})^2 (x_{23})^2 \\
	&\qquad
	2 a^{(l)}_{12} a^{(l)}_{13} x_{12} x_{13} +
	2 a^{(l)}_{12} a^{(l)}_{23} x_{12} x_{23} +
	2 a^{(l)}_{13} a^{(l)}_{23} x_{13} x_{23}
\end{align}

Relabel:
\begin{align*}
	x_{12} &\thus x_1 &
	x_{13} &\thus x_2 &
	x_{23} &\thus x_3
\\
	a_{12}^{(l)} &\thus a_1^{(l)} &
	a_{13}^{(l)} &\thus a_2^{(l)} &
	a_{23}^{(l)} &\thus a_3^{(l)}
\end{align*}

Begin again with resolving the system for $x_i$:
\begin{align}
	0
&=
	\qty(a_1^{(l)})^2 x_1^2 +
	\qty(a_2^{(l)})^2 x_2^2 +
	\qty(a_3^{(l)})^2 x_3^2 +
	a_1^{(l)}a_2^{(l)} x_1 x_2 +
	a_1^{(l)}a_3^{(l)} x_1 x_3 +
	a_2^{(l)}a_3^{(l)} x_2 x_3 -
	C_l^2
	\label{eqn:sysQuadratic}
\end{align}

Temporarily regard $x_1$ as the sole variable, pick $l=1$, get form:
\begin{align}
	0
&=
	\qty(a_1^{(1)})^2 x_1^2 +
	\beta_1 x_1 +
	\gamma_1
\\
	\beta_1
&=
	a_1^{(1)}a_2^{(1)} x_2 +
	a_1^{(1)}a_3^{(1)} x_3
\\
	\gamma_1
&=
	\qty(a_2^{(1)})^2 x_2^2 +
	\qty(a_3^{(1)})^2 x_3^2 +
	a_2^{(1)}a_3^{(1)} x_2 x_3 -
	C_1^2
\end{align}

With solution of quadratic equation:
\begin{align}
	x_1^{\pm}
&=
	\frac
		{-\beta_1 \pm \sqrt{\beta_1^2 - 4 \qty(a_1^{(1)})^2 \gamma_1}}
		{2 \qty(a_1^{(1)})^2}
\end{align}

Plug back into system (\ref{eqn:sysQuadratic}), resolve for $x_2$ in $l=2$:
\begin{align}
	0
&=
	\qty(a_2^{(2)})^2 x_2^2 +
	\beta_2 x_2 +
	\gamma_2
\\
	\beta_2
&=
	a_1^{(2)}a_2^{(2)} x_1 +
	a_2^{(2)}a_3^{(2)} x_3
\\
	\gamma_2
&=
	\qty(a_1^{(2)})^2 x_1^2 +
	\qty(a_3^{(2)})^2 x_3^2 +
	a_2^{(1)}a_3^{(1)} x_2 x_3 -
	C_2^2
\end{align}

and use expression for $x_1^{\pm}$:
\begin{align}
	\beta_2^\pm
&=
	a_1^{(2)}a_2^{(2)} x_1 +
	a_2^{(2)}a_3^{(2)} x_3
\end{align}

{\color{blue} You will find a new system with $x_1$ effectively removed, but already too ugly to really want to do that. But it \emph{is} possible to solve this with a free parameter $x_3$. Again you'll end up with a $x_2^{(r_2)}, r_2 \in \{1, \ldots, R_2\}$ as a function of $x_3$ (possibly a higher order polynomial). Do it a third time and you get a monster for $x_3^{(r_3)}, r_3 \in \{1, \ldots, R_3\}$, which will be really nasty to solve. Get a total of $2 \cdot R_2 \cdot R_3$ possible solutions.

The same path can be followed to solve the phase retrieval problem for any number of modes $K$.}



\subsection{Random Notes}
Using natural indices (starting at 1):
\begin{align}
	u_{jk}
&=
	\exp(-2\pi\iunit \frac{(j-1)(k-1)}{N})
\\
	u_{kl} u_{jl}^*
&=
	\exp(-2\pi\iunit \frac{(k-1)(l-1)}{N})  \exp(+2\pi\iunit \frac{(j-1)(l-1)}{N}) \\
&=
	\exp(+2\pi\iunit \frac{(j-k)(l-1)}{N})
=
	u_{(j-k+1), l}^*
\end{align}

\subsection{Faulty Approach}
{\color{red} This is based on the form in (\ref{eqn:sysQuadratic}), but I forgot about the square terms, which are not zero... maybe it can be used, anyway}

Eliminate the mixed terms with the $l$ constants $C_l$: \\
\begin{align}
	x_{12} x_{13}
&=
	\frac
		{1}
		{2 a^{(1)}_{12} a^{(1)}_{13}} \qty( \Big.
			C_1^2 -
			2 a^{(1)}_{12} a^{(1)}_{23} x_{12} x_{23} -
			2 a^{(1)}_{13} a^{(1)}_{23} x_{13} x_{23}
		)
\\
	{\color{blue} x_{12} x_{23}}
&=
	\frac
		{1}
		{2 a^{(2)}_{12} a^{(2)}_{23}} \qty( \Big.
			C_2^2 -
			2 a^{(2)}_{12} a^{(2)}_{13} x_{12} x_{13} -
			2 a^{(2)}_{13} a^{(2)}_{23} x_{13} x_{23}
		)
\\
&=
	\frac
		{1}
		{2 a^{(2)}_{12} a^{(2)}_{23}} \qty( \Big.
			C_2^2 -
			2 a^{(2)}_{12} a^{(2)}_{13} 
				\frac
						{1}
						{2 a^{(1)}_{12} a^{(1)}_{13}} \qty( \Big.
							C_1^2 -
							2 a^{(1)}_{12} a^{(1)}_{23} {\color{blue} x_{12} x_{23}} -
							2 a^{(1)}_{13} a^{(1)}_{23} x_{13} x_{23}
						)
			-
			2 a^{(2)}_{13} a^{(2)}_{23} x_{13} x_{23}
		)
\\
&=
	\frac
		{1}
		{2 a^{(2)}_{12} a^{(2)}_{23}} \qty( \Big.
			C_2^2 -
			2 a^{(2)}_{12} a^{(2)}_{13} 
				\frac
						{1}
						{2 a^{(1)}_{12} a^{(1)}_{13}} \qty( \Big.
							C_1^2 -
							2 a^{(1)}_{13} a^{(1)}_{23} x_{13} x_{23}
						)
			- 2 a^{(2)}_{12} a^{(2)}_{13} {\color{blue} x_{12} x_{23}}
			- 2 a^{(2)}_{13} a^{(2)}_{23} x_{13} x_{23}
	)
\\
	\qty(
		1 + 
		\frac
			{a^{(2)}_{12} a^{(2)}_{13}}
			{a^{(2)}_{12} a^{(2)}_{23}}
	) {\color{blue} x_{12} x_{23}}
&=
	\frac
		{1}
		{2 a^{(2)}_{12} a^{(2)}_{23}} \qty( \Big.
			C_2^2 -
			2 a^{(2)}_{12} a^{(2)}_{13} 
				\frac
						{1}
						{2 a^{(1)}_{12} a^{(1)}_{13}} \qty( \Big.
							C_1^2 -
							2 a^{(1)}_{13} a^{(1)}_{23} {\color{olive} x_{13} x_{23}}
						)
			- 2 a^{(2)}_{13} a^{(2)}_{23} {\color{olive} x_{13} x_{23}}
	)
\\
	{\color{blue} x_{12} x_{23}}
&=
	\underbrace{	\frac
		{1}
		{2 a^{(2)}_{12} a^{(2)}_{23}
			\qty(
				1 +
				\frac 
				  {a^{(2)}_{12} a^{(2)}_{13}}
					{a^{(2)}_{12} a^{(2)}_{23}}
			)
		}}_{\beta_1}
	\qty( \Big.
		\underbrace{
			C_2^2 -
			2 a^{(2)}_{12} a^{(2)}_{13} 
				\frac
					{C_1^2}
					{2 a^{(1)}_{12} a^{(1)}_{13}} 
		}_{\beta_2}
			-
			\underbrace{\qty( \Big.
				\frac
					{a^{(1)}_{13} a^{(1)}_{23}}
					{a^{(1)}_{12} a^{(1)}_{13}} 
				- 2 a^{(2)}_{13} a^{(2)}_{23}
			)}_{\beta_3}
		{\color{olive} x_{13} x_{23}}
	) \\
&=
	\beta_1 \beta_2 - \beta_1 \beta_3 {\color{olive} x_{13} x_{23}}
\end{align}

\subsection{Cross Terms solution}
{\color{blue} Assuming the cross terms $x_{kj} x_{pq} = x_s x_t$ are accessible, below you find a means to get to the individual $x_s$}

Start from a system of form:
\begin{align}
	x_1 x_2 &= \alpha_3 &
	x_2 x_3 &= \alpha_1 &
	x_3 x_1 &= \alpha_2
\end{align}

which allows:
\begin{align}
	x_1 &= \frac{\alpha_3}{x_2} &
	x_2 &= \frac{\alpha_1}{x_3} &
	x_3 &= \frac{\alpha_2}{x_1}
\\
	\Thus
	x_1 &= \frac{\alpha_3}{\alpha_1} x_3 = \frac{\alpha_3}{\alpha_1} \frac{\alpha_2}{x_1}
\\
	\Thus
	x_1^2 &= \frac{\alpha_2 \alpha_3}{\alpha_1} &
	x_1 &= \pm \sqrt{\frac{\alpha_2 \alpha_3}{\alpha_1}}
\end{align}

So we get two solutions for $x_1 = x_{12}$ and from each of them four solutions for $\theta_2$ for a total of eight values for $\theta_2$.
\end{document}