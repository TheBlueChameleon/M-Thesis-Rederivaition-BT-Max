\documentclass[
	english,
	a4paper,
	fontsize=10pt,
	parskip=half,
	titlepage=true,
	DIV=12,
	final
]{scrreprt}


%==============================================================================%
% PACKAGES
%
% Standard text formatting
\usepackage[utf8]{inputenc}
\usepackage{babel}
\usepackage[T1]{fontenc}
\usepackage{lmodern}
\usepackage{microtype}
\usepackage{ragged2e}

\usepackage{csquotes}
\usepackage{xspace}

\usepackage{placeins}	% FloatBarrier.
\usepackage{url}
\usepackage[bf, format=plain]{caption}

\usepackage{hyperref}
\hypersetup{
    colorlinks,
    citecolor=black,
    filecolor=black,
    linkcolor=black,
    urlcolor=black
}

% gfx
\usepackage{wrapfig}
\usepackage{xcolor}

% tables
\usepackage{tabularx}
\usepackage{booktabs}
\usepackage{multicol}
\usepackage{multirow}
\usepackage{makecell}
\usepackage{color, colortbl}

% math
\usepackage{amsmath}
\usepackage{amssymb}
\usepackage{dsfont}
\let\olddiv\div
\usepackage[arrowdel]{physics}
\usepackage{mathtools}

% indexes, links, page format
\usepackage{scrlayer-scrpage}

% misc
\usepackage[super]{nth}
\usepackage[
	output-decimal-marker={.},
	input-symbols = {()},  			% do not treat "(" and ")" in any special way
	group-digits  = true  			% guess what.
]{siunitx}
\usepackage{minted}

%==============================================================================%
% GLOBAL MACROS
%

% Document properties
\newcommand{\myName}{Stefan Hartinger\xspace}
\newcommand{\myTitle}{Master Thesis Notes: Optimization of numerics for the determinant of the Hessian\xspace}

\addtokomafont{labelinglabel}{\sffamily}

% Text abbreviations
\newcommand*{\ie}{i.\,e.\xspace}
\newcommand*{\eg}{e.\,g.\xspace}

% Misc Symbols
\newcommand*{\thus}{\ensuremath{\rightarrow}\xspace}
\newcommand*{\Thus}{\ensuremath{\Rightarrow}\xspace}

% Tables
\newcommand*{\tabcrlf}{\\ \hline}			% actually still allows for optional argument

% Math
\newcommand*{\numberthis}{\addtocounter{equation}{1}\tag{\theequation}}

\newcommand*{\smallfrac}  [2]{\ensuremath{{}^        {#1} \!/_        {#2}}}
\newcommand*{\smallfracrm}[2]{\ensuremath{{}^{\mathrm{#1}}\!/_{\mathrm{#2}}}}

\newcommand*{\transp}{\ensuremath{^\intercal}}

\newcommand*{\iunit}{\ensuremath{\mathrm{i}}}

\newcommand*{\setNaturals} {\ensuremath{\mathbb{N}}}
\newcommand*{\setIntegers} {\ensuremath{\mathbb{Z}}}
\newcommand*{\setReals}    {\ensuremath{\mathbb{R}}}
\newcommand*{\setRationals}{\ensuremath{\mathbb{Q}}}
\newcommand*{\setComplex}  {\ensuremath{\mathbb{C}}}

\newcommand*{\Lag}{\ensuremath{\mathcal{L}}\xspace}
\newcommand*{\Ham}{\ensuremath{\mathcal{H}}\xspace}

%\newcommand*{\Poisson}[2]{\ensuremath{\left\{ {#1}, {#2} \right\}}}
% physics has \pb which is poisson bracket
% also use alias acom: anticommutator, which is exactly the same.

\newcommand*{\equalCond}{  \mathop{=}\limits^!  }

\DeclareMathOperator{\arsinh}{arsinh}
\DeclareMathOperator{\diag}{diag}

\newcommand*{\DD}[1]{\ensuremath{\text{D}\vec{#1}\;}}

%==============================================================================%
% GLOBAL PARAMTERS
%

\title{\myTitle}
\author{\myName}
\date{\today}

% header, footer
\clearpairofpagestyles
	\cfoot
		[\pagemark]
		{\pagemark}
	\ohead
		[\myTitle, \myName]
		{\myTitle, \myName}
\pagestyle{scrheadings}

%==============================================================================%
% THE REAL STUFF
%	
\begin{document}
\tableofcontents
\newpage

\chapter{Analytical Form of $\det S$}
\section{non-ZO, 2x2}
\subsection{Resources}
For incoming non-ZOs (incoming ZO requires 50/50 up/down) we got:
\begin{align}
	\mu_{\pm}^{(l)}
&=
	\frac
	{
		- m_1 |u_{1,l}|^{2}  -  m_2 |u_{2,l}|^{2} + n_l
		\pm
		\sqrt{
			(m_1 |u_{1,l}|^{2}  +  m_2 |u_{2,l}|^{2} - n_l)^{2}
			-
			4 m_1 m_2 |u_{1,l}|^{2} |u_{2,l}|^{2}
		}
	}
	{ 2\sqrt{m_1 m_2} \, u_{1,l}^{*} \, u_{2,l} }
\end{align}
where $\mu = \exp(\iunit \theta_2)$.

Introduce $\nu_q = \exp(\iunit \chi_q)$ and use Fourier-Link to $\mu$:
\begin{align}
	\nu_q
&=
	\exp(+\iunit \chi_q)
=
	\sum_{k=1}^{K}
		\frac{\sqrt{m_k}}{\sqrt{n_q}}
		u_{k,q} \; \exp(+\iunit\theta_k) \\
&=
	\sqrt{\frac{1}{n_q}} \qty(
		  \sqrt{m_1} u_{1q}
		+ \sqrt{m_2} u_{2q} \mu
	)
=
	\frac{
	    \sqrt{m_1} u_{1q}
		+ \sqrt{m_2} u_{2q} \mu}
		{\sqrt{n_q}}
\\
	\nu_q^2
&=
	\frac{1}{n_q} \qty(
		  m_1 u_{1q}^2
		+ \sqrt{m_1 m_2} u_{1q} u_{2q} \mu
		+ m_2 u_{2q}^2 \mu^2
	)
\label{eqn:nuSquaredExpansion}
\end{align}

Determinant, max generic:
\begin{align}
	\det S
&=
	\frac{ \lambda^{-b_i^* - b_o^* + 2} }{ m_{j^*} }
	\qty( \prod_{k: n_k \neq 0} \exp(-2\iunit \chi_k) )
	\det\qty[\bigg.
		\diag\qty(\exp(\Big.2\iunit\vec{\chi}))
		-
		\tilde{\mathbb{U}}\transp
		\diag\qty(\exp(2\iunit\vec{\theta}))
		\tilde{\mathbb{U}}
	]
\end{align}

where
\begin{align}
	\lambda
&=
	\qty( \prod_{j : m_j \neq 0} m_j )
	\qty( \prod_{k : n_k \neq 0} n_k )
&
	\tilde{u}_{j,k} 
&=
	\begin{cases}
		0			&\qq*{for } j = j^* \\
		0			&\qq*{for } m_j = 0 \\
		0			&\qq*{for } n_k = 0 \\
		-u_{j,k}		&\qotherwise*
	\end{cases}
\end{align}

\subsection{Evaluation of det formula}

\begin{align}
	\det S
&=
	\frac{ \lambda^{-2} }{ m_1 }
	\qty( \nu_1 \nu_2 )^{-2}
	\det\qty[
		\begin{pmatrix}
			\nu_1^2 & 0 \\
			0 & \nu_2^2
		\end{pmatrix}
		-
		\frac{1}{2}
		\begin{pmatrix}
			 0 & -1 \\
			 0 &  1
		\end{pmatrix}
		\begin{pmatrix}
			1 & 0 \\
			0 & \mu^2
		\end{pmatrix}
		\begin{pmatrix}
			 0 &  0 \\
			-1 &  1
		\end{pmatrix}
	] \\
&=
	\frac{1}{m_1^3 (m_2 n_1 n_2 \nu_1 \nu_2)^2}
	\det \begin{pmatrix}
		\nu_1^2 - \frac{1}{2} \mu^2	& + \frac{1}{2} \mu^2 \\
		+ \frac{1}{2} \mu^2					& \nu_2^2 - \frac{1}{2} \mu^2
	\end{pmatrix}
\end{align}
see rederivation, 2.3.6: Complete Det of Hessian for the matrix multiplication.

\begin{align*}
	\det S
&=
	\frac{1}{m_1^3 (m_2 n_1 n_2 \nu_1 \nu_2)^2} \qty[
	\qty(\nu_1^2 - \frac{1}{2} \mu^2)
	\qty(\nu_2^2 - \frac{1}{2} \mu^2)
	-
	\qty(\frac{1}{2} \mu^2)
	\qty(\frac{1}{2} \mu^2) ] \\
&=
	\frac{1}{m_1^3 (m_2 n_1 n_2 \nu_1 \nu_2)^2}
	\qty(
		  \nu_1^2 \nu_2^2
		- \frac{1}{2} \mu^2 \nu_1
		- \frac{1}{2} \mu^2 \nu_2
		+ \frac{1}{4} \mu^4
		- \frac{1}{4} \mu^4
	) \\
&=
	\frac{1}{m_1^3 (m_2 n_1 n_2 \nu_1 \nu_2)^2}
	\qty[
		  \nu_1^2 \nu_2^2
		- \frac{1}{2} \mu^2 \qty(\Big. \nu_1 + \nu_2)
	] \\
&=
	\frac{1}{m_1^3 (m_2 n_1 n_2)^2}
	\qty[
		  1
		- \frac{\mu^2}{2} \frac{\Big. \nu_1 + \nu_2}{\nu_1^2 \nu_2^2}
	]
	& \texttt{Variant (a)} \\
&=
	\frac{1}{m_1^3 (m_2 n_1 n_2)^2 \nu_1 \nu_2}
	\qty[
		  \nu_1 \nu_2
		- \frac{\mu^2}{2} \frac{\Big. \nu_1 + \nu_2}{\nu_1 \nu_2}
	]
	& \texttt{Variant (b)}
\end{align*}

\subsection{Numerical Stability}
Variant a (initially preferred)
\begin{itemize}
\item prefactor (\smallfrac{1}{m_1^3...}) comprises only of ints \Thus product exact, inverse highly accurate.
\item Factors $\nu_q$ satisfy $|\nu_q| = 1$ -- very likely to get equal-magnitude components.
\item Factors $\nu_i^2$ should be computed as squares rather than as expansion (\ref{eqn:nuSquaredExpansion})
\item Read the condition of this directly from the parenthesis -- the closer to zero, the worse.
\item Storing the denominator of the prefactor separately can be done with perfect accuracy and applied once for the entire amplitude formula, thereby improving the condition
\end{itemize}

Variant b
\begin{itemize}
\item no $\nu^2$ terms -- might be advantageous, if this is actually the main source of numerical error
\item however more likely to make things worse
\end{itemize}

\section{Incoming ZO, 2x2}
No analytical form for $\mu$ available then. In principle, this is not a problem, since we still get a $\mu$ from PRA. If they are corrupt, there is a reasonable strategy to deal with this, as incoming ZO should only have at most two solutions in 2x2. Lower priority issue.

\section{Outgoing ZO, 2x2}
\subsection{Resources}
\begin{itemize}
\item Formula for $\mu$ holds.
\item Formula for $\nu$ holds, too, but may only be evaluated for occupied outgoing modes.
\end{itemize}

\subsection{detS}
Cf section 2.4 in rederivation: Get a much simpler form due to sparse form of $\tilde{U}$:
\begin{align}
	\det S
&=
	\frac{1}{m_1^2 m_2 n^*}
	\qty[
		1 - \frac{\mu}{2\nu^*}
	]
\end{align}
where $n^*, \nu^*$ refers to the occupied outgoing mode. This cannot be emulated from the generic from and requires (minor) casework. Again, we find a prefactor that only depends on the occupancy.

A second strategy already needs to be implemented for incoming ZOs, so this does not impose too much structural effort. Implementing a \texttt{strategy} enum could allow for exploiting analytical features of the prefactors.

\section{Higher Dimensions, no ZO and ZO}
Separation in a magnitude function $\Omega(\vec{m}, \vec{n})$ and an orientation part $\Upsilon(\vec{\nu}, \vec{\mu})$ will still be possible. Fourier-Link $\vec{\mu} \to \vec{\nu}$ still holds.
Rewrite:
\begin{align}
	\det S
&=
	\underbrace{
		\frac{ 1 }{ m_{j^*} \lambda^{b_i^* + b_o^* - 2} }
	}_{=\Omega}
	\underbrace{
	\qty( \prod_{k: n_k \neq 0} \nu_k^{-2} )
	\det\qty[\bigg.
		\begin{pmatrix}
			\nu_1^2 & 0 && \ldots  & 0 \\
			0 & \nu_2^2 & 0 & \ldots & 0 \\
			\vdots & & \ddots & & 0 \\
			0 & \ldots &&& \nu_K^2
		\end{pmatrix}
		-
		\tilde{\mathbb{U}}\transp
		\begin{pmatrix}
			\mu_1^2 & 0 && \ldots  & 0 \\
			0 & \mu_2^2 & 0 & \ldots & 0 \\
			\vdots & & \ddots & & 0 \\
			0 & \ldots &&& \mu_K^2
		\end{pmatrix}
		\tilde{\mathbb{U}}
	]
	}_{\Upsilon}
\end{align}

Since $|\mu_j| = |\nu_j| = 1$ and $\tilde{U}$: unitary, the condition of $\Upsilon$ should be quite good. The big number of additions in the latter matrix product might cause trouble for big systems.

The formula is thought to be ZO-proof. In the derivation, undefined $\mu_j, \nu_j$ were set to one (insertion of identity blocks for conservation of det).

\chapter{Analytical Form of $A$ for the generic case}
\section{Resources}
Formula as derived in zeroOcc, put into $\mu, \nu$ form:
\begin{multline}
	A_F(\vec{m}, \vec{n})
=
	(-1)^{b_o^*}
	2\pi \, \delta_{N,M} \, \qty(
		\frac{1}{2\pi\lambda}
	)^{\frac{b_i^* + b_o^* - 1}{2}}
	\exp(N)
	\qty( \prod_{j : m_j \neq 0}
		\sqrt{\frac
			{ m_j! }
			{ m_j^{m_j} }
	})
	\qty( \prod_{k : n_k \neq 0}
		\sqrt{\frac
			{ n_k! }
			{ n_k^{n_k} }
	})
\\
	\sum_r
		\frac
			{\prod_{k             : n_k \neq 0} \qty(\nu_k^{(r)})^{n_k}}
			{\prod_{j\neq j^* \land m_j \neq 0} \qty(\mu_j^{(r)})^{m_j}}
		\frac
			{1}
			{\sqrt{\det S_r}}
\end{multline}

\section{Reform}
Using the previous results, the amplitude can be factored into an occupation- and a phase-function, using the symbols $\Omega, \Upsilon$ from before:
\begin{multline}
	A_F(\vec{m}, \vec{n})
=
	(-1)^{b_o^*}
	2\pi \, \delta_{N,M} \, \qty(
		\frac{1}{2\pi\lambda}
	)^{\frac{b_i^* + b_o^* - 1}{2}}
	\qty( \prod_{j : m_j \neq 0}
		\sqrt{\frac
			{ m_j! }
			{ m_j^{m_j} }
	})
	\qty( \prod_{k : n_k \neq 0}
		\sqrt{\frac
			{ n_k! }
			{ n_k^{n_k} }
	})
	\frac
		{\exp(N)}
		{\sqrt{\Omega}}
\\
	\sum_r
		\frac
			{\prod_{k             : n_k \neq 0} \qty(\nu_k^{(r)})^{n_k}}
			{\prod_{j\neq j^* \land m_j \neq 0} \qty(\mu_j^{(r)})^{m_j}}
		\frac
			{1}
			{\sqrt{\Upsilon_r}}
\label{eqn:Amplitude}
\end{multline}

Note that $\frac{1}{\Omega} = m_{j^*} \lambda^{b_i^* + b_o^* - 2} $ is a product of integers and can be stored exactly. Thus, the entire first line of (\ref{eqn:Amplitude}) has only first order errors, but no propagation.

Evaluate $\sqrt{\Upsilon}$ first, then the fraction (rule: summation first; sqrt most likely involves sums).

\chapter{Stirling Formula}
Expressions of the form $\sqrt{\frac{x!}{x^x}}$ behave poorly in numerics. Use the Stirling Formula:
\begin{align*}
	n! 
&\approx
	\sqrt{2\pi n}
	\qty(\frac{n}{e})^n
\end{align*}
to get
\begin{align*}
	\sqrt{\frac{x!}{x^x}}
&\approx
	\sqrt{
		\frac
			{\sqrt{2\pi n}}
			{\exp(n)}
	}
=
	\qty(2\pi n)^{\smallfrac{1}{4}}
	\exp(-\smallfrac{n}{2})
\end{align*}
\end{document}