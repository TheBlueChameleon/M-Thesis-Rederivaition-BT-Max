\documentclass[
	english,
	a4paper,
	fontsize=10pt,
	parskip=half,
	titlepage=true,
	DIV=12,
	final
]{scrreprt}


%==============================================================================%
% PACKAGES
%
% Standard text formatting
\usepackage[utf8]{inputenc}
\usepackage{babel}
\usepackage[T1]{fontenc}
\usepackage{lmodern}
\usepackage{microtype}
\usepackage{ragged2e}

\usepackage{csquotes}
\usepackage{xspace}

\usepackage{placeins}	% FloatBarrier.
\usepackage{url}
\usepackage[bf, format=plain]{caption}

\usepackage{hyperref}
\hypersetup{
    colorlinks,
    citecolor=black,
    filecolor=black,
    linkcolor=black,
    urlcolor=black
}

% gfx
\usepackage{wrapfig}

% tables
\usepackage{tabularx}
\usepackage{booktabs}
\usepackage{multicol}
\usepackage{multirow}
\usepackage{makecell}
\usepackage{color, colortbl}

% math
\usepackage{amsmath}
\usepackage{amssymb}
\usepackage{dsfont}
\let\olddiv\div
\usepackage[arrowdel]{physics}
\usepackage{mathtools}

% indexes, links, page format
\usepackage{scrlayer-scrpage}

% misc
\usepackage[super]{nth}
\usepackage[
	output-decimal-marker={.},
	input-symbols = {()},  			% do not treat "(" and ")" in any special way
	group-digits  = true  			% guess what.
]{siunitx}
\usepackage{minted}

%==============================================================================%
% GLOBAL MACROS
%

% Document properties
\newcommand{\myName}{Stefan Hartinger\xspace}
\newcommand{\myTitle}{Rederivation of Max' B-Thesis\xspace}

\addtokomafont{labelinglabel}{\sffamily}

% Text abbreviations
\newcommand*{\ie}{i.\,e.\xspace}
\newcommand*{\eg}{e.\,g.\xspace}

% Misc Symbols
\newcommand*{\thus}{\ensuremath{\rightarrow}\xspace}
\newcommand*{\Thus}{\ensuremath{\Rightarrow}\xspace}

% Tables
\newcommand*{\tabcrlf}{\\ \hline}			% actually still allows for optional argument

% Math
\newcommand*{\numberthis}{\addtocounter{equation}{1}\tag{\theequation}}

\newcommand*{\smallfrac}  [2]{\ensuremath{{}^        {#1} \!/_        {#2}}}
\newcommand*{\smallfracrm}[2]{\ensuremath{{}^{\mathrm{#1}}\!/_{\mathrm{#2}}}}

\newcommand*{\transp}{\ensuremath{^\intercal}}

\newcommand*{\iunit}{\ensuremath{\mathrm{i}}}

\newcommand*{\setNaturals} {\ensuremath{\mathbb{N}}}
\newcommand*{\setIntegers} {\ensuremath{\mathbb{Z}}}
\newcommand*{\setReals}    {\ensuremath{\mathbb{R}}}
\newcommand*{\setRationals}{\ensuremath{\mathbb{Q}}}
\newcommand*{\setComplex}  {\ensuremath{\mathbb{C}}}

\newcommand*{\Lag}{\ensuremath{\mathcal{L}}\xspace}
\newcommand*{\Ham}{\ensuremath{\mathcal{H}}\xspace}

%\newcommand*{\Poisson}[2]{\ensuremath{\left\{ {#1}, {#2} \right\}}}
% physics has \pb which is poisson bracket
% also use alias acom: anticommutator, which is exactly the same.

\newcommand*{\equalCond}{  \mathop{=}\limits^!  }

\DeclareMathOperator{\arsinh}{arsinh}
\DeclareMathOperator{\diag}{diag}

%==============================================================================%
% GLOBAL PARAMTERS
%

\title{\myTitle}
\author{\myName}
\date{\today}

% header, footer
\clearpairofpagestyles
	\cfoot
		[\pagemark]
		{\pagemark}
	\ohead
		[\myTitle, \myName]
		{\myTitle, \myName}
\pagestyle{scrheadings}

%==============================================================================%
% THE REAL STUFF
%	
\begin{document}
\tableofcontents
\newpage

\chapter{Boson Sampling Amplitude (BSA)}
\section{Starting Point}
Statement:
\begin{align}
	A_F(\vec{m}, \vec{n})
&=
	\eval{\qty(
			\prod_{j}
			\frac
				{1}
				{\sqrt{m_j! \; n_j!}}
			\pdv[m_j]{x_j}
			\pdv[n_j]{y_j}
		) \exp(\vec{x}\transp \, \mathbb{U} \, \vec{y})
	}_{\vec{x} = \vec{y} = \vec{0}}
\label{eqn:BSA_raw}
\end{align}

Symbols:
\begin{itemize}
\item $\vec{m}, \vec{n}$: number of photons entering in the input modes ($\vec{m}$), emerging in the
	output modes ($\vec{n}$)
\item $A_F$ probability amplitude for the evolution $\vec{m} \thus \vec{n}$, \ie $|A_F|^{2} = P$
\item $\mathbb{U}$: unitary matrix describing the system.
\end{itemize}


\section{Derivatives to Integrals}
\subsection{1D} 
\label{sec:DevToInt_1D}
\begin{align}
	f^{(n)}(0)
&=
	\frac{n!}{2\pi \iunit}
	\oint_\gamma \dd{z}
		\frac
			{f(z)}
			{z^{n+1}}
\end{align}

Symbols:
\begin{itemize}
\item $f$: function matching the signature $\setComplex \thus \setComplex$; holomorphic in \setComplex
	(thus arbitrarily many times diff'able)
\item $\gamma$: arbitrary loop in the complex plane, enclosing the origin.
\end{itemize}

Holds because:
\begin{itemize}
\item Application of differential quotient and Cauchy's integral formula
\end{itemize}

\subsection{Generalization in $D$ dimensions} 
\label{sec:DevToInt_DD}
\begin{align}
	\eval{ f(\vec{z}) }_{\vec{z} = \vec{0}}
&=
	\qty(
		\frac{1}{2\pi \iunit}
	)^{D}
	\oint_{\gamma} \dd{\xi_1} \ldots \dd{\xi_D}
		\frac
			{f(\vec{\xi})}
			{\prod_{j=1}^{D} \xi_j}
\\
	\eval{ f^{(n)}(\vec{z}) }_{\vec{z} = \vec{0}}
&=
	\qty(
		\frac{n!}{2\pi \iunit}
	)^{D}
	\oint_{\gamma} \dd{\xi_1} \ldots \dd{\xi_D}
		\frac
			{f(\vec{\xi})}
			{\prod_{j=1}^{D} \xi_j^{n+1}}
\end{align}

Symbols:
\begin{itemize}
\item $f$: function matching the signature $\setComplex^{D} \thus \setComplex$; holomorphic, \ie
	holomorphic in each variable
\item $\vec{z}$: vector in $D$ dimensions, representing $\vec{x} \oplus \vec{y}$
\item $\vec{\xi}$: vector in $D$ dimensions, complex valued, associated to $\vec{z}$
\item $\gamma$: arbitrary loop in the complex plane, enclosing the origin.
\end{itemize}

Holds because:
\begin{itemize}
\item Repeatedly applying the argument from \ref{sec:DevToInt_1D}
\item Keep in mind: this is a \enquote{derivative in all directions}, thus we get an int in $D$ vars
\end{itemize}

\section{BSA in Integral From}
Statement:
\begin{align}
	A_F(\vec{m}, \vec{n})
&=
	\qty(
		\prod_{j=1}^{K}
		\frac
			{\sqrt{m_j! \; n_j!}}
			{-4\pi^{2}}
	)
	\oint_{\gamma}
		\qty(
			\frac
			{\dd{x_j} \dd{y_j}}
			{x_j^{m_j+1}  y_j^{n_j+1}}
		)
		\exp(\vec{x}\transp \, \mathbb{U} \, \vec{y})
\end{align}

Symbols
\begin{itemize}
\item $K$: number of ingoing/outgoing modes; $2K = D$; implicitly assuming $K=L$ (ingoing=outgoing)here.
\item $\gamma$: arbitrary loop in the complex plane, enclosing the origin.
\end{itemize}

Holds because:
\begin{itemize}
\item Apply ideas of \ref{sec:DevToInt_DD} to eqn. (\ref{eqn:BSA_raw})
\item Implicitly do $2K$ such transformations, since $\vec{z} = \vec{x} \oplus \vec{y}$ and thus has
	$2K$ components
\item[\Thus] prefactor $\smallfrac{1}{2\pi\iunit} \thus \qty( \smallfrac{1}{-4\pi^{2}} )^{K}$
\item Numerator gets a factor $\prod_{j} m_j! \; n_j!$ from applying the derivative-to-int method.
	This is partially \enquote{consumed} by the square root in the original BSA formula. \newline
	\Thus $\smallfrac{1}{\sqrt{\ldots}}$
\end{itemize}

\section{Absorbing Prefactors in BSA}
Statement:
\begin{align}
\nonumber
	A_F(\vec{m}, \vec{n})
=
	\qty(
		\prod_{j=1}^{K}
		\frac
			{\sqrt{m_j! \; n_j!}}
			{-4\pi^{2}}
	)
	\oint_{\gamma}
		& \qty(
			\prod_{j=1}^{K}
			\dd{x_j} \dd{y_j}
		)  \cdot \ldots \\
		& \exp[
			-\sum_{j=1}^{K} 
			\qty[ \Big.
				(m_j + 1) \log(x_j) +
				(n_j + 1) \log(y_j)
			]
		]
		\exp(\vec{x}\transp \, \mathbb{U} \, \vec{y})
\label{eqn:BSA_IntAbsorbed}
\end{align}

Symbols
\begin{itemize}
\item as before
\end{itemize}

Holds because
\begin{itemize}
\item $a \exp(x) = \exp[\log(a)] \exp(x) = \exp[\log(a) x]$
\item $\log[ \qty( \frac{1}{a} )^{n} ] = -n \log(a)$
\end{itemize}

\section{Change of Variables}
\label{sec:BSA_phases}
Using polar coordinates:
\begin{align}
	\begin{cases}
	x_j &= \sqrt{m_j} \exp( \iunit \theta_j) \\
	y_j &= \sqrt{n_j} \exp(-\iunit \chi  _j) 	
	\end{cases}
	\label{eqn:DefXY}
\end{align}

This in eqn. (\ref{eqn:BSA_IntAbsorbed}) gives:
\begin{multline}
	A_F(\vec{m}, \vec{n})
=
	\qty(
		\prod_{j=1}^{K}
		\frac
			{1}
			{-4\pi^{2}}
		\sqrt{\frac
			{m_j!      \; n_j!}
			{m_j^{m_j} \; n_j^{n_j} }
		}
	)
	\int_{0}^{2\pi} \dd{\vec{\theta}}
	\int_{0}^{2\pi} \dd{\vec{\chi}}\ \ldots \\
		\ldots		
		\exp[\iunit (\vec{n} \cdot \vec{\chi}  -  \vec{m} \cdot \vec{\theta})] \cdot 
		\exp[\sum_{k,l=1}^{K} \sqrt{m_k n_l} \; u_{k,l} \; \exp[\Big.\iunit(\theta_k - \chi_l)] ]
\end{multline}

Symbols
\begin{itemize}
\item as before
\end{itemize}

Holds because:
\begin{itemize}
\item Choose path $\gamma(t) = \exp(\iunit t)$ where $t \thus \theta_j$ or $t \thus \chi_j$,
	respectively.
\item Recall complex line integrals:
	\begin{itemize}
	\item $\int_{\gamma} \dd{\vec{r}} f(\vec{r})
		=
		\int_{a}^{b} \dd{t} f[\gamma(t)] \; \gamma'(t)$
	\item $\gamma'(t) 
		\thus 
	 	\qty( \prod_{j=1}^{K} \sqrt{m_j n_j} )
	 	\exp[\iunit(\sum_{j=1}^{K} \theta_j - \chi_j)]$
	\end{itemize}
\item Substituting $\vec{x}, \vec{y}$ as specified gives:
	\begin{itemize}
	\item[*] $\vec{x} \, \mathbb{U} \, \vec{y}$
		\thus
		$\sum_{k,l} \sqrt{m_k n_l} \; u_{k,l} \; \exp[\iunit(\theta_k - \chi_l)]$
	\item $\dd{x_j} \dd{y_j} \thus \dd{\theta_j} \dd{\chi_j}$
	\item $\exp[-(m_j + 1) \log(x_j)]$
		\thus
		$\exp[-(m_j + 1) (\frac{1}{2}\log(m_j) + \iunit \theta_j)]$\\
		$\thus
		 \exp[-(m_j + 1) (\frac{1}{2}\log(m_j)] \exp[ -\iunit (m_j + 1) \theta_j)]
		$
	\item Likewise (with $-\iunit \chi_j$) for $\vec{y}$
	\item $\exp(-\frac{a}{2} \log(x)) = x^{-\frac{a}{2}} = \sqrt{\frac{1}{x^{a}}}$
	\item[\Thus] $\exp[-(m_j + 1) \log(x_j)]$
		\thus
		$\sqrt{\frac{1}{ m_j^{m_j + 1}}} \exp[-\iunit (m_j + 1) \theta_j]$
	\item[\Thus] $\sqrt{m_j + 1}$ absorbs by prefactor $\sqrt{m_j}$ from $\gamma'$
		\thus $\sqrt{\frac{1}{ m_j^{m_j}}} \exp[-\iunit (m_j + 1)\theta_j]$
	\item[*] Hence the complete prefactor 
		$\sqrt{\frac
			{m_j!      \; n_j!}
			{m_j^{m_j} \; n_j^{n_j} }}$
	 \item $\exp[-\iunit (m_j + 1)\theta_j]$ together with $\exp$-part from $\gamma'$ gives
	 	$\exp(-\iunit m_j \theta_j)$
	 \item[*] Hence $\exp[\iunit(\vec{n} \cdot \vec{\chi}  -  \vec{m} \cdot \vec{\theta})]$
	\end{itemize}
\end{itemize}


\section{Integrating the $\theta_1$ DoF}
\label{sec:Inttheta1}
Statement:
\begin{multline}
	A_F(\vec{m}, \vec{n})
=
	2 \pi \delta_{M, N} \; C
	\int_{0}^{2\pi} \dd{\tilde{\vec{\theta}}}
	\int_{0}^{2\pi} \dd{\vec{\chi}} \ldots \\
		\ldots
		\exp[ \Bigg. -\iunit(\vec{m} \cdot \tilde{\vec{\theta}}  +  \vec{n} \cdot \vec{\chi})] \cdot
		\exp[ \sum_{k,l=1}^{K} \sqrt{m_k n_l} \; u_{k,l} \; 
			\exp[ \Big. \iunit(\tilde{\theta}_k - \chi_l)] 
		]
\label{eqn:AmplitudePreSPA}
\end{multline}

Symbols:
\begin{itemize}
\item $C = \prod_{j=1}^{K} 
	\frac{1}{-4\pi^{2}}
	\sqrt{\frac
			{m_j!      \; n_j!}
			{m_j^{m_j} \; n_j^{n_j} }}$:
	Amplitude normalization factor
\item $\tilde{\vec{\theta}} = (0, \theta_2, \ldots, \theta_K)\transp$:
	Phases to incoming photons, with $\theta_1$ arbitrarily set to zero. \newline
	Note that $\dd{\tilde{\vec{\theta}}}$ implies only $K-1$ integrals!
\item $M = \sum_{j=1}^{K} m_j$: Total incoming photons
\item $N = \sum_{j=1}^{K} n_j$: Total outgoing photons
\end{itemize}

Holds because:
\begin{itemize}
\item Can do a transformation of variables:
	\begin{itemize}
	\item $\vec{\theta'} = \vec{\alpha'} + 
			\underbrace{(\theta_1, \ldots, \theta_1)\transp}_{K-1 \text{ components}}
		\qquad
			\vec{\chi}    = \vec{\beta}   + 
			\underbrace{(\theta_1, \ldots, \theta_1)\transp}_{K \text{ components}}
		$
	\item Primed vars are $(K-1)$-dimensional, others are $K$-dimensional
	\item Allows separately doing the $\theta_1$ integral
	\end{itemize}
\item Rewriting the int from \ref{sec:BSA_phases} gives contribution
	$\exp[-\iunit(\sum_{k=1}^{K} m_k - \sum_{l=1}^{K} n_l)\theta_1]$
\item Integrating this wrt. $\theta_1$ gives $2\pi \delta_{M,N}$
\item Resubstituting old $\vec{\theta}, \vec{\chi}$ gives this result.
\item Note that this is now an int in $2K - 1$ dimensions.
\end{itemize}

\chapter{Applying the SPA}
\section{Generic Multidimensional SPA}
Statement
\begin{align}
	\int_{\gamma} \dd[n]{z}
		\exp[-\lambda f(\vec{z})]
&=
	\qty(
		\frac{2\pi}{\lambda}
	)^{\frac{n}{2}}
	\sum_j
		\frac
		{\exp[-\lambda f(\vec{z}_j)]}
		{\sqrt{\det S_j}}
\label{eqn:genericSPA}
\end{align}

Symbols:
\begin{itemize}
\item $\gamma$: closed loop in $\setComplex^{n}$, \ie $n$ is number of complex dimensions
\item $\lambda$: big constant (big wrt. parametrized length of $\gamma$)
\item $f$: arbitrary function matching the signature $\setComplex^{n} \thus \setComplex$
\item $\{ \vec{z}_j \}$: set of all zeros of $f$
\item $S_j = \eval{ \pdv{f(\vec{z})}{z_\alpha}{z_\beta}  }_{\vec{z} = \vec{z}_j} $:
	Hessian matrix of $f$, evaluated in root $\vec{z}_j$
\end{itemize}

Holds because:
\begin{itemize}
\item math says so.
\item 1D-form can be taken from lecture notes NLD
\item Should be possible to find this general form from there
\item However, in NLD we regarded only ints over $(-\infty; +\infty)$.
\item Possibly similar to: ints along the real axis can be solved using residue theorem
\end{itemize}

\section{Apply SPA to BSA}
Approach:
\begin{align}
\nonumber
	f(\vec{z})
&=
	f(\vec{\theta}, \vec{\chi}) \\
&=
	\underbrace{
		\qty( \prod_{j=1}^{K}
			\frac{1}{m_j n_j}
		)
	}_{\lambda^{-1}}
	\qty[
		\iunit(\vec{m} \cdot \vec{\theta} - \vec{n} \cdot \vec{\chi})
		-
		\sum_{k,l=1}^{K}
			\sqrt{m_k n_l} \; u_{k,l} \; \exp[ \Big. \iunit(\theta_k - \chi_l)]
	]
\label{eqn:DefF}
\\
	\lambda
&=
	\qty( \prod_{j=1}^{K} m_j n_j)
\label{eqn:DefLambda}
\end{align}

\newpage
Symbols:
\begin{itemize}
\item $\vec{\theta}$ is $\tilde{\vec{\theta}}$ from \ref{sec:Inttheta1} (renamed for convenience).
\item Note that $\vec{\theta}$ is $K$-dimensional, but $\theta_1 = 0$, \ie we got only $2K - 1$ DoF!
\end{itemize}


\subsection{Saddle Points}
\label{sec:Saddles}
Require:
\begin{align}
	\pdv{f}{\theta_p} &\equalCond 0
&
	\pdv{f}{\chi  _q} &\equalCond 0 
\end{align}
with $p \in \{2, \ldots, K\}$ and $q \in \{1, \ldots, K\}$

Find:
\begin{gather}
	\pdv{f}{\theta_p}
=
	\lambda^{-1}
	\qty[
		\iunit m_p
		-
		\sum_{l=1}^{K}
			\sqrt{m_p n_l} \; u_{p,l} \; \iunit \exp[ \Big. \iunit(\theta_p - \chi_l)]
	]
	\label{eqn:Jacobiantheta}
\equalCond
	0 \\
\Thus
	m_p
\equalCond
	\sum_{l=1}^{K}
		\sqrt{m_p n_l} \; u_{p,l} \; \exp[ \Big. \iunit(\theta_p - \chi_l) ] \\
\Thus
	\sqrt{m_p} \exp(-\iunit \theta_p)
\equalCond
	\sum_{l=1}^{K} \sqrt{n_l} \; u_{p,l} \; \exp(-\iunit\chi_l)
	\label{eqn:FourierLinkForward}
\end{gather}
with $p \in \{2, \ldots, K\}$ where $\theta_1 = 0$. Implicitly also holds for $p=1$ {\color{red}(Does it?)}, but the derivatives only start at $p=2$

Likewise:
\begin{gather}
	\pdv{f}{\chi_q}
=
	\lambda^{-1}
	\qty[
		-\iunit n_q
		+
		\sum_{k=1}^{K}
			\sqrt{m_k n_q} \; u_{k,q} \; \iunit \exp[ \Big. \iunit(\theta_k - \chi_q) ]
	]
	\label{eqn:JacobianChi}
\equalCond
	0 \\
\Thus
	\sqrt{n_q} \exp(+\iunit \chi_q)
\equalCond
	\sum_{k=1}^{K}
	\sqrt{m_k} \; u_{k,q} \; \exp(+\iunit\theta_k)
	\label{eqn:FourierLinkBackward}
\end{gather}
with $q \in \{1, \ldots, K\}$ where again $\theta_1 = 0$

Compactly:
\begin{align}
	\vec{x}^{*} &= \mathbb{U} \vec{y}
&
	\vec{y}^{*} &= \mathbb{U}\transp \vec{x}
	\label{eqn:MatrixCondition}
\end{align}

Compare: my B-thesis has symbols:
\begin{align}
	\vec{\Phi} &= \mathbb{U}\transp \vec{\Psi}
\end{align}
{\color{red} and missing transpose sign! Check code!}

where
\begin{align}
	\vec{\Psi} =
	\vec{x}
	&= \begin{pmatrix}
		\sqrt{m_1} \exp(\iunit \theta_1) \\
		\vdots                           \\
		\sqrt{m_K} \exp(\iunit \theta_K) \\
	\end{pmatrix}
&
	\vec{\Phi} =
	\vec{y}^{*}
	&= \begin{pmatrix}
		\sqrt{n_1} \exp(\iunit \chi_1) \\
		\vdots                           \\
		\sqrt{n_K} \exp(\iunit \chi_L) \\
	\end{pmatrix}
\end{align}
and
\begin{align}
	\theta_1 &= 0
&
	\theta_2, \ldots, \theta_K, \chi_1, \ldots, \chi_L \in [0, 2\pi)
\end{align}
(\ie explicitly including the $\theta_1$ dimension with its constraint $\theta_1 = 0$.)

\subsection{Hessian}
Recall: We start from
\begin{align}
	S_j
=
	\eval{
		\pdv{f(\vec{z})}%
			{z_\alpha}{z_\beta}
	}_{\vec{z} = \vec{z}_j}
\end{align}
but only true DoFs are relevant. Hence, regard $m_j, n_j$ and $\theta_1$ as const. \newline
\Thus $(2K - 1)^{2}$ derivatives wrt. $\theta_2, \ldots \theta_K, \chi_1, \ldots, \chi_K$

Compactly: Use $\vec{z}_j = \vec{\theta} \oplus \vec{\chi}$, \ie the entire Fourier pair. Re-use eqns (\ref{eqn:Jacobiantheta}) and (\ref{eqn:JacobianChi}) from \ref{sec:Saddles} by introducing (remembering the constraint $\theta_1 = 0$):
\begin{align}
	\mathbb{J} = \qty( \grad_{\vec{z}_j} f )\transp
\end{align}

\begin{align}
	J_{\theta, p} 
&= 
	\pdv{f}{\theta_p}
=
	\lambda^{-1}
	\qty[
		+\iunit m_p
		-
		\sum_{l=1}^{K}
			\sqrt{m_p n_l} \; u_{p,l} \; \iunit \exp[\Big. \iunit(\theta_p - \chi_l)]
	]
\\
	J_{\chi, q} 
&= 
	\pdv{f}{\chi_q}
=
	\lambda^{-1}
	\qty[
		-\iunit n_q
		+
		\sum_{k=1}^{K}
			\sqrt{m_k n_q} \; u_{k,q} \; \iunit \exp[\Big. \iunit(\theta_k - \chi_q)]
	]
\end{align}
Understand $\mathbb{J}$ as a \emph{row vector}. In spite of it having two indices here, there is only one dimension. $\theta, \chi$ refer to subblocks in the vector rather than degrees of freedom.

Total dimension of $\mathbb{J}$ is $1 \times (2K - 1)$ due to constraint $\theta_1 = 0$

Get Hessian matrix elements:
\begin{align}
	\qty( S_j )
&=
	\begin{pmatrix}
		\pdv{J_{\theta}}{\vec{\theta}} &
		\pdv{J_{\theta}}{\vec{\chi  }} 
		\\
		\pdv{J_{\chi  }}{\vec{\theta}} &
		\pdv{J_{\chi  }}{\vec{\chi  }} 
	\end{pmatrix}
\\
	\pdv{J_{\theta, p}}{\theta_{\alpha}}
&=
	+
	\lambda^{-1}
	\delta_{p,\alpha}
	\sum_{l=1}^{K}
		\sqrt{m_p n_l} \; u_{p,l} \; \exp[\Big. \iunit(\theta_p - \chi_l)]
\\
	\pdv{J_{\theta, p}}{\chi_{\alpha}}
&=
	-
	\lambda^{-1}
	\sqrt{m_p n_\alpha} \; u_{p,\alpha} \; \exp[\Big. \iunit(\theta_p - \chi_\alpha)]
\\
	\pdv{J_{\chi, q}}{\theta_{\alpha}}
&=
	-
	\lambda^{-1}
	\sqrt{m_\alpha n_q} \; u_{\alpha,q} \; \exp[\Big. \iunit(\theta_\alpha - \chi_q)]
\\
	\pdv{J_{\chi, q}}{\chi_{\alpha}}
&=
	+
	\lambda^{-1}
	\delta_{q,\alpha}
	\sum_{k=1}^{K}
		\sqrt{m_k n_q} \; u_{k,q} \; \exp[\Big. \iunit(\theta_k - \chi_q)]
\end{align}

Note how the signs comes from $-1 \cdot \iunit^{2}$ (diagonal blocks) and from 
$\pm\iunit \cdot (\pm\iunit)$ (off-diagonal blocks).

The Kronecker Delta is due to the fact that the derived-for variable is present only once at the stage of Jacobian $\mathbb{J}$.

Using the eqns (\ref{eqn:FourierLinkForward}) and (\ref{eqn:FourierLinkBackward}), we can further collapse the diagonal elements:
\begin{align}
	\pdv{J_{\theta, p}}{\theta_{\alpha}}
&=
	\lambda^{-1}
	\delta_{p,\alpha}
	m_p
&
	p \in \{2, \ldots, K\}
\\
	\pdv{J_{\chi, q}}{\chi_{\alpha}}
&=
	\lambda^{-1}
	\delta_{q,\alpha}
	n_q
&
	q \in \{1, \ldots, K\}
\end{align}

Thus:
\begin{align}
	(S_j)
&=
	\lambda^{-1}
	\diag(\vec{m}', \vec{n})
	+
	\begin{pmatrix}
		\mathds{O}_{K-1} & 
		\pdv{J_{\theta, p}}{\chi_{\alpha}}
		\\
		\pdv{J_{\chi, q}}{\theta_{\alpha}} &
		\mathds{O}_{K}
	\end{pmatrix}
=
	\lambda^{-1}
	\begin{pmatrix}
		\diag(\vec{m}') & F 			\\
		F\transp & \diag(\vec{n})
	\end{pmatrix}
\end{align}
where
\begin{align}
	F
&:=
	\lambda \qty( \pdv{J_{\theta}}{\vec{\chi}} ) 
	= \lambda \qty( \pdv{J_{\chi}}{\vec{\theta}} )\transp
	= \qty( -\sqrt{m_p n_q} \; u_{p,q} \; \exp[\Big. \iunit(\theta_p - \chi_q)] )_{
		\substack{p = 2, \ldots, K \\ q = 1, \ldots, K}
	}	
\label{eqn:defDetOffDiagonalF}
\end{align}
where $\mathds{O}_{D}$ is a $D \times D$ null-matrix, and $\vec{m}' = (m_2, \ldots, m_K)\transp$

The implied symmetry of the off-diagonal blocks is guaranteed by this being a Hessian matrix.

Note:
\begin{itemize}
\item $(S_j)$ is of dimension $(2K-1)^{2}$.
\item $F$ is of dimension $(K - 1) \times K$
\end{itemize}


\section{Determinant of the Hessian}
\subsection{Expansion to $K \times K$}
Do expansion:
\begin{align}
	S
=
	\lambda^{-1}
	\begin{pmatrix}
		\diag(\vec{m}') & F 			\\
		F\transp & \diag(\vec{n})
	\end{pmatrix}
\quad\thus\quad
	\lambda^{-1}
	\begin{pmatrix}
		\lambda		& 0					& 0			\\
		0			& \diag(\vec{m}')	& F			\\
		0			& F\transp			& \diag(\vec{n})
	\end{pmatrix}
\end{align}
(where the $0$ objects are such to match the rows/columns of the total matrix). While preserving the determinant, this new object is $K \times K$ again.

Get:
\begin{align}
	\det S
&=
	\lambda^{-2K+1}
	\underbrace{\mqty|
		1	& 0					& 0			\\
		0	& \diag(\vec{m}')	& F			\\
		0	& F\transp			& \diag(\vec{n})
	|}_{= S'}
=
	\lambda^{-2K+1}
	\mqty|
		\diag(\tilde{\vec{m}})	& \tilde{F}			\\
		\tilde{F}\transp			& \diag(\vec{n})
	|
\end{align}
explicit:
\begin{itemize}
\item $\tilde{\vec{m}} = (1, m_2, \ldots, m_K)\transp$ is of dimension $K$
\item $\tilde{F} = \begin{pmatrix}
		0 \\ F
	\end{pmatrix}$
	is of dimension $K \times K$
\end{itemize}

\subsection{Alt Form}
Statement:
\begin{align}
	\det S'
&=
	\mqty|
		\underbrace{\begin{pmatrix}
			\diag(\tilde{\vec{m}})	& \mathds{O}			\\
			\tilde{F}\transp			& \mathds{1}
		\end{pmatrix}}_{2K \times 2K}
		\cdot
		\underbrace{\begin{pmatrix}
			\mathds{1}	&	\qty(\diag(\tilde{\vec{m}}))^{-1} \tilde{F}		\\
			\mathds{O}	&	\diag(\vec{n}) - \tilde{F}\transp \qty(\diag(\tilde{\vec{m}}))^{-1}\tilde{F}
		\end{pmatrix}}_{2K \times 2K}
	|
	\label{eqn:detProto}
\end{align}
holds because:
\begin{itemize}
\item simply doing the matrix multiplication yields identity
\item interpret this as a $2 \times 2$ matrix multiplication
\item all elements in these two matrices are $K \times K$ \Thus well defined product
\end{itemize}

\subsection{Partial evaluation of det}
Statement
\begin{align}
	\det S'
&=
	\underbrace{
		\qty( \prod_{j=1}^{K} \tilde{m}_j n_j )
	}_{= \mathcal{N} }
	\det[
		\mathds{1} -
		\tilde{F}\transp \cdot \qty( \diag(\tilde{\vec{m}}) )^{-1}
		\cdot
		\tilde{F} \cdot \qty( \diag(\vec{n}) )^{-1}
	]
\label{eqn:detSPrimePartial}
\end{align}
Note: $\lambda \neq \mathcal{N}$, but $\lambda = m_1 \mathcal{N}$

Holds because:
\begin{itemize}
\item product separation in det: $\det(A \cdot B) = \det(A) \det(B)$
\item both matrices are of triangular form. the determinant of triangular matrices is the product of
	its diagonal elements.
\item first matrix yields $\det[ \diag(\tilde{\vec{m}}) ]$
\item second matrix yields 
	\begin{align*}
	\quad& \det[ \diag(\vec{n}) - \tilde{F}\transp \qty(\diag(\tilde{\vec{m}}))^{-1}\tilde{F} ] \\
	=    &  \det[\Big. \diag(\vec{n}) ] 
			\det[\Big. 
				\mathds{1} - 
				\tilde{F}\transp \cdot \qty( \diag(\tilde{\vec{m}}) )^{-1}
				\cdot
				\tilde{F} \cdot \qty( \diag(\tilde{\vec{n}}) )^{-1}
			]
	\end{align*}
%\item in second matrix: multiply from right with 
%	$\mathds{1} = \diag(\vec{n})^{-1} \cdot \diag(\vec{n})$
%\item again separate factor $\diag(\vec{n})$
%\item combine first matrix contribution and separated second matrix contribution to get $\mathcal{N}$
\end{itemize}

\subsection{Reforming $\tilde{F}$}
Statement:
\begin{align}
	\tilde{F}
&=
	\diag\qty[ \sqrt{ \tilde{\vec{m}}} * \exp(+\iunit \vec{\theta}) ]
	\cdot \tilde{\mathbb{U}} \cdot
	\diag\qty[ \sqrt{        \vec{n} } * \exp(-\iunit \vec{\chi  }) ]
\end{align}
(where $*$ denotes component-wise multiplication. Likewise, $\sqrt{.}$ and $\exp(.)$ are understood to act on the components of the vectors individually.)

Where:
\begin{equation}
	\tilde{u}_{i,j} = \begin{cases}
		-u_{i,j}		&\qq*{for } i > 1 \\
		0			&\qotherwise*
	\end{cases}
\end{equation}
Note that this suppresses the necessity of the condition $\theta_1 = 0$.

Holds because:
\begin{itemize}
\item see definition of F in eqn. (\ref{eqn:defDetOffDiagonalF})
\end{itemize}

\subsection{Apply reformed $\tilde{F}$}
Statement:
\begin{align}
	\det S'
&=
	\mathcal{N}
	\qty( \prod_{j=1}^{K} \exp(-2\iunit \chi_j) )
	\det\qty[\bigg.
		\diag\qty(\exp(\Big.2\iunit\vec{\chi}))
		-
		\tilde{\mathbb{U}}\transp
		\diag\qty(\exp(2\iunit\vec{\theta}))
		\tilde{\mathbb{U}}
	]
\end{align}
where, again, $\exp(.)$ is understood to act on the individual components of a vector.

Holds because:
\begin{itemize}
\item $(AB)\transp = B\transp A\transp$
\item $\tilde{F}
=
	\diag\qty[ \sqrt{ \tilde{\vec{m}}} * \exp(+\iunit \vec{\theta}) ]
	\cdot \tilde{\mathbb{U}} \cdot
	\diag\qty[ \sqrt{        \vec{n} } * \exp(-\iunit \vec{\chi  }) ]$
	(cf. eqn. (\ref{eqn:defDetOffDiagonalF}))
\item $\tilde{F}\transp
=
	\diag\qty[ \sqrt{        \vec{n} } * \exp(-\iunit \vec{\chi  }) ]
	\cdot \tilde{\mathbb{U}}\transp \cdot
	\diag\qty[ \sqrt{ \tilde{\vec{m}}} * \exp(+\iunit \vec{\theta}) ]$
\item start from eqn. (\ref{eqn:detSPrimePartial}):
	\begin{align*}
		\det S'
	&=
		\underbrace{
			\qty( \prod_{j=1}^{K} \tilde{m}_j n_j )
		}_{= \mathcal{N} }
		\det[
			\mathds{1} -
			\tilde{F}\transp \cdot \qty( \diag(\tilde{\vec{m}}) )^{-1}
			\cdot
			\tilde{F} \cdot \qty( \diag(\vec{n}) )^{-1}
		]
	\end{align*}
\item contributions $\propto \sqrt{\tilde{\vec{m}}}, \sqrt{\vec{n}}$ cancel out due to inverses in eqn.
	(\ref{eqn:detSPrimePartial})
\item multiply from the right: $\exp(+2\iunit \vec{\chi})$ and its inverse from the left.
\item separate emerging $\det\qty(\diag(\exp(-2\iunit \vec{\chi})))$ to get phase prefactor in Statement and transform unity matrix into minuend in Statement.
\end{itemize}

\subsection{Complete Det of the Hessian}
Statement:
\begin{align}
	\det S
&=
	\frac{ \lambda^{-2K+2} }{ m_1 }
	\qty( \prod_{j=1}^{K} \exp(-2\iunit \chi_j) )
	\det\qty[\bigg.
		\diag\qty(\exp(\Big.2\iunit\vec{\chi}))
		-
		\tilde{\mathbb{U}}\transp
		\diag\qty(\exp(2\iunit\vec{\theta}))
		\tilde{\mathbb{U}}
	]
\label{eqn:detComplete}
\end{align}
For use in code:
\begin{multline*}
	\det S
=
	\frac{1}{m_1}
	\qty( \prod_{j=1}^{2} m_j n_j)^{-2}
	\qty( \prod_{j=1}^{2} \exp(-2\iunit \chi_j) )
\\
	\qty[
		\exp( \Big. 2\iunit (\chi_1 + \chi_2) )
		-\frac{1}{2} \qty[ \Big.
			\exp[2\iunit (\theta_2 + \chi_1)] +
			\exp[2\iunit (\theta_2 + \chi_2)]
		]
		-
		\frac{1}{4}\exp( \Big. 4\iunit \theta_2 )
	]
\end{multline*}

Holds because:
\begin{itemize}
\item simply plug back in everything
\item $\lambda = m_1 \mathcal{N}$ 
	and thus 
	$\mathcal{N} \lambda^{-2K+1} = \frac{ \lambda^{-2K+2} }{ m_1 }$
\item \color{red} make sure you always give proper lower/upper boundaries in your thesis paper. It certainly was no fun re-doing this over and over to find out when $m_1$ is part of the product and when it is not
\end{itemize}

Evaluation for in-code formula:
\begin{align*}
	\tilde{\mathbb{U}}
&=
	\frac{1}{\sqrt{2}}
	\begin{pmatrix}
		 0 &  0 \\
		+1 & -1
	\end{pmatrix}
&
	\tilde{\mathbb{U}}\transp
&=
	\frac{1}{\sqrt{2}}
	\begin{pmatrix}
		0 & +1 \\
		0 & -1
	\end{pmatrix}
\end{align*}
\begin{align*}
	\tilde{\mathbb{U}}\transp
	\diag\qty[\exp(2\iunit\vec{\theta})]
	\tilde{\mathbb{U}}
&=
	\frac{1}{\sqrt{2}}
	\begin{pmatrix}
		0 & +1 \\
		0 & -1
	\end{pmatrix}
	\cdot
	\begin{pmatrix}
		1 & 0 \\
		0 & \exp(2\iunit \theta_2)
	\end{pmatrix}
	\cdot
	\frac{1}{\sqrt{2}}
	\begin{pmatrix}
		 0 &  0 \\
		+1 & -1
	\end{pmatrix} \\
&=
	\frac{1}{2}
	\begin{pmatrix}
		0 & +\exp(2\iunit \theta_2) \\
		0 & -\exp(2\iunit \theta_2)
	\end{pmatrix}
	\cdot
	\begin{pmatrix}
		 0 &  0 \\
		+1 & -1
	\end{pmatrix} \\
&=
	\frac{1}{2}
	\begin{pmatrix}
		+\exp(2\iunit \theta_2)	&	-\exp(2\iunit \theta_2) \\
		-\exp(2\iunit \theta_2)	&	+\exp(2\iunit \theta_2)
	\end{pmatrix}
\\
	\diag\qty[\exp(\Big.2\iunit\vec{\chi})]
	-
	\tilde{\mathbb{U}}\transp
	\diag\qty[\exp(2\iunit\vec{\theta})]
	\tilde{\mathbb{U}}
&=
	\begin{pmatrix}
		\exp(2\iunit \chi_1) - \frac{1}{2}\exp(2\iunit \theta_2)	&
		+\frac{1}{2}\exp(2\iunit \theta_2)
		\\
		+\frac{1}{2}\exp(2\iunit \theta_2)	&
		\exp(2\iunit \chi_2) - \frac{1}{2}\exp(2\iunit \theta_2)
	\end{pmatrix}
\end{align*}
\begin{align*}
	\det(\ldots)
&=
	\qty[ \exp(2\iunit \chi_1) - \frac{1}{2}\exp(2\iunit \theta_2) ]
	\qty[ \exp(2\iunit \chi_2) - \frac{1}{2}\exp(2\iunit \theta_2) ]
	-
	\frac{1}{2}\exp(2\iunit \theta_2)
	\frac{1}{2}\exp(2\iunit \theta_2) \\
&=
	\exp[ \Big. 2\iunit (\chi_1 + \chi_2) ]
	-\frac{1}{2} \qty[ \Big.
		\exp[2\iunit (\theta_2 + \chi_1)] +
		\exp[2\iunit (\theta_2 + \chi_2)]
	]
	-
	\frac{1}{4}\exp(4\iunit \theta_2)
\end{align*}

\section{Case 2x2}
Start from eqn. (\ref{eqn:FourierLinkForward}) and (\ref{eqn:FourierLinkBackward}) with $K=2$:
\begin{align}
	\sqrt{m_p} \exp(-\iunit \theta_p)
&\equalCond
	\sum_{k=1}^{2} \sqrt{n_k} \; u_{p,k} \; \exp(-\iunit\chi_k)
&
	\sqrt{n_l} \exp(+\iunit \chi_l)
&\equalCond
	\sum_{k=1}^{2}
	\sqrt{m_k} \; u_{k,l} \; \exp(+\iunit\theta_k)
\label{eqn:start2x2}
\end{align}
and regard cases $p=2, l = 1, 2$ with $\theta_1 = 0$ as usual.

Define
\begin{align}
	\mu := \exp(\iunit \theta_2)
&&
	\mu^{*} := \exp(-\iunit \theta_2) = \frac{1}{\mu}
\label{eqn:DefMu}
\end{align}

\subsection{Mod-Squared of eqn. (\ref{eqn:start2x2})}
Statement:
\begin{align}
	\underbrace{\sqrt{m_1 m_2} \,
		u_{1, l}^{*} \, u_{2, l}
	}_{a}
	\, \mu^{2}
	+
	\underbrace{
		(m_1 |u_{1,l}|^{2}  +  m_2 |u_{2,l}|^{2} - n_l)
	}_{b}
	\, \mu
	+
	\underbrace{
		\sqrt{m_1 m_2} \,
		u_{1, l} \, u_{2, l}^{*}
	}_{c}
=
	0
\end{align}
Note that $\mu^{2} \neq |\mu|^{2}$ -- this is still a complex value!


Holds because: (explicitly do multiplication)
\begin{align}
	\qty( \Big. \sqrt{n_l} \exp(+\iunit \chi_l) ) \,
	\qty( \Big. \sqrt{n_l} \exp(+\iunit \chi_l) )^{*}
&=
	\qty( \Big.
		\sqrt{m_1} \; u_{1,l} +
		\sqrt{m_2} \; u_{2,l} \mu
	)
	\qty( \Big.
		\sqrt{m_1} \; u_{1,l} +
		\sqrt{m_2} \; u_{2,l} \mu
	)^{*}
\\
	n_l
&=
	m_1 |u_{1,l}|^{2} +
	\sqrt{m_1 m_2} \, u_{1,l}     \, u_{2,l}^{*} \, \mu^{*} +  \\
\nonumber
&\qquad
	m_2 |u_{2,l}|^{2} \, \mu\mu^{*} +
	\sqrt{m_1 m_2} \, u_{1,l}^{*} \, u_{2,l}     \, \mu
\qquad
	| \olddiv \mu^{*} \Leftrightarrow \cdot \mu
\\
	n_l \mu
&=
	m_1 |u_{1,l}|^{2} \, \mu +
	\sqrt{m_1 m_2} \, u_{1,l}     \, u_{2,l}^{*} +  \\
\nonumber
&\qquad
	m_2 |u_{2,l}|^{2} \, \mu +
	\sqrt{m_1 m_2} \, u_{1,l}^{*} \, u_{2,l} \mu^{2}
\end{align}
and rearrange by orders of $\mu$.

\subsection{Quadratic Formula Solution}
Rem:
\begin{align}
	a z^{2} + b z + c 
&\equalCond 
	0
&\Thus&&
	z_{\pm} 
&= 
	\frac
		{-b \pm \sqrt{b^{2} - 4ac}}
		{2a}
=
	\frac
		{-b \pm \sqrt{D}}
		{2a}
\end{align}
gives:
\begin{align}
	\mu_{\pm}^{(l)}
&=
	\frac
	{
		- m_1 |u_{1,l}|^{2}  -  m_2 |u_{2,l}|^{2} + n_l
		\pm
		\sqrt{
			(m_1 |u_{1,l}|^{2}  +  m_2 |u_{2,l}|^{2} - n_l)^{2}
			-
			4 m_1 m_2 |u_{1,l}|^{2} |u_{2,l}|^{2}
		}
	}
	{ 2\sqrt{m_1 m_2} \, u_{1,l}^{*} \, u_{2,l} }
\label{eqn:muQuadraticProto}
\end{align}
by simply plugging in the coefficients.

\subsection{Universality Condition}
Statement:
\begin{equation}
	\mu_{\pm}^{(1)} \equalCond \mu_{\mp}^{(2)}
\end{equation}

Holds because:

Require
\begin{align}
	\mu^{(1)} &\equalCond \mu^{(2)}		&& \qq{universality} \\
	m_1 + m_2 &\equalCond n_1 + n_2		&& \qq{particle conservation}
\end{align}

Get:
\begin{multline}
	\frac
	{
		- m_1 |u_{1,1}|^{2}  -  m_2 |u_{2,1}|^{2} - n_1
		\pm
		\sqrt{
			(m_1 |u_{1,1}|^{2}  +  m_2 |u_{2,1}|^{2} - n_1)^{2}
			-
			4 m_1 m_2 |u_{1,1}|^{2} |u_{2,1}|^{2}
		}
	}
	{ 2\sqrt{m_1 m_2} \, u_{1,1}^{*} \, u_{2,1} }
= \ldots \\
	\frac
	{
		- m_1 |u_{1,2}|^{2}  -  m_2 |u_{2,2}|^{2} - n_2
		\pm
		\sqrt{
			(m_1 |u_{1,2}|^{2}  +  m_2 |u_{2,2}|^{2} - n_2)^{2}
			-
			4 m_1 m_2 |u_{1,2}|^{2} |u_{2,2}|^{2}
		}
	}
	{ 2\sqrt{m_1 m_2} \, u_{1,2}^{*} \, u_{2,2} }
\end{multline}

Multiply both sides by $2\sqrt{m_1 m_2}$ to get:
\begin{multline}
	\frac
	{
		- m_1 |u_{1,1}|^{2}  -  m_2 |u_{2,1}|^{2} - n_1
		\pm
		\sqrt{
			(m_1 |u_{1,1}|^{2}  +  m_2 |u_{2,1}|^{2} - n_1)^{2}
			-
			4 m_1 m_2 |u_{1,1}|^{2} |u_{2,1}|^{2}
		}
	}
	{ u_{1,1}^{*} \, u_{2,1} }
= \ldots \\
	\frac
	{
		- m_1 |u_{1,2}|^{2}  -  m_2 |u_{2,2}|^{2} - n_2
		\pm
		\sqrt{
			(m_1 |u_{1,2}|^{2}  +  m_2 |u_{2,2}|^{2} - n_2)^{2}
			-
			4 m_1 m_2 |u_{1,2}|^{2} |u_{2,2}|^{2}
		}
	}
	{ \, u_{1,2}^{*} \, u_{2,2} }
\end{multline}

Use unitarity of $\mathbb{U}$: 
\begin{align}
	\mathbb{U}\mathbb{U}^{\dagger} &= \mathds{1}
&\Thus&&
	u_{1,1} \, u_{2,1}^{*} + u_{1,2} \, u_{2,2}^{*} &= 0 = \qty(\mathbb{U}\mathbb{U}^{\dagger})_{1,2} 
	\label{eqn:unitaryU12}
\\
&&\Leftrightarrow&&
u_{1,1} \, u_{2,1}^{*} &= -u_{1,2} \, u_{2,2}^{*}
\end{align}

Use this to reform RHS denominator and thus get rid of both sides denominators:
\begin{multline}
	- m_1 |u_{1,1}|^{2}  -  m_2 |u_{2,1}|^{2} - n_1
		\pm
		\sqrt{
			(m_1 |u_{1,1}|^{2}  +  m_2 |u_{2,1}|^{2} - n_1)^{2}
			-
			4 m_1 m_2 |u_{1,1}|^{2} |u_{2,1}|^{2}
		}
=\\
	\underbrace{+ m_1 |u_{1,2}|^{2}  +  m_2 |u_{2,2}|^{2} - n_2}_{(*)}
		\mp
		\sqrt{
			(\underbrace{m_1 |u_{1,2}|^{2}  +  m_2 |u_{2,2}|^{2} - n_2}_{(*)} )^{2}
			-
			4 m_1 m_2 \underbrace{|u_{1,2}|^{2} |u_{2,2}|^{2}}_{(\dagger)}
		}
\end{multline}

Again, unitarity:
\begin{align}
	\mathbb{U}\mathbb{U}^{\dagger} &= \mathds{1}
&\Thus
	\begin{cases}
		|u_{1,1}|^{2} + |u_{1,2}|^{2} = 1	& = \qty(\mathbb{U}\mathbb{U}^{\dagger})_{1,1}\\
		|u_{2,1}|^{2} + |u_{2,2}|^{2} = 1	& = \qty(\mathbb{U}\mathbb{U}^{\dagger})_{2,2}
	\end{cases}	
\end{align}
allows further reforming:
\begin{align}
	(*)
=
	m_1 |u_{1,2}|^{2}  +  m_2 |u_{2,2}|^{2} - n_2
&=
	m_1 \qty(1 - |u_{1,1}|^{2}) + m_2 \qty(1 - |u_{2,1}|^{2}) - n_2 \\
&=
	(m_1 + m_2 - n_2) - m_1 |u_{1,1}|^{2}  +  m_2 |u_{2,1}|^{2}\\
&=
	n_1 - m_1 |u_{1,1}|^{2}  +  m_2 |u_{2,1}|^{2}
\end{align}
where the last step uses particle conservation to collapse the parenthesis to $n_1$

Also from unitarity:
\begin{align}
	(\dagger) =
	|u_{1,2}|^{2} |u_{2,2}|^{2}
&=
	u_{1,2} \, u_{1,2}^{*} \; u_{2,2} \, u_{2,2}^{*} 
	&&\qq{unfold square modulus} \\
&=
	u_{1,2} \, u_{2,2}^{*} \; u_{2,2} \, u_{1,2}^{*}
	&&\qq{reorder multiplication}\\
&=
	u_{1,1} \, u_{2,1}^{*} \; u_{2,1} \, u_{1,1}^{*}
	&&\qq{use (\ref{eqn:unitaryU12}) (double minus cancels.)} \\
&=
	|u_{1,1}|^{2} |u_{2,1}|^{2}
\end{align}

Plug these back in and get the statement.

\subsection{Condition $\theta_2 \in \setReals$}
Statement:
\begin{align}
	|\mu_{\pm}^{l}|^{2} \equalCond 1
\end{align}

Holds because:
\begin{itemize}
\item Definition of $\mu = \exp(\iunit \Theta_2)$ in eqn. (\ref{eqn:DefMu})
\item Assume $\theta_2 = x + \iunit y \in \setComplex \Thus |\mu| = |\exp(\iunit x) \exp(-y)| \neq 1$
	if $y \neq 0$ 
\end{itemize}

\subsection{Apply Condition Real Theta}
Using symbols:
\begin{align}
	1 = \mu_{\pm}^{(l)} 
&=
	\frac
		{\alpha \pm \sqrt{D}}
		{\gamma}
\end{align}
\begin{itemize}
\item $\alpha, D \in \setReals$
\item $\gamma \in \setComplex$
\item See eqn. (\ref{eqn:muQuadraticProto}) for this.
\end{itemize}

Statement:
\begin{align}
	\qty| \mu_{\pm}^{(l)} |^{2}
&=
	\begin{cases}
		\frac
			{\alpha^{2} + D \, \pm \, 2 \alpha \sqrt{D}}
			{|\gamma|^{2}}
		& \qif D > 0
	\\
		\frac
			{\alpha^{2} + |D|}
			{|\gamma|^{2}}
		& \qif D \leq 0
	\end{cases}
\end{align}

Holds because:
\begin{itemize}
\item case $D > 0$ from direct product
\item $(\alpha \pm \iunit\sqrt{|D|}) \cdot (\alpha \mp \iunit\sqrt{|D|})
	=
	\alpha^{2} + |D|$
	(Binom III)
\end{itemize}

Yields statements:
\begin{itemize}
\item Case 1 yields contradiction $D \leq 0$
\item $\qty( \Big. m_1 |u_{1,l}|^{2} + m_2 |u_{2,l}|^{2} - n_l )^{2}
	<
	4 \, m_1 m_2 \, |u_{1,l}|^{2} |u_{2,l}|^{2}$
\end{itemize}
(see subsubsections below for reasons)


\subsubsection{Case $D > 0$}
Yields the contradiction $D = 0$.

Holds because:
\begin{itemize}
\item plug in the definitions and see that the result is equivalent to $D=0$
\item note that $\frac{D}{|\gamma|^{2}} = -1$, following the ideas marked $(*), (\dagger)$
\end{itemize}

\subsubsection{Case $D \leq 0$}
Statement:
\begin{align}
	\qty( \Big. m_1 |u_{1,l}|^{2} + m_2 |u_{2,l}|^{2} - n_l )^{2}
&<
	4 \, m_1 m_2 \, |u_{1,l}|^{2} |u_{2,l}|^{2}
\end{align}

Holds because:
\begin{itemize}
\item apply real-theta condition, \ie plug in definitions
\item get structure $1 = \frac{[\text{0: block from D > 0}] + 
	\qty( m_1 |u_{1,l}|^{2} + m_2 |u_{2,l}|^{2} - n_l )^{2}}
	{4 \, m_1 m_2 \, |u_{1,l}|^{2} |u_{2,l}|^{2}}
	=
	1$
\end{itemize}

\subsubsection{Question}
\begin{itemize}
\item {\color{red} what is this good for?}
\end{itemize}

\subsection{2x2 SPA-Amplitude}
\subsubsection{Update Amplitude Formula: Exponential Part}
Statement:
\begin{align}
	A_F
\propto
	\exp[ 
		\qty(- \sum_{k=2}^{K} \iunit m_k \theta_k) + 
		\qty(  \sum_{k=1}^{K} \iunit n_k \chi_k)   +
		\vec{x}\transp \, \mathbb{U} \, \vec{y}
	]
\end{align}
Note that implicitly, still $\theta_1 = 0$

Holds because:
Start from eqn. (\ref{eqn:AmplitudePreSPA}):
\begin{align*}
	A_F(\vec{m}, \vec{n})
=
	2 \pi \delta_{M, N} \; C
	\int_{0}^{2\pi} \dd{\tilde{\vec{\theta}}}
	\int_{0}^{2\pi} \dd{\vec{\chi}} 
		\exp[ \Bigg. -\iunit(\vec{m} \cdot \tilde{\vec{\theta}}  +  \vec{n} \cdot \vec{\chi}) 
		+ 
		\sum_{k,l=1}^{K} \sqrt{m_k n_l} \; u_{k,l} \; 
			\exp[ \Big. \iunit(\tilde{\theta}_k - \chi_l)] 
		]
\end{align*}

Recall definitions from eqn. (\ref{eqn:DefXY}):
\begin{align*}
	x_j &= \sqrt{m_j} \exp( \iunit \theta_j) \\
	y_j &= \sqrt{n_j} \exp(-\iunit \chi  _j)
\end{align*}

and immediately get the statement.

{\color{red} Note that I've wasted an entire day because p. 16 suggests this was about the determinant and could be derived starting from eqn. (42)... make sure your sources aren't ambiguous...}

\subsubsection{Reform Exponential Part}
Statement:
\begin{align}
	\vec{x}\transp
	\mathbb{U}
	\vec{y}
&=
	N
\end{align}

Holds because:
\begin{itemize}
\item eqn. (\ref{eqn:MatrixCondition}):
	$\vec{x}^{*} = \mathbb{U} \vec{y}$
\item $\vec{x}\transp \mathbb{U} \vec{y} = \vec{x}\transp \vec{x}^{*} = \sum_{j} m_j = M = N$
\end{itemize}


\subsubsection{Complete Expression for 2x2 case}
Statement:
\begin{multline}
	A_F(\vec{m}, \vec{n})
\approx
	\underbrace{
		\qty(\frac
			{2\pi}
			{\prod_{j=1}^{K} m_j n_j}
		)^{3/2}
	}_{\alpha}
	\underbrace{
	\qty(
		\prod_{j=1}^{2}
		\frac
			{1}
			{-4\pi^{2}}
		\sqrt{\frac
			{m_j!      \; n_j!}
			{m_j^{m_j} \; n_j^{n_j} }
		}
	)}_{C}
	\underbrace{2\pi \, \delta_{M,N}}_{\beta}
	\sum_{j}
		\underbrace{ \qty(\bigg. \mu_j^*)^{m_2} }_{\gamma}
	\ldots
\\
	\ldots
		\underbrace{
			\qty(\sqrt{\frac{m_1}{n_1}} u_{1,1}  +  \sqrt{\frac{m_2}{n_1}} u_{2,1} \mu_{j})^{n_1}
		}_{\omega_{1}}
		\underbrace{
			\qty(\sqrt{\frac{m_1}{n_2}} u_{1,2}  +  \sqrt{\frac{m_2}{n_2}} u_{2,2} \mu_{j})^{n_2}
		}_{\omega_{2}}
	\frac
		{\exp(N)}
		{\sqrt{\det S_j}}
\end{multline}

Holds because:
\begin{itemize}
\item Cf. eqn. (\ref{eqn:genericSPA}):
\begin{align*}
	\int_{\gamma} \dd[n]{z}
		\exp[-\lambda f(\vec{z})]
&=
	\underbrace{
		\qty(
			\frac{2\pi}{\lambda}
		)^{\frac{n}{2}}
	}_{\alpha}
	\sum_j
		\frac
		{\exp[-\lambda f(\vec{z}_j)]}
		{\sqrt{\det S_j}}
\end{align*}
From $K=2 \Thus n=2K-1 = 3$ (DoF $\theta_1$ was already int'ed out). Together with def of 
	$\lambda$ in eqn. (\ref{eqn:DefLambda}) we directly get $\alpha$

\item From eqn. (\ref{eqn:AmplitudePreSPA}):
\begin{align*}
	A_F(\vec{m}, \vec{n})
&=
	\underbrace{2 \pi \delta_{M, N}}_{\beta} \; C
	\int_{0}^{2\pi} \dd{\tilde{\vec{\theta}}}
	\int_{0}^{2\pi} \dd{\vec{\chi}} 
		\exp[ \Bigg. -\iunit(\vec{m} \cdot \tilde{\vec{\theta}}  +  \vec{n} \cdot \vec{\chi}) 
		+
		\underbrace{
			\sum_{k,l=1}^{K} \sqrt{m_k n_l} \; u_{k,l} \; 
				\exp[ \Big. \iunit(\tilde{\theta}_k - \chi_l)] 
		}_{N}
		]
	\\
&=
	C \beta
	\exp(N)
	\int_{0}^{2\pi} \dd{\tilde{\vec{\theta}}}
	\int_{0}^{2\pi} \dd{\vec{\chi}} 
		\exp[
			\Big. -\iunit(\vec{m} \cdot \tilde{\vec{\theta}}  +  \vec{n} \cdot \vec{\chi}) 
		]
\end{align*}

get:
\begin{align}
	\exp[
		-
		\iunit(\vec{m} \cdot \vec{\theta} - \vec{n} \cdot \vec{\chi})
	]
&=
	\underbrace{ \exp[ \Big. -\iunit m_2 \theta_2] }_{\qty(\mu^{*})^{m_2} = \gamma}
	\exp[ \Big. \iunit \vec{n} \cdot \vec{\chi} ]
\end{align}

\item Use eqn. (\ref{eqn:FourierLinkBackward}):
\begin{align*}
%	\sqrt{m_p} \exp(-\iunit \theta_p)
%&\equalCond
%	\sum_{l=1}^{K} \sqrt{n_l} \; u_{p,l} \; \exp(-\iunit\chi_l)
%&
	\sqrt{n_q} \exp(+\iunit \chi_q)
%\equalCond
	\sum_{k=1}^{K}
	\sqrt{m_k} \; u_{k,q} \; \exp(+\iunit\theta_k)
%
&&\Thus&&
%
	\exp(+\iunit \chi_q)
&=
	\sum_{k=1}^{K}
		\frac{\sqrt{m_k}}{\sqrt{n_q}}
		u_{k,q} \; \exp(+\iunit\theta_k)
\end{align*}
get
\begin{align}
\nonumber
	\exp[ \iunit \vec{n} \cdot \vec{\chi} ]
=
	\exp( \iunit \chi_1 )^{n_1} \exp( \iunit \chi_2 )^{n_2}
&=
	\qty(
		\frac{\sqrt{m_1}}{\sqrt{n_1}}
		u_{1,1} \;
		\underbrace{ \exp(\iunit\theta_1) }_{=1}
		+
		\frac{\sqrt{m_2}}{\sqrt{n_1}}
		u_{2,1} \;
		\underbrace{ \exp(+\iunit\theta_2) }_{= \mu}
	)^{n_1} \\
	& \quad
	\qty(
		\frac{\sqrt{m_1}}{\sqrt{n_2}}
		u_{1,2} \;
		\underbrace{ \exp(\iunit\theta_1) }_{=1}
		+
		\frac{\sqrt{m_2}}{\sqrt{n_2}}
		u_{2,2} \;
		\underbrace{ \exp(+\iunit\theta_2) }_{= \mu}
	)^{n_2} \\
&=
	\omega_1 \omega_2
\end{align}
\end{itemize}

\chapter{Exact BSA: Taylor instead of SPA}
\section{Analytical Simplifications}
\subsection{BSA as Taylor Series}
Statement:
\begin{align}
	A_F
&\propto
	\frac{1}{N!}
	\qty( \big. \vec{x}\transp \; \mathbb{U} \; \vec{y} )^{N}
\end{align}

Holds because:

\begin{itemize}
\item Start from eqn. (\ref{eqn:BSA_raw}):
\begin{align*}
	A_F(\vec{m}, \vec{n})
&=
	\eval{\qty(
		\prod_{j=1}^{K}
			\frac
				{1}
				{\sqrt{m_j! \; n_j!}}
			\pdv[m_j]{x_j}
			\pdv[n_j]{y_j}
		) \exp(\vec{x}\transp \, \mathbb{U} \, \vec{y})
	}_{\vec{x} = \vec{y} = \vec{0}}
\end{align*}
and write $\exp$ as its Taylor expansion:
\begin{align}
	\exp
	( \underbrace{\vec{x}\transp \, \mathbb{U} \, \vec{y}}_{a} )
&=
	T_{a_0 = 0}^{\exp} ( \vec{x}\transp \, \mathbb{U} \, \vec{y} )
=
	\sum_{k=1}^{\infty}
		\frac{1}{k!}
		\underbrace{ \exp( \vec{x}_0\transp \, \mathbb{U} \, \vec{y}_0 ) }_{=\exp(a_0) = 1}
		\qty( \vec{x}\transp \, \mathbb{U} \, \vec{y} - 0 )^{k}
	\\
&=
	\sum_{k=1}^{\infty}
		\frac{1}{k!}
		\qty( \vec{x}\transp \, \mathbb{U} \, \vec{y} )^{k}
\end{align}

\item apply derivatives:
\begin{align}
	\pdv[m_j]{x_j}
	\pdv[n_j]{y_j}
	\exp(\vec{x}\transp \, \mathbb{U} \, \vec{y})
&=
	\pdv[m_j]{x_j}
	\pdv[n_j]{y_j}
	\sum_{k=1}^{\infty}
		\frac{1}{k!}
		\qty( \vec{x}\transp \, \mathbb{U} \, \vec{y} )^{k}
	\\
&=
	\pdv[m_j]{x_j}
	\pdv[n_j]{y_j}
	\sum_{k=1}^{\infty}
		\frac{1}{k!}
		\qty( 
			\sum_{r=1}^{K}
			\sum_{c=1}^{L}
			x_c \; u_{r,c} \; y_r
		)^{k}
\end{align}
and note that summands vanish where exponent $k$ less than order of derivative $m_i, n_i$:
\begin{align}
	\pdv[m_j]{x_j}
	\pdv[n_j]{y_j}
	\exp(\vec{x}\transp \, \mathbb{U} \, \vec{y})
&=
	\pdv[n_j]{y_j}
	\sum^{\infty}_{k=m_j}
		\frac{m_j !}{k!}
			\sum_{r=1}^{K}
			\sum_{c=1}^{L}
		\qty(x_c)^{k - m_j}
		\qty(u_{r,c} \; y_r)^{k}
	\\
&=
	\sum^{\infty}_{k=\max(m_j, n_j)}
		\frac{m_j ! n_j !}{k!}
			\sum_{r=1}^{K}
			\sum_{c=1}^{L}
				\qty(x_c)^{k - m_j}
				\qty(y_r)^{k - n_j}
				\qty(u_{r,c})^{k}
\end{align}
get:
\begin{align}
	A_F
&=
	\prod_{j=1}^{K}
	\sum_{k=\max(m_j, n_j)}^{\infty}
		\frac
			{\sqrt{m_j! \; n_j!}}
			{k!}
		\qty( 
			\sum_{r=1}^{K}
			\sum_{c=1}^{L}
				\qty(x_c)^{k - m_j}
				\qty(y_r)^{k - n_j}
				\qty(u_{r,c})^{k}
		)
\end{align}

\item cf.:
\begin{align*}
	\qty( \big. \vec{x}\transp \; \mathbb{U} \; \vec{y} )^{N}
&=
	\qty( 
			\sum_{r=1}^{K}
			\sum_{c=1}^{L}
			x_c \; u_{r,c} \; y_r
		)^{N}
\end{align*}


\item {\color{red} I'm stuck here. I guess apply the product and neglect all summands other than $k=N$.
	With $\sum_j m_j = M = N$ this gives:}
\begin{align}
	A_F
&\approx
	\frac
		{\prod_{j=1}^{K} \sqrt{m_j! \; n_j!}}
		{(N!)^K}
	\qty( 
		\sum_{r=1}^{K}
		\sum_{c=1}^{L}
			\qty(x_c)^{KN - N}
			\qty(y_r)^{KN - N}
			\qty(u_{r,c})^{KN}
	) \\
&=
	\prod_{j=1}^{K}
		\frac
			{\sqrt{m_j! \; n_j!}}
			{N!}
		\qty(
			\sum_{r=1}^{K}
			\sum_{c=1}^{L}
				\qty(x_c)^{N-1}
				\qty(y_r)^{N-1}
				\qty(u_{r,c})^{N}
		)
\end{align}
	{\color{red} which is already pretty close to q.e.d. Assume the Statement holds, anyway. 
	(cf. question marks above)}
\item {\color{red} I have no clue in how far dimensionality $K \neq L$ is compatible with that.}
\end{itemize}

\subsection{Multinomial Theorem}
Statement:
\begin{align}
	\qty(
		\sum_{l=1}^{m} x_l
	)^{n}
&=
	\sum_{\sum_l^m k_l = n}
	\begin{pmatrix}
		n \\ k_1, k_2, \ldots, k_m
	\end{pmatrix}
	\prod_{t=1}^{m}
		x_t^{k_t}
\end{align}
where
\begin{align}
	\begin{pmatrix}
		n \\ k_1, k_2, \ldots, k_m
	\end{pmatrix}
&=
	\frac{1}
	{\prod_{l=1}^m k_l !}
\end{align}
is the multinomial coefficient.

Holds because
\begin{itemize}
\item math says so
\end{itemize}

\subsection{Apply Multinomial Theorem}
Statement
\begin{align}
	\qty( \vec{x}\transp \; \mathbb{U} \; \vec{y} )^{N}
&=
	\sum_{\sum_{k_j}^K = N}
		\begin{pmatrix}
			N \\ k_1, \ldots, k_K
		\end{pmatrix}
		\prod_{t=1}^K
			x_t^{k_t} ~
			\sum_{\sum_{j=1}^{L} \mu_j^{(t)} = k_t}
				\begin{pmatrix}
					k_t \\ \mu_1^{(t)}, \ldots, \mu_L^{(t)}
				\end{pmatrix}
				\prod_{\tau=1}^{L}
					\qty(
						u_{t, \tau} \; y_\tau
					)^{\mu_\tau^{(t)}}
\end{align}

Identities:
\begin{align}
	m_j &= k_j
&
	n_j &= \sum_{{\color{blue}} t=1}^{K} \mu_{j}^{{\color{blue} (t)}}
	\label{eqn:sumIndexIdentitiesO1}
\\
	m_j &= \sum_{{\color{blue}} j=1}^{L} \mu_{{\color{blue} j}}^{(t)} = k_j
	\label{eqn:sumIndexIdentitiesO2}
\end{align}

Holds because
\begin{itemize}
\item Straight forward: Use
	\begin{equation}
		\qty( \vec{x} \; \mathbb{U} \; \vec{y} )^N
		=
		\qty( \sum_{r} \sum_{c} x_c \; u_{r,c} \; y_r )^N
	\end{equation}

\item apply the multinomial theorem twice on each sum.
\item identities in eqn. (\ref{eqn:sumIndexIdentitiesO1}) from comparing the summation boundaries with
	given system restrictions
\item identity for $\vec{n}$ is hidden in the first application of the multinomial theorem
\item (\ref{eqn:sumIndexIdentitiesO2}) directly from multinomial theorem with $\vec{k} = \vec{m}$
\end{itemize}


\subsection{Evaluate Prefactors}
(gives ugly factors but no new insights. Copy (95), (95) from Max' paper when needed and regard it as proven. They are about the factors to $x_{\ldots}^{\ldots}, u_{\ldots}^{\ldots}, y_{\ldots}^{\ldots}$)

\section{2x2 Amplitude}
Statement
\begin{multline}
	A_F (m_1, m_2, n_1, n_2)
=
	\frac{\sqrt{m_1! \; m_2! \; n_1! \; n_2!}}{N!}
	\begin{pmatrix}
		N \\ m_1, m_2
	\end{pmatrix}
	\sum_{\nu_1^{(1)} = 0}^{n_1}
	\sum_{\nu_1^{(2)} = 0}^{n_2}
\\
	\Theta\qty[ m_1 - \nu_1^{(1)} - \nu_1^{(2)} ]
	\begin{pmatrix}
		m_1 \\ \nu_1^{(1)}, \nu_1^{(2)}
	\end{pmatrix}
	\Theta\qty[ m_2 - \qty(n_1 - \nu_1^{(1)})  - \qty(n_2 - \nu_1^{(2)}) ]
	\begin{pmatrix}
		m_2 \\ n_1 - \nu_1^{(1)} , n_2 - \nu_1^{(2)}
	\end{pmatrix}
\\
	\qty[
		u_{1,1}^{\nu_1^{(1)}} \; u_{2,1}^{n_1 - \nu_1^{(1)}} \;
		u_{1,2}^{\nu_1^{(2)}} \; u_{2,2}^{n_2 - \nu_1^{(2)}}
	]
\end{multline}
where $\Theta$ is the Heaviside step function.

Evaluating the multinomial coefficients and relabelling $\nu_1^{(x)} \thus \nu_x$ gives:
\begin{multline}
	A_F (m_1, m_2, n_1, n_2)
=
	\sqrt{m_1! \; m_2! \; n_1! \; n_2!}
	\sum_{\nu_1=0}^{n_1}
	\sum_{\nu_2=0}^{n_2}
\\
	\Theta\qty[ \big. m_1 - \nu_1 - \nu_2 ]
	\Theta\qty[ \big. m_2 - \qty(n_1 - \nu_1)  - \qty(n_2 - \nu_2) ]
	\frac
		{u_{1,1}^{\nu_1} \; u_{2,1}^{n_1 - \nu_1} \;
		 u_{1,2}^{\nu_2} \; u_{2,2}^{n_2 - \nu_2}}
		{\nu_1! \; (n_1 - \nu_1)! \;
		 \nu_2! \; (n_2 - \nu_2)!}		
\end{multline}

Holds because:
\begin{itemize}
\item just expand the expressions. A whole lot of work and lot's of indices to get confused with, but it works out.
\end{itemize}
\end{document}