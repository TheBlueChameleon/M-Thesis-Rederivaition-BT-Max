\documentclass[
	english,
	a4paper,
	fontsize=10pt,
	parskip=half,
	titlepage=true,
	DIV=12,
	final
]{scrreprt}


%==============================================================================%
% PACKAGES
%
% Standard text formatting
\usepackage[utf8]{inputenc}
\usepackage{babel}
\usepackage[T1]{fontenc}
\usepackage{lmodern}
\usepackage{microtype}
\usepackage{ragged2e}

\usepackage{csquotes}
\usepackage{xspace}

\usepackage{placeins}	% FloatBarrier.
\usepackage{url}
\usepackage[bf, format=plain]{caption}

\usepackage{hyperref}
\hypersetup{
    colorlinks,
    citecolor=black,
    filecolor=black,
    linkcolor=black,
    urlcolor=black
}

% gfx
\usepackage{wrapfig}
\usepackage{xcolor}

% tables
\usepackage{tabularx}
\usepackage{booktabs}
\usepackage{multicol}
\usepackage{multirow}
\usepackage{makecell}
\usepackage{color, colortbl}

% math
\usepackage{amsmath}
\usepackage{amssymb}
\usepackage{dsfont}
\let\olddiv\div
\usepackage[arrowdel]{physics}
\usepackage{mathtools}

% indexes, links, page format
\usepackage{scrlayer-scrpage}

% misc
\usepackage[super]{nth}
\usepackage[
	output-decimal-marker={.},
	input-symbols = {()},  			% do not treat "(" and ")" in any special way
	group-digits  = true  			% guess what.
]{siunitx}
\usepackage{minted}

%==============================================================================%
% GLOBAL MACROS
%

% Document properties
\newcommand{\myName}{Stefan Hartinger\xspace}
\newcommand{\myTitle}{Master Thesis Notes: Rederiving a Zero Occupancy Formula\xspace}

\addtokomafont{labelinglabel}{\sffamily}

% Text abbreviations
\newcommand*{\ie}{i.\,e.\xspace}
\newcommand*{\eg}{e.\,g.\xspace}

% Misc Symbols
\newcommand*{\thus}{\ensuremath{\rightarrow}\xspace}
\newcommand*{\Thus}{\ensuremath{\Rightarrow}\xspace}

% Tables
\newcommand*{\tabcrlf}{\\ \hline}			% actually still allows for optional argument

% Math
\newcommand*{\numberthis}{\addtocounter{equation}{1}\tag{\theequation}}

\newcommand*{\smallfrac}  [2]{\ensuremath{{}^        {#1} \!/_        {#2}}}
\newcommand*{\smallfracrm}[2]{\ensuremath{{}^{\mathrm{#1}}\!/_{\mathrm{#2}}}}

\newcommand*{\transp}{\ensuremath{^\intercal}}

\newcommand*{\iunit}{\ensuremath{\mathrm{i}}}

\newcommand*{\setNaturals} {\ensuremath{\mathbb{N}}}
\newcommand*{\setIntegers} {\ensuremath{\mathbb{Z}}}
\newcommand*{\setReals}    {\ensuremath{\mathbb{R}}}
\newcommand*{\setRationals}{\ensuremath{\mathbb{Q}}}
\newcommand*{\setComplex}  {\ensuremath{\mathbb{C}}}

\newcommand*{\Lag}{\ensuremath{\mathcal{L}}\xspace}
\newcommand*{\Ham}{\ensuremath{\mathcal{H}}\xspace}

%\newcommand*{\Poisson}[2]{\ensuremath{\left\{ {#1}, {#2} \right\}}}
% physics has \pb which is poisson bracket
% also use alias acom: anticommutator, which is exactly the same.

\newcommand*{\equalCond}{  \mathop{=}\limits^!  }

\DeclareMathOperator{\arsinh}{arsinh}
\DeclareMathOperator{\diag}{diag}

\newcommand*{\DD}[1]{\ensuremath{\text{D}\vec{#1}\;}}

%==============================================================================%
% GLOBAL PARAMTERS
%

\title{\myTitle}
\author{\myName}
\date{\today}

% header, footer
\clearpairofpagestyles
	\cfoot
		[\pagemark]
		{\pagemark}
	\ohead
		[\myTitle, \myName]
		{\myTitle, \myName}
\pagestyle{scrheadings}

%==============================================================================%
% THE REAL STUFF
%	
\begin{document}
\tableofcontents
\newpage

\chapter{Redo}
\section{Symbols and Notation}
\subsection{Definitions}
\begin{itemize}
\item $K$: number of physically present channels
\item $b_i, b_o$: Number of occupied incoming/outgoing modes, respectively
\item $j^*$: reference mode on incoming end (all phases set to zero for this mode)
\end{itemize}

\subsection{Assumptions}
\begin{itemize}
\item $m_j = 0 \Thus \theta_j = 0$
\item $n_k = 0 \Thus \chi  _k = 0$
\item $m_{j^*} \neq 0$
\end{itemize}


\section{Starting Point}
\begin{align}
	A_F(\vec{m}, \vec{n})
&=
	\eval{\qty(
			\prod_{j=1}^{K}
			\frac
				{1}
				{\sqrt{m_j! \; n_j!}}
			\pdv[m_j]{x_j}
			\pdv[n_j]{y_j}
		) \exp(\vec{x}\transp \, \mathbb{U} \, \vec{y})
	}_{\vec{x} = \vec{y} = \vec{0}}
\label{eqn:BSA_raw}
\end{align}

\section{ZO-Form}
\begin{align}
	A_F(\vec{m}, \vec{n})
&=
	\eval{
		\qty(
			\prod_{j : m_j \neq 0}
			\frac {1} {\sqrt{m_j!}}
			\pdv[m_j]{x_j}
		)
		\qty(
			\prod_{k : n_k \neq 0}
			\frac
				{1}
				{\sqrt{n_k!}}
			\pdv[n_k]{y_k}
		)
		\exp(\vec{x}\transp \, \mathbb{U} \, \vec{y})
	}_{\vec{x} = \vec{y} = \vec{0}}
\label{eqn:BSA_ZO}
\end{align}

Holds because:
\begin{itemize}
\item $0! = 1$
\item $\pdv[0]{z} = 1$
\end{itemize}

\section{BSA in Integral Form}
\subsection{Formalism}
\begin{align}
	\eval{ f^{(n)}(\vec{z}) }_{\vec{z} = \vec{0}}
&=
	\qty(
		\frac{n!}{2\pi \iunit}
	)^{D}
	\oint_{\gamma} \dd{\zeta_1} \ldots \dd{\zeta_D}
		\frac
			{f(\vec{\zeta})}
			{\prod_{j=1}^{D} \zeta_j^{n+1}}
\end{align}

Symbols:
\begin{itemize}
\item $f$: function matching the signature $\setComplex^{D} \thus \setComplex$; holomorphic, \ie holomorphic in each variable
\item $f^{(n)}$: $n^{\text{th}}$ derivative of $f$
\item $\vec{z}$: vector in $D$ dimensions, representing $\vec{x} \oplus \vec{y}$
\item $\vec{\zeta}$: vector in $D$ dimensions, complex valued, associated to $\vec{z}$
\item $\gamma$: arbitrary loop in the complex plane, enclosing the origin.
\end{itemize}

Holds because:
\begin{itemize}
\item Cf. Max' thesis paper
\item I have followed this train of thought multiple times, it is sound.
\end{itemize}

\subsection{Application}
\begin{align}
	A_F(\vec{m}, \vec{n})
&=
	(2\pi\iunit)^{-(b_i + b_o)}
	\oint_{\gamma}
		\qty( \prod_{j : m_j \neq 0}
			\frac
				{\sqrt{m_j!} \dd{x_j}}
				{x_j^{m_j+1}}
		)
		\qty( \prod_{k : n_k \neq 0}
			\frac
				{\sqrt{n_k!} \dd{y_k}}
				{ y_k^{n_k+1} }
		)
		\exp( \Big. \vec{x}\transp \, \mathbb{U} \, \vec{y})
	\label{eqn:BSA_Int_BZO}
\end{align}

Holds because:
\begin{itemize}
\item $f = \exp( \vec{x}\transp \, \mathbb{U} \, \vec{y})$
\item Integral form generates terms of the form $m_j!$, partially cancelled by $\smallfrac{1}{\sqrt{m_j!}}$ (and alike for $n$), gives the $\sqrt{m_j!}$ terms.
\end{itemize}

\section{Change of Variables}
Using polar coordinates:
\begin{align}
	\begin{cases}
	x_j &= \sqrt{m_j} \exp( \iunit \theta_j) \\
	y_j &= \sqrt{n_j} \exp(-\iunit \chi  _j)
	\end{cases}
	\label{eqn:DefXY}
\end{align}

in eqn. (\ref{eqn:BSA_Int_BZO}) gives:
\begin{multline}
	A_F(\vec{m}, \vec{n})
=
	(2\pi\iunit)^{-(b_i + b_o)}
	\int_{0}^{2\pi}
		\qty( \prod_{j : m_j \neq 0}
			\sqrt{m_j!}
			\frac
				{ {\color{blue} \iunit} \sqrt{m_j}}
				{ \sqrt{m_j}^{m_j + 1} }
			\dd{\theta_j}
		)
		\qty( \prod_{k : n_k \neq 0}
			\sqrt{n_k!}
			\frac
				{ {\color{blue} -\iunit} \sqrt{n_k}}
				{ \sqrt{n_k}^{n_k + 1} }
			\dd{\chi_k}
		)
\\
	\underbrace{
		\exp[ \iunit \qty(
			{\textstyle \sum_{j=1}^{b_i} \theta_j} - 
			{\textstyle \sum_{k=1}^{b_o} \chi  _k}
		)]
	}_{\text{from } \dd{x}, \dd{y}}
	\underbrace{
		\exp[
			\iunit \qty(
				-{\textstyle \sum_{j=1}^{b_i} (m_j + 1) \theta_j}
				+{\textstyle \sum_{k=1}^{b_o} (n_k + 1) \chi  _k}
		)]
	}_{\text{from } \smallfrac{1}{x}^{m+1}, \smallfrac{1}{y}^{n+1}}
	\exp( \Big. \vec{x}\transp \, \mathbb{U} \, \vec{y})
\end{multline}
where the imaginary units come from substitution $(\vec{x}, \vec{y}) \to (\vec{\theta}, \vec{\chi})$:
\begin{align}
	\dd{x_j}
&=
	\dv{x_j}{\theta_j} \dd{\theta_j}
=
	\iunit \, \sqrt{m_j} \, \exp(\iunit \theta_j) \dd{\theta_j}
&
	\dd{y_k}
&=
	\dv{y_k}{\theta_k} \dd{\chi_k}
=
	-\iunit \, \sqrt{n_k} \, \exp(-\iunit \chi_k) \dd{\chi_k}
\end{align}

Note that we assumed that unoccupied modes have phase zero, so the sums in $\exp$ are well defined. This is justified, as unoccupied modes shouldn't give any contribution, and with this convention we get a factor 1, \ie no change.

Consolidate:
\begin{multline}
	A_F(\vec{m}, \vec{n})
=
	(-1)^{b_o}
	(2\pi)^{-(b_i + b_o)}
	\int_{0}^{2\pi}
		\qty( \prod_{j : m_j \neq 0}
			\sqrt{\frac
				{ m_j! }
				{ m_j^{m_j} }
			} \dd{\theta_j}
		)
		\qty( \prod_{k : n_k \neq 0}
			\sqrt{\frac
				{ n_k! }
				{ n_k^{n_k} }
			} \dd{\chi_k}
		)
\\
		\exp[
			\iunit \qty(
			-\vec{m} \cdot \vec{\theta}
			+\vec{n} \cdot \vec{\chi}
		)]
		\exp( \Big. \vec{x}\transp \, \mathbb{U} \, \vec{y})
\end{multline}

Introduce notation:
\begin{align}
	C
&=
	(-1)^{b_o}
	(2\pi)^{-(b_i + b_o)}
	\qty( \prod_{j : m_j \neq 0}
		\sqrt{\frac
			{ m_j! }
			{ m_j^{m_j} }
		}
	)
	\qty( \prod_{k : n_k \neq 0}
		\sqrt{\frac
			{ n_k! }
			{ n_k^{n_k} }
		}
	)
\end{align}
\begin{align}
	\DD{\theta}
&=
	\prod_{\substack{j : m_j \neq 0 \\ {\color{blue} j \neq j^*}}}
		\dd{\theta_j}
&
	\DD{\chi}
&=
	\prod_{k : n_k \neq 0}
		\dd{\chi  _k}
\label{eqn:DefC_m}
\end{align}

Consolidate again:
\begin{align}
	A_F(\vec{m}, \vec{n})
&=
	C
	\int_{0}^{2\pi}
		{\color{blue} \dd{\theta_{j^*}}}
		\DD{\theta}
		\DD{\chi  }
			\exp[
				\iunit \qty(
				-\vec{m} \cdot \vec{\theta}
				+\vec{n} \cdot \vec{\chi}
			)]
			\exp( \Big. \vec{x}\transp \, \mathbb{U} \, \vec{y})
	\label{eqn:BSA_Int_BZO_Polar}
\end{align}

\section{Integrating out the $\theta_{j^*}$ DoF}
Use temporary variables:
\begin{align}
	\vec{\alpha} &= \vec{\theta} - (\theta_1, \ldots, \theta_1)\transp \in \setReals^{K}
	&
	\vec{\theta} &= \vec{\alpha} + (\theta_1, \ldots, \theta_1)\transp \in \setReals^{K}
\\
	\vec{\beta } &= \vec{\chi  } - (\theta_1, \ldots, \theta_1)\transp \in \setReals^{K}
	&
	\vec{\chi  } &= \vec{\beta } + (\theta_1, \ldots, \theta_1)\transp \in \setReals^{K}
\end{align}
and get
\begin{multline}
	A_F(\vec{m}, \vec{n})
=
	C
	\int_{0}^{2\pi} {\color{blue} \dd{\theta_{j^*}}}
	\int_{0}^{2\pi}
		\qty( \prod_{\substack{j : m_j \neq 0 \\ {\color{blue} j \neq j^*}}} \dd{\alpha_j} )
		\qty( \prod_{k : n_k \neq 0}                 \dd{\beta_k} )
			\exp[
				\iunit \qty(
				{\textstyle \sum_{s=1}^{K}   n_s \beta _s}   -
				{\textstyle \sum_{t=1}^{b^*} m_t \alpha_t}
			)]
\\
	\underbrace{
	\exp[
		\iunit \qty(
			{\textstyle \sum_{s=1}^{K}   n_s}   -
			{\textstyle \sum_{t=1}^{b^*} m_t}
		)
		{\color{blue} \theta_{j^*}}
	]
	}_{\thus 2\pi\delta_{MN} \text{ after integration}}
	\underbrace{
	\exp[
		{\textstyle \sum_{s,t=1}^{K} }
			\sqrt{m_s \, n_t} \;
			u_{s,t}
			\exp(\Big. \iunit(\alpha_s - \beta_t) )
	]}_{\exp( \vec{x}\transp \, \mathbb{U} \, \vec{y})}
\end{multline}

Resubstitute $(\vec{\alpha}, \vec{\beta}) \to (\vec{\theta}, \vec{\chi})$:
\begin{multline}	
	A_F(\vec{m}, \vec{n})
=
	2\pi C \, \delta_{M,N} \,
	\int_{0}^{2\pi}
		\DD{\theta} \DD{\chi}
	\exp[
		\iunit \qty(
			\vec{n} \cdot \vec{\chi}
			-
			\vec{m} \cdot \vec{\theta}
		)]
	\exp[
		{\textstyle \sum_{s,t=1}^{K} }
			\sqrt{m_s, n_t} \;
			u_{s,t}
			\exp(\Big. \iunit(\theta_s - \chi_t) )
	]
\end{multline}
with arbitrary def: $\theta_1 = 0$

\section{Applying SPA}
\subsection{Definitions}
Using
\begin{align}
	\lambda
&=
	\qty( \prod_{j : m_j \neq 0} m_j )
	\qty( \prod_{k : n_k \neq 0} n_k )
\label{eqn:DefLambda}
\\
	f(
		\underbrace{ \vec{\theta}, \vec{\chi} }_{\vec{z}}
	)
&=
	\lambda^{-1}
	\qty[
		\iunit
		\qty(
			{\color{blue} +}
			\sum_{\substack{j : m_j \neq 0 \\ j \neq j^*}}
				m_j \theta_j \;
			{\color{blue} -} \,
			\sum_{k : n_k \neq 0}
				n_k \chi  _k
		)
	{\color{blue} -}
		\qty(
			\sum_{s,t=1}^{K}
			\sqrt{m_s n_t} \;
			u_{s,t}
			\exp(\Big.
				\iunit(\theta_s - \chi_t)
			)
		)
	]
\\
	\oint_{\gamma} \dd[n]{\vec{z}}
		\exp[{\color{blue} -}\lambda f(\vec{z})]
&=
	\qty( \frac
		{2\pi}
		{\lambda}
	)^{\frac{n}{2}}
	\sum_{r}
		\frac
		{\exp(-\lambda f(\vec{z}_r))}
		{\sqrt{\det S_r}}
\label{eqn:Def_SPA}
\end{align}
where $\vec{z}_r$ are the local extrema of $f$: $\eval{\dv{\vec{z}}f}_{\vec{z} = \vec{z}_r} = 0$\\
and $S_r$ is the Hessian of $f$ evaluated in $\vec{z}_r$

Use this to solve \ref{eqn:BSA_Int_BZO_Polar}.

\section{Saddle Points}
Require:
\begin{align}
	\pdv{f}{\theta_p} &\equalCond 0
&
	\pdv{f}{\chi  _q} &\equalCond 0 
\end{align}
with conditions
\begin{itemize}
\item $p, q \in \{1, \ldots, K\}$
\item $m_p \neq 0$
\item $n_p \neq 0$
\item $p \neq j^*$
\end{itemize}

Find:
\begin{gather}
	\pdv{f}{\theta_p}
=
	\iunit \lambda^{-1}
	\qty[
		m_p
		-
		\sum_{l=1}^{K}
			\sqrt{m_p n_l} \; u_{p,l} \; \exp[ \Big. \iunit(\theta_p - \chi_l)]
	]
	\label{eqn:Jacobiantheta}
\equalCond
	0 \\
\Thus
	m_p
\equalCond
	\sum_{l=1}^{K}
		\sqrt{m_p n_l} \; u_{p,l} \; \exp[ \Big. \iunit(\theta_p - \chi_l) ] \\
\Thus
	\sqrt{m_p} \exp(-\iunit \theta_p)
\equalCond
	\sum_{l=1}^{K} \sqrt{n_l} \; u_{p,l} \; \exp(-\iunit\chi_l)
	\label{eqn:FourierLinkForward}
\end{gather}
where the $n_l, m_l \neq 0$ conditions are releaved since in the sum they give a null contribution. Use virtual $\theta_l, \chi_l = 0$

Likewise:
\begin{gather}
	\pdv{f}{\chi_q}
=
	\iunit\lambda^{-1}
	\qty[
		- n_q
		+
		\sum_{k=1}^{b^*}
			\sqrt{m_k n_q} \; u_{k,q} \; \exp[ \Big. \iunit(\theta_k - \chi_q) ]
	]
	\label{eqn:JacobianChi}
\equalCond
	0 \\
\Thus
	\sqrt{n_q} \exp(+\iunit \chi_q)
\equalCond
	\sum_{k=1}^{b^*}
	\sqrt{m_k} \; u_{k,q} \; \exp(+\iunit\theta_k)
	\label{eqn:FourierLinkBackward}
\end{gather}

Compactly:
\begin{align}
	\vec{x}^{*} &= \mathbb{U} \vec{y}
&
	\vec{y}^{*} &= \mathbb{U}\transp \vec{x}
	\label{eqn:MatrixCondition}
\end{align}
\end{document}