\documentclass[
	english,
	a4paper,
	fontsize=10pt,
	parskip=half,
	titlepage=true,
	DIV=12,
	final
]{scrreprt}


%==============================================================================%
% PACKAGES
%
% Standard text formatting
\usepackage[utf8]{inputenc}
\usepackage{babel}
\usepackage[T1]{fontenc}
\usepackage{lmodern}
\usepackage{microtype}
\usepackage{ragged2e}

\usepackage{csquotes}
\usepackage{xspace}

\usepackage{placeins}	% FloatBarrier.
\usepackage{url}
\usepackage[bf, format=plain]{caption}

\usepackage{hyperref}
\hypersetup{
    colorlinks,
    citecolor=black,
    filecolor=black,
    linkcolor=black,
    urlcolor=black
}

% gfx
\usepackage{wrapfig}
\usepackage{xcolor}

% tables
\usepackage{tabularx}
\usepackage{booktabs}
\usepackage{multicol}
\usepackage{multirow}
\usepackage{makecell}
\usepackage{color, colortbl}

% math
\usepackage{amsmath}
\usepackage{amssymb}
\usepackage{dsfont}
\let\olddiv\div
\usepackage[arrowdel]{physics}
\usepackage{mathtools}

% indexes, links, page format
\usepackage{scrlayer-scrpage}

% misc
\usepackage[super]{nth}
\usepackage[
	output-decimal-marker={.},
	input-symbols = {()},  			% do not treat "(" and ")" in any special way
	group-digits  = true  			% guess what.
]{siunitx}
\usepackage{minted}

%==============================================================================%
% GLOBAL MACROS
%

% Document properties
\newcommand{\myName}{Stefan Hartinger\xspace}
\newcommand{\myTitle}{Master Thesis Notes: Rederiving a Zero Occupancy Formula\xspace}

\addtokomafont{labelinglabel}{\sffamily}

% Text abbreviations
\newcommand*{\ie}{i.\,e.\xspace}
\newcommand*{\eg}{e.\,g.\xspace}

% Misc Symbols
\newcommand*{\thus}{\ensuremath{\rightarrow}\xspace}
\newcommand*{\Thus}{\ensuremath{\Rightarrow}\xspace}

% Tables
\newcommand*{\tabcrlf}{\\ \hline}			% actually still allows for optional argument

% Math
\newcommand*{\numberthis}{\addtocounter{equation}{1}\tag{\theequation}}

\newcommand*{\smallfrac}  [2]{\ensuremath{{}^        {#1} \!/_        {#2}}}
\newcommand*{\smallfracrm}[2]{\ensuremath{{}^{\mathrm{#1}}\!/_{\mathrm{#2}}}}

\newcommand*{\transp}{\ensuremath{^\intercal}}

\newcommand*{\iunit}{\ensuremath{\mathrm{i}}}

\newcommand*{\setNaturals} {\ensuremath{\mathbb{N}}}
\newcommand*{\setIntegers} {\ensuremath{\mathbb{Z}}}
\newcommand*{\setReals}    {\ensuremath{\mathbb{R}}}
\newcommand*{\setRationals}{\ensuremath{\mathbb{Q}}}
\newcommand*{\setComplex}  {\ensuremath{\mathbb{C}}}

\newcommand*{\Lag}{\ensuremath{\mathcal{L}}\xspace}
\newcommand*{\Ham}{\ensuremath{\mathcal{H}}\xspace}

%\newcommand*{\Poisson}[2]{\ensuremath{\left\{ {#1}, {#2} \right\}}}
% physics has \pb which is poisson bracket
% also use alias acom: anticommutator, which is exactly the same.

\newcommand*{\equalCond}{  \mathop{=}\limits^!  }

\DeclareMathOperator{\arsinh}{arsinh}
\DeclareMathOperator{\diag}{diag}

\newcommand*{\DD}[1]{\ensuremath{\text{D}\vec{#1}\;}}
\newcommand*{\set}[1]{\ensuremath{\{#1\}}}

%==============================================================================%
% GLOBAL PARAMTERS
%

\title{\myTitle}
\author{\myName}
\date{\today}

% header, footer
\clearpairofpagestyles
	\cfoot
		[\pagemark]
		{\pagemark}
	\ohead
		[\myTitle, \myName]
		{\myTitle, \myName}
\pagestyle{scrheadings}

%==============================================================================%
% THE REAL STUFF
%	
\begin{document}
\tableofcontents
\newpage

\chapter{Redo}
\section{Symbols and Notation}
\subsection{Definitions}
\begin{itemize}
\item $K$: number of physically present channels
\item $b_i, b_o$: Number of occupied incoming/outgoing modes, respectively
\item $j^*$: reference mode on incoming end (phase set to zero for this mode)
\end{itemize}

\subsection{Assumptions}
\begin{itemize}
\item $m_j = 0 \Thus \theta_j = 0$
\item $n_k = 0 \Thus \chi  _k = 0$
\item $m_{j^*} \neq 0$; $\theta_{j^*} = 0$
\end{itemize}


\section{Starting Point}
\begin{align}
	A_F(\vec{m}, \vec{n})
&=
	\eval{\qty(
			\prod_{j=1}^{K}
			\frac
				{1}
				{\sqrt{m_j! \; n_j!}}
			\pdv[m_j]{x_j}
			\pdv[n_j]{y_j}
		) \exp(\vec{x}\transp \, \mathbb{U} \, \vec{y})
	}_{\vec{x} = \vec{y} = \vec{0}}
\label{eqn:BSA_raw}
\end{align}

\section{ZO-Form}
\begin{align}
	A_F(\vec{m}, \vec{n})
&=
	\eval{
		\qty(
			\prod_{j : m_j \neq 0}
			\frac {1} {\sqrt{m_j!}}
			\pdv[m_j]{x_j}
		)
		\qty(
			\prod_{k : n_k \neq 0}
			\frac
				{1}
				{\sqrt{n_k!}}
			\pdv[n_k]{y_k}
		)
		\exp(\vec{x}\transp \, \mathbb{U} \, \vec{y})
	}_{\vec{x} = \vec{y} = \vec{0}}
\label{eqn:BSA_ZO}
\end{align}

Holds because:
\begin{itemize}
\item $0! = 1$
\item $\pdv[0]{z} = 1$
\end{itemize}

\section{BSA in Integral Form}
\subsection{Formalism}
\begin{align}
	\eval{ f^{(n)}(\vec{z}) }_{\vec{z} = \vec{0}}
&=
	\qty(
		\frac{n!}{2\pi \iunit}
	)^{D}
	\oint_{\gamma} \dd{\zeta_1} \ldots \dd{\zeta_D}
		\frac
			{f(\vec{\zeta})}
			{\prod_{j=1}^{D} \zeta_j^{n+1}}
\end{align}

Symbols:
\begin{itemize}
\item $f$: function matching the signature $\setComplex^{D} \thus \setComplex$; holomorphic, \ie holomorphic in each variable
\item $f^{(n)}$: $n^{\text{th}}$ derivative of $f$
\item $\vec{z}$: vector in $D$ dimensions, representing $\vec{x} \oplus \vec{y}$
\item $\vec{\zeta}$: vector in $D$ dimensions, complex valued, associated to $\vec{z}$
\item $\gamma$: arbitrary loop in the complex plane, enclosing the origin.
\end{itemize}

Holds because:
\begin{itemize}
\item Cf. Max' thesis paper
\item I have followed this train of thought multiple times, it is sound.
\end{itemize}

\subsection{Application}
\begin{align}
	A_F(\vec{m}, \vec{n})
&=
	(2\pi\iunit)^{-(b_i + b_o)}
	\oint_{\gamma}
		\qty( \prod_{j : m_j \neq 0}
			\frac
				{\sqrt{m_j!} \dd{x_j}}
				{x_j^{m_j+1}}
		)
		\qty( \prod_{k : n_k \neq 0}
			\frac
				{\sqrt{n_k!} \dd{y_k}}
				{ y_k^{n_k+1} }
		)
		\exp( \Big. \vec{x}\transp \, \mathbb{U} \, \vec{y})
	\label{eqn:BSA_Int_BZO}
\end{align}

Holds because:
\begin{itemize}
\item $f = \exp( \vec{x}\transp \, \mathbb{U} \, \vec{y})$
\item Integral form generates terms of the form $m_j!$, partially cancelled by $\smallfrac{1}{\sqrt{m_j!}}$ (and alike for $n$), gives the $\sqrt{m_j!}$ terms.
\end{itemize}

\section{Change of Variables}
Using polar coordinates:
\begin{align}
	\begin{cases}
	x_j &= \sqrt{m_j} \exp( \iunit \theta_j) \\
	y_j &= \sqrt{n_j} \exp(-\iunit \chi  _j)
	\end{cases}
	\label{eqn:DefXY}
\end{align}

in eqn. (\ref{eqn:BSA_Int_BZO}) gives:
\begin{multline}
	A_F(\vec{m}, \vec{n})
=
	(2\pi\iunit)^{-(b_i + b_o)}
	\int_{0}^{2\pi}
		\qty( \prod_{j : m_j \neq 0}
			\sqrt{m_j!}
			\frac
				{ {\color{blue} \iunit} \sqrt{m_j}}
				{ \sqrt{m_j}^{m_j + 1} }
			\dd{\theta_j}
		)
		\qty( \prod_{k : n_k \neq 0}
			\sqrt{n_k!}
			\frac
				{ {\color{blue} -\iunit} \sqrt{n_k}}
				{ \sqrt{n_k}^{n_k + 1} }
			\dd{\chi_k}
		)
\\
	\underbrace{
		\exp[ \iunit \qty(
			{\textstyle \sum_{j=1}^{b_i} \theta_j} - 
			{\textstyle \sum_{k=1}^{b_o} \chi  _k}
		)]
	}_{\text{from } \dd{x}, \dd{y}}
	\underbrace{
		\exp[
			\iunit \qty(
				-{\textstyle \sum_{j=1}^{b_i} (m_j + 1) \theta_j}
				+{\textstyle \sum_{k=1}^{b_o} (n_k + 1) \chi  _k}
		)]
	}_{\text{from } \smallfrac{1}{x}^{m+1}, \smallfrac{1}{y}^{n+1}}
	\exp( \Big. \vec{x}\transp \, \mathbb{U} \, \vec{y})
\end{multline}
where the imaginary units come from substitution $(\vec{x}, \vec{y}) \to (\vec{\theta}, \vec{\chi})$:
\begin{align}
	\dd{x_j}
&=
	\dv{x_j}{\theta_j} \dd{\theta_j}
=
	\iunit \, \sqrt{m_j} \, \exp(\iunit \theta_j) \dd{\theta_j}
&
	\dd{y_k}
&=
	\dv{y_k}{\theta_k} \dd{\chi_k}
=
	-\iunit \, \sqrt{n_k} \, \exp(-\iunit \chi_k) \dd{\chi_k}
\end{align}

Note that we assumed that unoccupied modes have phase zero, so the sums in $\exp$ are well defined. This is justified, as unoccupied modes shouldn't give any contribution, and with this convention we get a factor 1, \ie no change.

Consolidate:
\begin{multline}
	A_F(\vec{m}, \vec{n})
=
	(-1)^{b_o}
	(2\pi)^{-(b_i + b_o)}
	\int_{0}^{2\pi}
		\qty( \prod_{j : m_j \neq 0}
			\sqrt{\frac
				{ m_j! }
				{ m_j^{m_j} }
			} \dd{\theta_j}
		)
		\qty( \prod_{k : n_k \neq 0}
			\sqrt{\frac
				{ n_k! }
				{ n_k^{n_k} }
			} \dd{\chi_k}
		)
\\
		\exp[
			\iunit \qty(
			-\vec{m} \cdot \vec{\theta}
			+\vec{n} \cdot \vec{\chi}
		)]
		\exp( \Big. \vec{x}\transp \, \mathbb{U} \, \vec{y})
\end{multline}

Introduce notation:
\begin{align}
	C
&=
	(-1)^{b_o}
	(2\pi)^{-(b_i + b_o)}
	\qty( \prod_{j : m_j \neq 0}
		\sqrt{\frac
			{ m_j! }
			{ m_j^{m_j} }
		}
	)
	\qty( \prod_{k : n_k \neq 0}
		\sqrt{\frac
			{ n_k! }
			{ n_k^{n_k} }
		}
	)
\label{eqn:DefC}
\end{align}
\begin{align}
	\DD{\theta}
&=
	\prod_{\substack{j : m_j \neq 0 \\ {\color{blue} j \neq j^*}}}
		\dd{\theta_j}
&
	\DD{\chi}
&=
	\prod_{k : n_k \neq 0}
		\dd{\chi  _k}
\end{align}

Consolidate again:
\begin{align}
	A_F(\vec{m}, \vec{n})
&=
	C
	\int_{0}^{2\pi}
		{\color{blue} \dd{\theta_{j^*}}}
		\DD{\theta}
		\DD{\chi  }
			\exp[
				\iunit \qty(
				-\vec{m} \cdot \vec{\theta}
				+\vec{n} \cdot \vec{\chi}
			)]
			\exp( \Big. \vec{x}\transp \, \mathbb{U} \, \vec{y})
	\label{eqn:BSA_Int_BZO_Polar}
\end{align}

\section{Integrating out the $\theta_{j^*}$ DoF}
Use temporary variables:
\begin{align}
	\vec{\alpha} &= \vec{\theta} - (\theta_1, \ldots, \theta_1)\transp \in \setReals^{K}
	&
	\vec{\theta} &= \vec{\alpha} + (\theta_1, \ldots, \theta_1)\transp \in \setReals^{K}
\\
	\vec{\beta } &= \vec{\chi  } - (\theta_1, \ldots, \theta_1)\transp \in \setReals^{K}
	&
	\vec{\chi  } &= \vec{\beta } + (\theta_1, \ldots, \theta_1)\transp \in \setReals^{K}
\end{align}
and get
\begin{multline}
	A_F(\vec{m}, \vec{n})
=
	C
	\int_{0}^{2\pi} {\color{blue} \dd{\theta_{j^*}}}
	\int_{0}^{2\pi}
		\qty( \prod_{\substack{j : m_j \neq 0 \\ {\color{blue} j \neq j^*}}} \dd{\alpha_j} )
		\qty( \prod_{k : n_k \neq 0}                 \dd{\beta_k} )
			\exp[
				\iunit \qty(
				{\textstyle \sum_{s=1}^{K}   n_s \beta _s}   -
				{\textstyle \sum_{t=1}^{b^*} m_t \alpha_t}
			)]
\\
	\underbrace{
	\exp[
		\iunit \qty(
			{\textstyle \sum_{s=1}^{K}   n_s}   -
			{\textstyle \sum_{t=1}^{b^*} m_t}
		)
		{\color{blue} \theta_{j^*}}
	]
	}_{\thus 2\pi\delta_{MN} \text{ after integration}}
	\underbrace{
	\exp[
		{\textstyle \sum_{s,t=1}^{K} }
			\sqrt{m_s \, n_t} \;
			u_{s,t}
			\exp(\Big. \iunit(\alpha_s - \beta_t) )
	]}_{\exp( \vec{x}\transp \, \mathbb{U} \, \vec{y})}
\end{multline}

Resubstitute $(\vec{\alpha}, \vec{\beta}) \to (\vec{\theta}, \vec{\chi})$:
\begin{align}	
	A_F(\vec{m}, \vec{n})
=
	2\pi C \, \delta_{M,N} \,
	\int_{0}^{2\pi}
		\DD{\theta} \DD{\chi}
	\exp[
		\iunit \qty(
			\vec{n} \cdot \vec{\chi}
			-
			\vec{m} \cdot \vec{\theta}
		)]
	\exp[
		{\textstyle \sum_{s,t=1}^{K} }
			\sqrt{m_s, n_t} \;
			u_{s,t}
			\exp(\Big. \iunit(\theta_s - \chi_t) )
	]
\label{eqn:BSA_preSPA}
\end{align}
with arbitrary def: $\theta_1 = 0$

\section{Applying SPA}
\subsection{Definitions}
Using
\begin{align}
	\lambda
&=
	\qty( \prod_{j : m_j \neq 0} m_j )
	\qty( \prod_{k : n_k \neq 0} n_k )
\label{eqn:DefLambda}
\\
	f(
		\underbrace{ \vec{\theta}, \vec{\chi} }_{\vec{z}}
	)
&=
	\lambda^{-1}
	\qty[
		\iunit
		\qty(
			{\color{blue} +}
			\sum_{\substack{j : m_j \neq 0 \\ j \neq j^*}}
				m_j \theta_j \;
			{\color{blue} -} \,
			\sum_{k : n_k \neq 0}
				n_k \chi  _k
		)
	{\color{blue} -}
		\qty(
			\sum_{s,t=1}^{K}
			\sqrt{m_s n_t} \;
			u_{s,t}
			\exp(\Big.
				\iunit(\theta_s - \chi_t)
			)
		)
	]
\\
	\oint_{\gamma} \dd[n]{\vec{z}}
		\exp[{\color{blue} -}\lambda f(\vec{z})]
&\approx
	\qty( \frac
		{2\pi}
		{\lambda}
	)^{\frac{n}{2}}
	\sum_{r}
		\frac
		{\exp(-\lambda f(\vec{z}_r))}
		{\sqrt{\det S_r}}
\label{eqn:Def_SPA}
\end{align}
where $\vec{z}_r$ are the local extrema of $f$: $\eval{\dv{\vec{z}}f}_{\vec{z} = \vec{z}_r} = 0$\\
and $S_r$ is the Hessian of $f$ evaluated in $\vec{z}_r$

Use this to solve \ref{eqn:BSA_Int_BZO_Polar}.

\subsection{Saddle Points}
Require:
\begin{align}
	\pdv{f}{\theta_p} &\equalCond 0
&
	\pdv{f}{\chi  _q} &\equalCond 0 
\end{align}
with conditions
\begin{itemize}
\item $p, q \in \{1, \ldots, K\}$
\item $m_p \neq 0$
\item $n_p \neq 0$
\item $p \neq j^*$
\end{itemize}

Find:
\begin{gather}
	\pdv{f}{\theta_p}
=
	\iunit \lambda^{-1}
	\qty[
		m_p
		-
		\sum_{l=1}^{K}
			\sqrt{m_p n_l} \; u_{p,l} \; \exp[ \Big. \iunit(\theta_p - \chi_l)]
	]
	\label{eqn:Jacobiantheta}
\equalCond
	0 \\
\Thus
	m_p
\equalCond
	\sum_{l=1}^{K}
		\sqrt{m_p n_l} \; u_{p,l} \; \exp[ \Big. \iunit(\theta_p - \chi_l) ] \\
\Thus
	\sqrt{m_p} \exp(-\iunit \theta_p)
\equalCond
	\sum_{l=1}^{K} \sqrt{n_l} \; u_{p,l} \; \exp(-\iunit\chi_l)
	\label{eqn:FourierLinkForward}
\end{gather}
where the $n_l, m_l \neq 0$ conditions are releaved since in the sum they give a null contribution. Use virtual $\theta_l, \chi_l = 0$

Likewise:
\begin{gather}
	\pdv{f}{\chi_q}
=
	\iunit\lambda^{-1}
	\qty[
		- n_q
		+
		\sum_{k=1}^{K}
			\sqrt{m_k n_q} \; u_{k,q} \; \exp[ \Big. \iunit(\theta_k - \chi_q) ]
	]
	\label{eqn:JacobianChi}
\equalCond
	0 \\
\Thus
	\sqrt{n_q} \exp(+\iunit \chi_q)
\equalCond
	\sum_{k=1}^{K}
	\sqrt{m_k} \; u_{k,q} \; \exp(+\iunit\theta_k)
	\label{eqn:FourierLinkBackward}
\end{gather}

Compactly:
\begin{align}
	\vec{x}^{*} &= \mathbb{U} \vec{y}
&
	\vec{y}^{*} &= \mathbb{U}\transp \vec{x}
	\label{eqn:MatrixCondition}
\end{align}

\subsection{Hessian}
Recall: We start from
\begin{align}
	S_r
=
	\eval{
		\pdv{f(\vec{z})}%
			{z_\alpha}{z_\beta}
	}_{\vec{z} = \vec{z}_r}
\qfor{}
%	\text{where }
	\eval{\pdv{f}{z}}_{z=z_r} = \vec{0}
\end{align}
but only true DoFs are relevant. Hence, regard $m_j, n_j$ and $\theta_{j^*}$ as const. \\
\Thus $(b_i + b_o - 1)^{2}$ derivatives wrt. $\theta_j, \chi_k$, where $m_j, n_k \neq 0$ and $j \neq j^*$

Define
\begin{align}
	\vec{z}
&\coloneqq
	\vec{\theta} \oplus \vec{\chi}
\\
	\mathbb{J}^{(r)}
&\coloneqq
	\eval{
		\qty( \big. \grad_z f  )
	}_{\vec{z} = \vec{z}_r}
	\in \setComplex^{1 \times 2K}
\end{align}

Formally subdivide $\mathbb{J}$ in a $\theta$- and a $\chi$-Block for convenient notation; does not alter mathmematical structure (row-vector) as stated above.

Since some phases $\theta_j, \chi_k$ are ill-defined, not all of these derivatives do exist. This is handled by virtually setting the emerging expressions to zero.

Recap: We had:
\begin{align}
	J_{\theta, p} 
&= 
	\pdv{f}{\theta_p}
=
	\iunit \lambda^{-1}
	\qty[
		+m_p
		-
		\sum_{l=1}^{K}
			\sqrt{m_p n_l} \; u_{p,l} \; \exp[\Big. \iunit(\theta_p - \chi_l)]
	]
\qfor{} m_p \neq 0, p \neq j^*
\\
	J_{\chi, q} 
&= 
	\pdv{f}{\chi_q}
=
	\iunit \lambda^{-1}
	\qty[
		-n_q
		+
		\sum_{k=1}^{K}
			\sqrt{m_k n_q} \; u_{k,q} \; \exp[\Big. \iunit(\theta_k - \chi_q)]
	]
\qfor{} n_q \neq 0
\end{align}
and generic $p, q \in \set{1, \ldots, K}$

Get Hessian matrix elements:
\begin{align}
	\qty( S_r )
&=
	\eval{
		\begin{pmatrix}
			\pdv{J_{\theta}}{\vec{\theta}} &
			\pdv{J_{\chi  }}{\vec{\theta}} 
			\\
			\pdv{J_{\theta}}{\vec{\chi  }} &
			\pdv{J_{\chi  }}{\vec{\chi  }} 
		\end{pmatrix}
	}_{\vec{z} = \vec{z_r}}
	\in \setComplex^{2K \times 2K}
\end{align}


Note that only $(b_i + b_o - 1) \times (b_i + b_o - 1)$ elements of this matrix are actually defined (and needed). In order to preserve natural ordering of indices, we insert $\delta_{ij}$ where the above equations are undefined. This insertion will be justified in section \ref{sec:InsertionTheorems}.

The matrix elements (including the insertion) read:
\begin{align}
	\pdv{J_{\theta, p}}{\theta_{\alpha}}
&=
	\begin{cases}
		+
		\lambda^{-1}
		\delta_{p,\alpha}
		\sum_{l=1}^{K}
			\sqrt{m_p n_l} \; u_{p,l} \; \exp[\Big. \iunit(\theta_p - \chi_l)]
		&
			\text{if } \alpha : m_\alpha \neq 0, \alpha \neq j^*
		\\
			\delta_{p\alpha}
		&
			\text{else}
	\end{cases}
\\
	\pdv{J_{\theta, p}}{\chi_{\alpha}}
&=
	\begin{cases}
		-
		\lambda^{-1}
		\sqrt{m_p n_\alpha} \; u_{p,\alpha} \; \exp[\Big. \iunit(\theta_p - \chi_\alpha)]
	&
		\text{for } \alpha : m_\alpha \neq 0
	\\
		0
	&
		\text{else}
	\end{cases}
\\
	\pdv{J_{\chi, q}}{\theta_{\alpha}}
&=
	\begin{cases}
		-
		\lambda^{-1}
		\sqrt{m_\alpha n_q} \; u_{\alpha,q} \; \exp[\Big. \iunit(\theta_\alpha - \chi_q)]
	&
		\text{for } \alpha : m_\alpha \neq 0, \alpha \neq j^*
	\\
		0
	&	\text{else}		
	\end{cases}
\\
	\pdv{J_{\chi, q}}{\chi_{\alpha}}
&=
	\begin{cases}
		+
		\lambda^{-1}
		\delta_{q,\alpha}
		\sum_{k=1}^{K}
			\sqrt{m_k n_q} \; u_{k,q} \; \exp[\Big. \iunit(\theta_k - \chi_q)]
	&
		\text{for } \alpha : m_\alpha \neq 0
	\\
		\delta_{q\alpha}
	&
		\text{else}
	\end{cases}
\end{align}

Using the eqns (\ref{eqn:FourierLinkForward}) and (\ref{eqn:FourierLinkBackward}) (\ie Fourier Links), we can further collapse the diagonal elements:
\begin{align}
	\pdv{J_{\theta, p}}{\theta_{\alpha}}
&=
	\begin{cases}
			\lambda^{-1}
			\delta_{p,\alpha}
			m_p
		&
			\text{for } p : m_p \neq 0, p \neq j^*
		\\
			\delta_{p,\alpha}
		&
			\text{else}
	\end{cases}
\\
	\pdv{J_{\chi, q}}{\chi_{\alpha}}
&=
	\begin{cases}
		\lambda^{-1}
		\delta_{q,\alpha}
		n_q
	&
		\text{for } q : n_q \neq 0
	\\
		\delta_{q,\alpha}
	&
		\text{else}
	\end{cases}
\end{align}

This gives 
\begin{align}
	(S_j)
&=
	\diag(\vec{m'}, \vec{n'})
	+
	\begin{pmatrix}
		\mathds{O}_{K} & 
		\pdv{J_{\chi, q}}{\theta_{\alpha}}
		\\
		\pdv{J_{\theta, p}}{\chi_{\alpha}} &
		\mathds{O}_{K}
	\end{pmatrix}
=
	\begin{pmatrix}
		\diag(\vec{m'}) & F 			\\
		F\transp & \diag(\vec{n'})
	\end{pmatrix}
\end{align}
where
\begin{itemize}
\item $m'_j = 
\begin{cases}
	\smallfrac{m_j}{\lambda}			& \text{if } m_j \neq 0 \land j \neq j^*\\
	1	& \text{else}
\end{cases}$
	and 
$n'_k =
\begin{cases}
	\smallfrac{n_k}{\lambda} 			& \text{if } n_k \neq 0 \\
	1	& \text{else}
\end{cases}$ 
\item $\mathds{O}_{D} =
	\left.
	\mqty(
		0		& \ldots		& 0 \\
		\vdots	& \ddots 	& \vdots \\
		0		& \ldots		& 0
	)
	\right\} D \text{ rows}, D \text{ columns}$
\item By definition: 
	\begin{align}
	F_{pq}
&:=
	\begin{cases}
		\qty( \pdv{J_{\theta}}{\vec{\chi}} )_{pq} 
	&
		\text{for } p : m_p \neq 0 \land p \neq j^*; q : n_q \neq 0
	\\
		0 &\text{otherwise}
	\end{cases} \\
%&\;=
%	\begin{cases}
%		\qty( \pdv{J_{\chi}}{\vec{\theta}} )_{qp} &\text{for } p : m_p \neq 0 \land p \neq j^* ; q : n_q \neq 0 \\
%		0 &\text{otherwise}
%	\end{cases} \\
&\;=
	\begin{cases}
		\lambda^{-1}
		\qty( -\sqrt{m_p n_q} \; u_{p,q} \; \exp[\Big. \iunit(\theta_p - \chi_q)] )_{
			\substack{p, q = 1, \ldots, K}
		}
	& 
		\text{for same restrictions on }  p, q
	\\
		0 &\text{otherwise}
	\end{cases}
	\label{eqn:defF}
	\end{align}
	where the implied symmetry ($F$ vs. $F\transp$) is guaranteed by S: Hessian.\\
\end{itemize}

\section{Determinant of the Hessian}
\subsection{Determinant-Preserving Expansion}
\label{sec:InsertionTheorems}
\subsubsection{Theorem: Determinant-perserving expansion by unity block}
Let there be a matrix $M \in \setReals^{D \times D}$, as well as a matrix $M'$:
\begin{align}
	M'
&=
	\begin{pmatrix}
		\mathds{1}_{E} & \mathds{O}_{E \times D} \\
		\mathds{O}_{D \times E} & M
	\end{pmatrix}
\end{align}
where $\mathds{1}_{E}$ is the unit matrix of dimension $E$ and $\mathds{O}_{E \times D}$ is a null matrix of dimension $E \times D$.

Then,
\begin{equation}
	\forall E \in \setNaturals : \det M = \det M'
\end{equation}

\textbf{Proof:}\\
Regard the Matrix $M^{(n)}$:
\begin{align}
	M^{(n)}
&=
	\begin{pmatrix}
		\mathds{1}_{n} & \mathds{O}_{n \times D} \\
		\mathds{O}_{D \times n} & M
	\end{pmatrix}
\end{align}
\ie $M^{(E)} = M'$. The determinant of $M^{(1)}$ can be easily found from the Laplace expansion:

For an $n \times n$ matrix $B = (b_{i,j})$, and an arbitrary, fixed $r = 1, \ldots, n$, the determinant is given by:
\begin{align}
	\det B
&=
	\sum_{j=1}^{n}
		b_{rj} C_{rj}
\end{align}
where $C_{ij}$ is the cofactor to the $i^{\text{th}}$ row and $j^{\text{th}}$ column, \ie the determinant of the submatrix with the indexed row and column removed, times $(-1)^{i+j}$.

So for the given $M^{(1)}$, and choosing $r = 1$, the determinant reads:
\begin{align}
	\det M^{(1)}
&=
	1 \det M^{(0)} + 0 \\
&=
	\det M
\end{align}

Further, it can easily be seen that
\begin{align}
	\det M^{(n)} = \det M^{(n-1)} = \det M
\end{align}
and from this follows the claim.
\begin{flushright}
	\emph{q.e.d.}
\end{flushright}

\subsubsection{Theorem: Determinant-perserving expansion by unity infix}
Let $M$ be as above, with the blocks:
\begin{align}
	M
&=
	\begin{pmatrix}
		A & B \\ C & D
	\end{pmatrix}
\end{align}
and
\begin{align}
	M'
&=
	\begin{pmatrix}
		A & \mathds{O} & B \\
		\mathds{O} & \mathds{1}_{E} & \mathds{O} \\
		C & \mathds{O} & D
	\end{pmatrix}
\end{align}
where $\mathds{O}$ are null matrices of dimensions such that they match the blocks implicitly defined by $A, B, C, D, \mathds{1}_{E}$.

Then
\[ \forall E \in \setNaturals : \det M = \det M' \]

\textbf{Proof:}\\
From the same arguments as in the last theorem, but choosing $r$ such that it denotes the row above $D$.
\begin{flushright}
	\emph{q.e.d.}
\end{flushright}

\subsubsection{Justification $2K \times 2K$ Form}
We only need $\det \tilde{S}_r$, where $\tilde{S}_r \in \setComplex^{(b_i + b_o -1) \times (b_i + b_o -1)}$ only holds the derivatives as defined above, but not the insertion of $\delta_{ij}$. Using the Infix-Theorems we get $\det \tilde{S}_r = \det S_r$, \ie the insertion does not change the overall result.

\subsection{Alternative Form}
Statement:
\begin{align}
	\det S
&=
	\mqty|
		\begin{pmatrix}
			\diag(\vec{m'})	& \mathds{O}			\\
			F\transp			    & \mathds{1}_{K}
		\end{pmatrix}
		\cdot
		\begin{pmatrix}
			\mathds{1}_K		&	\qty(\diag(\vec{m'}))^{-1} F		\\
			\mathds{O}			&	\diag(\vec{n'}) - F\transp \qty(\diag(\vec{m'}))^{-1} F
		\end{pmatrix}
	|
	\label{eqn:detSAlt}
\end{align}

Holds because:
\begin{itemize}
\item Multiplication yields identity
\item Recall:
\begin{align}
	(S_j)
&=
	\begin{pmatrix}
		\diag(\vec{m'}) & F 			\\
		F\transp & \diag(\vec{n'})
	\end{pmatrix}
\end{align}
\end{itemize}

\subsection{Partial Evaluation}
Statement
\begin{align}
	\det S
&=
	m_{j^*}^{-1} \lambda^{-(b_i + b_o - 2)}
	\det[
		\mathds{1} -
		F\transp \cdot 
		\qty( \diag(\vec{m'}) )^{-1} \cdot
		F \cdot \qty( \diag(\vec{n'}) )^{-1}
	]
\label{eqn:detSPrimePartial}
\end{align}

Holds because:
\begin{align}
	\det S
&=
	\det\underbrace{
		\begin{pmatrix}
			\diag(\vec{m'})	& \mathds{O}			\\
			F\transp			    & \mathds{1}_{K}
		\end{pmatrix}
	}_{\text{triangular matrix}}
	\cdot 
	\det\underbrace{
		\begin{pmatrix}
			\mathds{1}_K		&	\qty(\diag(\vec{m'}))^{-1} F		\\
			\mathds{O}			&	\diag(\vec{n'}) - F\transp \qty(\diag(\vec{m'}))^{-1} F
		\end{pmatrix}
	}_{\text{triangular matrix}}
\end{align}
The determinant of a triangular matrix is the product of its diagonal elements. Hence:
\begin{align}
	\det S
&=
	\qty( \prod_{\substack{j : m_j \neq 0 \\ j \neq j^*}} \frac{m_j}{\lambda} )
	\det \qty(\Big. \diag(\vec{n'}) - F\transp \qty(\diag(\vec{m'}))^{-1} F)
\\
	\diag(\vec{n'}) - F\transp \qty(\diag(\vec{m'}))^{-1} F
&=
	\qty[
		\mathds{1}_K
		-
		F\transp \qty(\diag(\vec{m'}))^{-1} F \qty(\diag(\vec{n'}))^{-1}
	]
	\cdot
	\diag(\vec{n'})
\end{align}
\begin{align}
	\det S
&=
	\qty( \prod_{\substack{j : m_j \neq 0 \\ j \neq j^*}} \frac{m_j}{\lambda} )
	\det[
		\mathds{1} -
		F\transp \cdot 
		\qty( \diag(\vec{m'}) )^{-1} \cdot
		F \cdot \qty( \diag(\vec{n'}) )^{-1}
	]
	\qty( \prod_{          k : n_k \neq 0               } \frac{n_k}{\lambda} )
	\label{eqn:detSPenultimate}
\end{align}
\begin{align}
	\qty( \prod_{\substack{j : m_j \neq 0 \\ j \neq j^*}} \frac{m_j}{\lambda} )
	\qty( \prod_{          k : n_k \neq 0               } \frac{n_k}{\lambda} )
&=
	\frac{\lambda}{m_{j^*} \lambda^{b_i + b_o - 1}}
=
	m_{j^*}^{-1} \lambda^{-(b_i + b_o - 2)}
	\label{eqn:detSPrefactor}
\end{align}

\subsection{Reforming $F$}
\subsubsection{Definition: Square Root of a Matrix}
Let $\mathbb{M} \in \setReals^{K \times K}$, diagonizable with Eigenvalues $\lambda_i$ and decomposition
\begin{align}
	\mathbb{M}
&=
	\mathbb{Q} \; \diag(\lambda_1, \ldots, \lambda_K) \; \mathbb{Q}\transp 
\\
	\mathbb{Q}\transp
&=
	\mathbb{Q}^{-1}
\end{align}
Then its square root is defined as:
\begin{align}
	\sqrt{\mathbb{M}}
&=
	\mathbb{Q} \diag(\sqrt{\lambda_1}, \ldots, \sqrt{\lambda_K}) \mathbb{Q}\transp 
\end{align}
such that it holds:
\begin{align}
	\sqrt{\mathbb{M}}  \sqrt{\mathbb{M}} &= \mathbb{M}
\end{align}

In particular, for diagonal matrices, $\mathbb{Q} = \mathds{1}$

\subsubsection{Definition: Reduced DFT-matrix}
Let $\tilde{\mathbb{U}}$ be the reduced DFT-matrix:
\begin{align}
	\tilde{u}_{ij}
&=
	\begin{cases}
		0			&\qq*{for } j = j^* \\
		0			&\qq*{for } m_j = 0 \\
		0			&\qq*{for } n_k = 0 \\
		-u_{j,k}		&\qotherwise*
	\end{cases}
\end{align}

\subsubsection{Application of Definitions}
Statement:
\begin{align}
	F
&=
		\sqrt{\diag(\vec{m'})}
		\exp [\diag\qty(+\iunit\vec{\theta})]
	\cdot
		\tilde{\mathbb{U}}
	\cdot
		\sqrt{\diag(\vec{n'})}
		\exp [\diag\qty(\big. -\iunit\vec{\chi})]
\end{align}

Holds because:
\begin{itemize}
\item Trivial rewrite of eqn. (\ref{eqn:defF}) (definition of $F$)
\item {\color{red} It might be sufficient to set only the reference column of $\tilde{\mathbb{U}}$ to zero; the rest should be covered by the multiplication with zero occupations. However, it is necessary to artificially suppress the reference mode and flip the sign.}
\item $\sqrt{\diag{\vec{m'}}}$ and $\sqrt{\diag{\vec{n'}}}$ both give a factor of $\sqrt{\lambda^{-1}}$, restoring that prefactor.
\item $\tilde{\mathbb{U}}$ recreates the zeros that $\vec{m'}, \vec{n'}$ do no longer hold.
\end{itemize}

\subsection{Reformed $F$ in the Determinant}
Let:
\begin{align}
	S'
&\coloneqq
	\mathds{1} -
	F\transp \cdot 
	\qty( \diag(\vec{m'}) )^{-1} \cdot
	F \cdot \qty( \diag(\vec{n'}) )^{-1}
\end{align}
Statement:
\begin{align}
	\det S'
&=
	\qty( \prod_{k: n_k \neq 0} \exp(-2\iunit \chi_k) )
	\det\qty[\bigg.
		\exp(\Big. 
			2\iunit 
			\diag(\vec{\chi})
		)
		-
		\tilde{\mathbb{U}}\transp
		\exp(
			2\iunit 
			\diag(\vec{\theta})
		)
		\tilde{\mathbb{U}}
	]
\end{align}

Holds because:
\begin{align}
	S'
&=
	\mathds{1} -
	F\transp \cdot 
	\qty( \diag(\vec{m'}) )^{-1} \cdot
	F \cdot \qty( \diag(\vec{n'}) )^{-1}
\end{align}
regard only well defined matrix element $s_{pq}$ where $m_p \neq 0, p \neq j^*, n_q \neq 0$
\begin{align}
	s'_{pq}
&=
	\nonumber
	\delta_{pq} -
	\underbrace{
		\lambda^{-1}
		\qty( -\sqrt{m_p n_q} \; u_{p,q} \; \exp[\Big. \iunit(\theta_p - \chi_q)] )
	}_{\text{from } F\transp}
	\underbrace{
		\frac{\lambda}{m_p}
	}_{\text{from } \diag(\vec{m'})^{-1}}
	\ldots
	\\ 
	& \qquad\qquad
	\underbrace{
		\lambda^{-1}
		\qty( -\sqrt{m_p n_q} \; u_{p,q} \; \exp[\Big. \iunit(\theta_p - \chi_q)] )
	}_{\text{from } F}
	\underbrace{
		\frac{\lambda}{n_q}
	}_{\text{from } \diag(\vec{n'})^{-1}} \\
\end{align}
\begin{align}
	s'_{pq}
&=
	\delta_{pq} -
	\frac{\lambda^2}{\lambda^2}
	\frac{m_p n_q}{m_p n_q}
	u_{p,q}^{2}
	\exp(2\iunit(\theta_p - \chi_q)) \\
&=
	\exp(-2\iunit\chi_q) \qty[ \Big.
		\exp(2\iunit\chi_q)\delta_{pq}
		-
		u_{p,q}^{2}
		\exp(2\iunit \theta_p)
	]
\end{align}
Given that the ill-defined matrix elements should evaluate to zero (guaranteed by the choice of $\tilde{\mathbb{U}}$), we find this representation of the matrix:
\begin{align*}
	S'
&=
	\exp[\Big. -2\iunit\diag(\vec{\chi})]
	\qty[
		\exp(\Big. 2\iunit\diag(\vec{\chi}))
		-
		\tilde{\mathbb{U}}\transp
		\exp(\Big. 2\iunit\diag(\vec{\theta}))
		\tilde{\mathbb{U}}
	]
\end{align*}
The determinant can again be evaluated factor-wise, giving the statement.

\section{Complete Det of the Hessian}
\subsection{Collected Terms}
Plugging this in the results (\ref{eqn:detSPenultimate})/(\ref{eqn:detSPrefactor}), we get:
\begin{align}
	\det S
&=
	m_{j^*}^{-1} \lambda^{-(b_i + b_o - 2)}
	\qty( \prod_{k: n_k \neq 0} \exp(-2\iunit \chi_k) )
	\det\qty[\bigg.
		\exp(\Big. 
			2\iunit 
			\diag(\vec{\chi})
		)
		-
		\tilde{\mathbb{U}}\transp
		\exp(
			2\iunit 
			\diag(\vec{\theta})
		)
		\tilde{\mathbb{U}}
	]
\end{align}

\subsection{Numerics Optimization}
In implementations, the numeric values of $\exp(\iunit \theta_j), \exp(\iunit \chi_k)$ appear as natural variables, while their arguments $\vec{\theta}, \vec{\chi}$ need to be computed in a step of its own, introducing a new source of error. The exponentials are of unity magnitude, therefore summation/multiplication is well conditionned.

It is thus advisable to introduce new variables and split terms into an exact contribution (proportional to exact quantities derived from $\vec{m}, \vec{n}$) and approximate quantities (derived from $\exp(\iunit \theta_j), \exp(\iunit \chi_k)$). We get:
\begin{align}
	\mu_j &= \exp(\iunit \theta_j) & \text{for } m_j \neq 0 \\
	\nu_k &= \exp(\iunit \chi  _k) & \text{for } n_k \neq 0 \\
\end{align}
\begin{align}
	\det S
&=
	\underbrace{
		m_{j^*}^{-1} \lambda^{-(b_i + b_o - 2)}
	}_{\eqqcolon \Omega(\vec{m}, \vec{n})}
	\underbrace{
		\qty( \prod_{k: n_k \neq 0} \nu_k^{-2} )
		\det\qty[\bigg.
				\diag^2(\vec{\nu})
			-
			\tilde{\mathbb{U}}\transp
			\diag^2(\vec{\mu})
			\tilde{\mathbb{U}}
		]
	}_{=\Upsilon(\vec{\mu}, \vec{\nu})}
\end{align}


\section{Complete Amplitude Formula}
\subsection{Collect Terms}
\begin{itemize}
\item Eqn. (\ref{eqn:BSA_preSPA}):
	\begin{align*}	
		A_F(\vec{m}, \vec{n})
	=
		2\pi C \, \delta_{M,N} \,
		\int_{0}^{2\pi}
			\DD{\theta} \DD{\chi}
		\exp[
			\iunit \qty(
				\vec{n} \cdot \vec{\chi}
				-
				\vec{m} \cdot \vec{\theta}
			)]
		\exp[
			{\textstyle \sum_{s,t=1}^{K} }
				\sqrt{m_s, n_t} \;
				u_{s,t}
				\exp(\Big. \iunit(\theta_s - \chi_t) )
		]
	\end{align*}
\item Eqn. (\ref{eqn:DefC})
	\begin{align*}
		C
	&=
		(-1)^{b_o}
		(2\pi)^{-(b_i + b_o)}
		\qty( \prod_{j : m_j \neq 0}
			\sqrt{\frac
				{ m_j! }
				{ m_j^{m_j} }
			}
		)
		\qty( \prod_{k : n_k \neq 0}
			\sqrt{\frac
				{ n_k! }
				{ n_k^{n_k} }
			}
		)
	\end{align*}
\item Eqn. (\ref{eqn:Def_SPA})
	\begin{align*}
		\oint_{\gamma} \dd[n]{\vec{z}}
			\exp[{\color{blue} -}\lambda f(\vec{z})]
	&\approx
		\qty( \frac
			{2\pi}
			{\lambda}
		)^{\frac{n}{2}}
		\sum_{r}
			\frac
			{\exp(-\lambda f(\vec{z}_r))}
			{\sqrt{\det S_r}}
	\end{align*}
\end{itemize}

From these, get directly:
\begin{multline}
	A_F(\vec{m}, \vec{n})
\approx
	2\pi \delta_{M,N} \,	
	(-1)^{b_o}
	(2\pi)^{-(b_i + b_o)}
	\underbrace{
		\qty( \prod_{j : m_j \neq 0}
			\sqrt{\frac
				{ m_j! }
				{ m_j^{m_j} }
			}
		)
		\qty( \prod_{k : n_k \neq 0}
			\sqrt{\frac
				{ n_k! }
				{ n_k^{n_k} }
			}
		)
	}_{\eqqcolon \Pi}
	\qty( \frac
		{2\pi}
		{\lambda}
	)^{\frac{b_i + b_o - 1}{2}}
\\
	\sum_{r}
		\frac
		{\exp[
			\iunit
			\qty(
				\vec{n} \cdot \vec{\chi^{(r)}}
				- 
				\vec{m} \cdot \vec{\theta^{(r)}}
			)
		+
		\qty(
			\sum_{s,t=1}^{K}
			\sqrt{m_s n_t} \;
			u_{s,t}
			\exp(\big.
				\iunit(\theta_s^{(r)} - \chi_t^{(r)})
			)
		)
		]}
		{\sqrt{\Omega \Upsilon_r}}
\end{multline}

Regard the exponential's argument. From (\ref{eqn:DefXY}), and with new symbols $\mu_j, \nu_k$, get:
\begin{align}
	\mathcal{E}
&\coloneqq
	\exp[
			\iunit
			\qty(
				\vec{n} \cdot \vec{\chi^{(r)}}
				- 
				\vec{m} \cdot \vec{\theta^{(r)}}
			)
		+
		\underbrace{\qty(
			\sum_{s,t=1}^{K}
			\sqrt{m_s n_t} \;
			u_{s,t}
			\exp(\big.
				\iunit(\theta_s^{(r)} - \chi_t^{(r)})
			)
		)}_{=N}
	] \\
&=
	\frac
		{\prod_{k : n_k \neq 0}                  \qty(\nu_k^{(r)})^{n_k}}
		{\prod_{j : m_j \neq 0 \land j \neq j^*} \qty(\mu_j^{(r)})^{m_j}}
	\exp(N)
\end{align}

Collect terms:
\begin{align}
		A_F(\vec{m}, \vec{n})
&\approx
	(-1)^{b_o} \,	
	\delta_{M,N} \,
	(2\pi \lambda)^{-\frac{(b_i + b_o - 1)}{2}}
	\Pi
	\exp(N)
	\underbrace{
		\sqrt{m_{j^*} \lambda^{b_i + b_o - 2}}
	}_{\Omega^{-\smallfrac{1}{2}}}
	\underbrace{
		\sum_{r}	
		\frac
			{\prod_{k : n_k \neq 0}                  \qty(\nu_k^{(r)})^{n_k}}
			{\prod_{j : m_j \neq 0 \land j \neq j^*} \qty(\mu_j^{(r)})^{m_j}}
		\frac{1}{\sqrt{\Upsilon_r}}
	}_{\eqqcolon \Sigma} \\
&=
	(-1)^{b_o} \,	
	\delta_{M,N} \,
	(2\pi)^{-\frac{(b_i + b_o - 1)}{2}}
	\sqrt{
		\frac{m_{j^*}}{\lambda}
	}
	\Pi
	\exp(N)
	\Sigma \\
&=
	(-1)^{b_o} \,	
	\delta_{M,N} \,
	(2\pi)^{-\frac{(b_i + b_o - 1)}{2}}
	\underbrace{
		\qty[
			\qty(\prod_{\substack{j : m_j \neq 0\\\land j \neq j^*}} m_j)
			\qty(\prod_{          k : n_k \neq 0                   } n_k)
		]^{-\smallfrac{1}{2}}
	}_{\sqrt{\smallfrac{m_{j^*}}{\lambda}}}
	\Pi
	\exp(N)
	\Sigma
\end{align}

{\color{red}Note that replacing the product with $\sqrt{\smallfrac{m_{j^*}}{\lambda}}$ introduces a too strong dependence on $m_{j^*}$ and gives wrong results for small $m_{j^*}$}

\subsection{Stirling Approximation: A more computationally performant form for $\Pi$}
Use Stirling's Formula:
\begin{align}
	n!
&=
	\sqrt{2\pi n} n^n \exp(-n)
\end{align}
to evaluate $\Pi$:

\begin{align}
	\Pi
&=
	\qty( \prod_{j : m_j \neq 0}
		\sqrt{\frac
			{ m_j! }
			{ m_j^{m_j} }
		}
	)
	\qty( \prod_{k : n_k \neq 0}
		\sqrt{\frac
			{ n_k! }
			{ n_k^{n_k} }
		}
	) \\
&=
	\qty( \prod_{j : m_j \neq 0}
		\sqrt{\frac
			{ \sqrt{2\pi m_k} m_j^{m_j} \exp(-m_j) }
			{ m_j^{m_j} }
		}
	)
	\qty( \prod_{k : n_k \neq 0}
		\sqrt{\frac
			{ \sqrt{2\pi n_k} n_k^{n_k} \exp(-n_k) }
			{ n_k^{n_k} }
		}
	) \\
&=
	\qty( \prod_{j : m_j \neq 0}
		\sqrt{
			\sqrt{2\pi m_k}
		}
	)
	\exp(-\smallfrac{N}{2})
	\qty( \prod_{k : n_k \neq 0}
		\sqrt{
			\sqrt{2\pi n_k}
		}
	)
	\exp(-\smallfrac{N}{2}) \\
&=
	(2\pi)^{\frac{b_i + b_o}{4}}
	\lambda^{\smallfrac{1}{4}}
	\exp(-N)
\end{align}


Collect terms (form for big values fo $m_{j^*}$):
\begin{align}
		A_F(\vec{m}, \vec{n})
&=
	(-1)^{b_o} \,	
	\delta_{M,N} \,
	(2\pi)^{-\frac{(b_i + b_o - 1)}{2}}
	\sqrt{\smallfrac{m_{j^*}}{\lambda}}
	(2\pi)^{\frac{b_i + b_o}{4}}
	\lambda^{\smallfrac{1}{4}}
	\exp(-N)
	\exp(N)
	\Sigma \\
&=
	(-1)^{b_o} \,	
	\delta_{M,N} \,
	(2\pi)^{-\frac{b_i + b_o - 2}{4}}
	\sqrt{m_{j^*}}
	\lambda^{-\smallfrac{1}{4}}
	\Sigma
\end{align}

\end{document}