\documentclass[
	english,
	a4paper,
	fontsize=10pt,
	parskip=half,
	titlepage=true,
	DIV=12,
	final
]{scrreprt}


%==============================================================================%
% PACKAGES
%
% Standard text formatting
\usepackage[utf8]{inputenc}
\usepackage{babel}
\usepackage[T1]{fontenc}
\usepackage{lmodern}
\usepackage{microtype}
\usepackage{ragged2e}

\usepackage{csquotes}
\usepackage{xspace}

\usepackage{placeins}	% FloatBarrier.
\usepackage{url}
\usepackage[bf, format=plain]{caption}

\usepackage{hyperref}
\hypersetup{
    colorlinks,
    citecolor=black,
    filecolor=black,
    linkcolor=black,
    urlcolor=black
}

% gfx
\usepackage{wrapfig}

% tables
\usepackage{tabularx}
\usepackage{booktabs}
\usepackage{multicol}
\usepackage{multirow}
\usepackage{makecell}
\usepackage{color, colortbl}

% math
\usepackage{amsmath}
\usepackage{amssymb}
\usepackage{dsfont}
\let\olddiv\div
\usepackage[arrowdel]{physics}
\usepackage{mathtools}

% indexes, links, page format
\usepackage{scrlayer-scrpage}

% misc
\usepackage[super]{nth}
\usepackage[
	output-decimal-marker={.},
	input-symbols = {()},  			% do not treat "(" and ")" in any special way
	group-digits  = true  			% guess what.
]{siunitx}
\usepackage{minted}

%==============================================================================%
% GLOBAL MACROS
%

% Document properties
\newcommand{\myName}{Stefan Hartinger\xspace}
\newcommand{\myTitle}{Master Thesis Notes: Deriving a Zero Occupancy Formula\xspace}

\addtokomafont{labelinglabel}{\sffamily}

% Text abbreviations
\newcommand*{\ie}{i.\,e.\xspace}
\newcommand*{\eg}{e.\,g.\xspace}

% Misc Symbols
\newcommand*{\thus}{\ensuremath{\rightarrow}\xspace}
\newcommand*{\Thus}{\ensuremath{\Rightarrow}\xspace}

% Tables
\newcommand*{\tabcrlf}{\\ \hline}			% actually still allows for optional argument

% Math
\newcommand*{\numberthis}{\addtocounter{equation}{1}\tag{\theequation}}

\newcommand*{\smallfrac}  [2]{\ensuremath{{}^        {#1} \!/_        {#2}}}
\newcommand*{\smallfracrm}[2]{\ensuremath{{}^{\mathrm{#1}}\!/_{\mathrm{#2}}}}

\newcommand*{\transp}{\ensuremath{^\intercal}}

\newcommand*{\iunit}{\ensuremath{\mathrm{i}}}

\newcommand*{\setNaturals} {\ensuremath{\mathbb{N}}}
\newcommand*{\setIntegers} {\ensuremath{\mathbb{Z}}}
\newcommand*{\setReals}    {\ensuremath{\mathbb{R}}}
\newcommand*{\setRationals}{\ensuremath{\mathbb{Q}}}
\newcommand*{\setComplex}  {\ensuremath{\mathbb{C}}}

\newcommand*{\Lag}{\ensuremath{\mathcal{L}}\xspace}
\newcommand*{\Ham}{\ensuremath{\mathcal{H}}\xspace}

%\newcommand*{\Poisson}[2]{\ensuremath{\left\{ {#1}, {#2} \right\}}}
% physics has \pb which is poisson bracket
% also use alias acom: anticommutator, which is exactly the same.

\newcommand*{\equalCond}{  \mathop{=}\limits^!  }

\DeclareMathOperator{\arsinh}{arsinh}
\DeclareMathOperator{\diag}{diag}

%==============================================================================%
% GLOBAL PARAMTERS
%

\title{\myTitle}
\author{\myName}
\date{\today}

% header, footer
\clearpairofpagestyles
	\cfoot
		[\pagemark]
		{\pagemark}
	\ohead
		[\myTitle, \myName]
		{\myTitle, \myName}
\pagestyle{scrheadings}

%==============================================================================%
% THE REAL STUFF
%	
\begin{document}
\tableofcontents
\newpage

\chapter{Bunched Incoming Zero-Occupancy Treatment}
\section{Assumptions, Terminology}
For starters, we will assume that only $m_1 = 0$. This shall be denoted as \emph{single zero occupancy} or \emph{SZO}.

From there on, the \emph{bunched zero occupancy} (\emph{BZO}) can be treated:
\begin{align}
	\exists b^{*} : \forall j > b^{*} : m_j > 0 \\
	\forall m_j \leq b^{*} : m_j = 0
\end{align}
This means, the \emph{incoming modes} $j = 1 \ldots b^{*}$ have occupancy zero. 

In both, SZO and BZO, we will further assume that the outgoing modes are still all occupied:
\begin{align}
	\forall j : n_j > 0
\end{align}

For the first cases, assume further only one mode to be of zero occupancy, \ie $b^* = 1$. This will be denoted as \emph{single zero occupancy} or \emph{SZO}.

From there on, the generic case follows:
\begin{itemize}
\item The scenario is assumed to be symmetric in $m, n$ -- verify!
\item non-bunched configurations can be obtained by re-ordering coordinates -- verify!
\end{itemize}

\subsection{Notes on L'Hopital}
\begin{itemize}
\item It should be possible to obtain BZO solutions from SZO. Regard only one degree of freedom, treat the other ZO variables as fixed parameters.
\item After obtaining a const factor pertaining the terms depending on the SZO-zero, proceed with the next DoF.
\end{itemize}


\section{Starting Point}
Statement:
\begin{align}
	A_F(\vec{m}, \vec{n})
&=
	\eval{\qty(
			\prod_{j}
			\frac
				{1}
				{\sqrt{m_j! \; n_j!}}
			\pdv[m_j]{x_j}
			\pdv[n_j]{y_j}
		) \exp(\vec{x} \, \mathbb{U} \, \vec{y})
	}_{\vec{x} = \vec{y} = \vec{0}}
\label{eqn:BSA_raw}
\end{align}

Notes:
\begin{itemize}
\item was given
\end{itemize}


Split into BZO form:
\begin{align}
	A_F(\vec{m}, \vec{n})
&=
	\eval{
		\qty(
			\prod_{j=1}^{b^*}
			\frac
				{1}
				{\sqrt{n_j!}}
			\pdv[n_j]{y_j}
		) 
		\qty(
			\prod_{j=b^* + 1}^{K}
			\frac
				{1}
				{\sqrt{m_j! \; n_j!}}
			\pdv[m_j]{x_j}
			\pdv[n_j]{y_j}
		)
		\exp(\vec{x} \, \mathbb{U} \, \vec{y})
	}_{\vec{x} = \vec{y} = \vec{0}}
\label{eqn:BSA_raw_BZO}
\end{align}

Holds because:
\begin{itemize}
\item $0! = 1$
\item $\pdv[0]{z} = 1$
\end{itemize}

\section{BSA in integral form}
Equation:
\begin{align}
	A_F(\vec{m}, \vec{n})
&=
	\qty(
		\prod_{j=1}^{b^*}
		\frac
			{\sqrt{n_j !}}
			{2\pi\iunit}
		\oint_{\gamma}
			\frac
				{\dd{y_j}}
				{y_j^{n_j + 1}}
	)
	\underbrace{
		\qty(
			\prod_{j=b^* + 1}^{K}
			\frac
				{\sqrt{m_j! \; n_j!}}
				{-4\pi^{2}}
		)
		\oint_{\gamma}
			\qty(
				\frac
				{\dd{x_j} \dd{y_j}}
				{x_j^{m_j+1}  y_j^{n_j+1}}
			)
			\exp(\vec{x} \, \mathbb{U} \, \vec{y})
		}_{=A^*}
	\label{eqn:BSA_Int_BZO}
\end{align}

Holds because:
\begin{itemize}
\item principles of the generic derivation with applied with setting some $m_j$ to zero.
\end{itemize}


\chapter{Backup}
The below has not lead (yet) to any usable results. Keep in stock, just in case
\section{BSA in Integral From}
Statement:
\begin{align}
	A_F(\vec{m}, \vec{n})
&=
	\qty(
		\prod_{j=1}^{K}
		\frac
			{\sqrt{m_j! \; n_j!}}
			{-4\pi^{2}}
	)
	\oint_{\gamma}
		\qty(
			\frac
			{\dd{x_j} \dd{y_j}}
			{x_j^{m_j+1}  y_j^{n_j+1}}
		)
		\exp(\vec{x} \, \mathbb{U} \, \vec{y})
	\label{eqn:BSA_Int}
\end{align}

Notes
\begin{itemize}
\item Copy \& Paste from Max
\item Zero in any $m_j$ does not affect this, as rewriting is based on Cauchy's formula, which initially
	does not involve derivatives after all:
	\begin{equation}
		f(0)
	=
		\frac{1}{2\pi \iunit}
		\oint_{\gamma} \dd{z}
			\frac{f(z)}{z}
	\end{equation}
\end{itemize}

\section{Change of Variables I: how not to do it}
\subsection{Absorbing Prefactors in BSA}
Statement:
\begin{multline}
	A_F(\vec{m}, \vec{n})
=
	\qty(
		\prod_{j=1}^{K}
		\frac
			{\sqrt{m_j! \; n_j!}}
			{-4\pi^{2}}
	)
	\oint_{\gamma}
		\qty(
			\prod_{j=1}^{K}
			\dd{x_j} \dd{y_j}
		)  \cdot \ldots
\\
		\ldots \exp[
			-\sum_{j=1}^{K} 
			\qty[ \Big.
				(m_j + 1) \log(x_j) +
				(n_j + 1) \log(y_j)
			]
		]
		\exp(\vec{x} \, \mathbb{U} \, \vec{y})
	\label{eqn:BSA_IntAbsorbed}
\end{multline}

Notes:
\begin{itemize}
\item Copy \& Paste from Max
\item This form is not affected by setting either variable to zero
\item But there already exists no solution to $\log(0)$
\end{itemize}

\subsection{Apply Change of Coordinates}
Using polar coordinates:
\begin{align}
	\begin{cases}
	x_j &= \sqrt{m_j} \exp( \iunit \theta_j) \\
	y_j &= \sqrt{n_j} \exp(-\iunit \chi  _j)
	\end{cases}
	\label{eqn:DefXY}
\end{align}

in eqn. (\ref{eqn:BSA_IntAbsorbed}) gives:
\begin{multline}
	A_F(\vec{m}, \vec{n})
=
	\qty(
		\prod_{j=1}^{K}
		\frac
			{1}
			{-4\pi^{2}}
		\sqrt{\frac
			{m_j!      \; n_j!}
			{m_j^{m_j} \; n_j^{n_j} }
		}
	)
	\int_{0}^{2\pi} \dd{\vec{\theta}}
	\int_{0}^{2\pi} \dd{\vec{\chi}}\ \ldots \\
		\ldots		
		\exp[\iunit (\vec{n} \cdot \vec{\chi}  -  \vec{m} \cdot \vec{\theta})] \cdot 
		\exp[\sum_{k,l=1}^{K} \sqrt{m_k n_l} \; u_{k,l} \; \exp[\Big.\iunit(\theta_k - \chi_l)] ]
\end{multline}

Notes:
\begin{itemize}
\item Copy \& Paste from Max
\item This cannot be used here since setting any $m_j$ to zero would yield several undefined expressions:
	\begin{itemize}
	\item $\frac{\text{const}}{0}$ contribution in the prefactor 
	\item $0^0$ component in denominator of prefactor
	\item $\log(0)$ in eqn. \ref{eqn:BSA_IntAbsorbed}
	\end{itemize}
	
\end{itemize}


\section{Change of variables II: electric boogaloo}
\subsection{Plug in substitution}
Use same coordinates as in eqn. \ref{eqn:DefXY}, but in \ref{eqn:BSA_Int}. Get:
\begin{align}
	A_F(\vec{m}, \vec{n})
&=
	\qty(
		\prod_{j=1}^{K}
		\frac
			{\sqrt{m_j! \; n_j!}}
			{-4\pi^{2}}
	)
	\oint_{\gamma}
		\qty(
			\frac
			{\dd{x_j} \dd{y_j}}
			{[\sqrt{m_j} \exp( \iunit \theta_j)]^{m_j+1}
			 [\sqrt{n_j} \exp(-\iunit \chi  _j)]^{n_j+1}}
		)
		\exp(\vec{x} \, \mathbb{U} \, \vec{y})
\end{align}

separate the components $j=1 \ldots b^{*}$, assuming $m_j = 0$:
\begin{multline}
	A_F(\vec{m}, \vec{n})
=
	\qty[
		\prod_{j=1}^{b^*}
		\frac{1}{-4\pi^{2}}
		\frac
			{m_j! \; n_k!}
			{[\sqrt{m_j} \exp( \iunit \theta_1)]^{m_1 + 1}
			 [\sqrt{n_j} \exp(-\iunit \chi  _1)]^{n_1 + 1}
			}
	]
	\qty(
		\prod_{j=b^{*} + 1}^{K}
		\frac
			{\sqrt{m_j! \; n_j!}}
			{-4\pi^{2}}
	)
\\
	\oint_{\gamma}
		\qty(
			\frac
			{\dd{x_j} \dd{y_j}}
			{[\sqrt{m_j} \exp( \iunit \theta_j)]^{m_j+1}
			 [\sqrt{n_j} \exp(-\iunit \chi  _j)]^{n_j+1}}
		)
		\exp(\vec{x} \, \mathbb{U} \, \vec{y})
\end{multline}

\subsection{Do L'Hopital}
Focus on the prefactor, denote non-singular contribution (\ie the part where $j > b^{*}$) as $A^{*}$:
\begin{align}
	A_F(\vec{m}, \vec{n})
=
	\qty[
		\prod_{j=1}^{b^*}
		\frac{1}{-4\pi^{2}}
		\frac
			{m_j! \; n_k!}
			{[\sqrt{m_j} \exp( \iunit \theta_1)]^{m_1 + 1}
			 [\sqrt{n_j} \exp(-\iunit \chi  _1)]^{n_1 + 1}
			}
	] A^*
\end{align}

\subsubsection{Theorem}
\begin{displayquote}
L'Hôpital's rule states that for functions $f$ and $g$ which are differentiable on an open interval $I$ except possibly at a point $c$ contained in $I$, if 
$\lim _{x\to c}f(x)=\lim _{x\to c}g(x) = 0 \text{ or } \pm \infty$,
$g'(x) \neq 0$ for all $x$ in $I$ with $x \neq c$, and 
$\lim_{x \to c}
	\frac
		{f'(x)}
		{g'(x)}$ 
exists, then
\[
	\lim _{x\to c}{
		\frac 
			{f(x)}
			{g(x)}
	}
=
	\lim _{x\to c}{
		\frac
			{f'(x)}
			{g'(x)}
	}
\]
\end{displayquote}
(Cf. \url{https://en.wikipedia.org/wiki/L%27H%C3%B4pital%27s_rule}.)

\subsubsection{Applicability I: function form}
\begin{itemize}
\item $f$: differentiable on open interval:
	\begin{itemize}
	\item $f = \prod_{j=1}^{b^*} m_j! \; n_j! = 0$
	\item continuous wrt. $m_j$, interpreted as real values
	\end{itemize}
\item $g$: differentiable on open interval:
	\begin{itemize}
	\item $g = [\sqrt{m_j} \exp( \iunit \theta_1)]^{m_1 + 1}
			 [\sqrt{n_j} \exp(-\iunit \chi  _1)]^{n_1 + 1} = 0$
	\item continuous wrt. $m_j$, interpreted as real values
	\end{itemize}
\item[\Thus] okay
\end{itemize}

\subsubsection{Appicabilty II: Derivative of $f$}
Derivative $f'$: needs to exist
\begin{itemize}
\item Multivariate case will be treated later; regard only SZO
\item $f' = \dv{m_1} (m_1! \; n_1!) = n_1! \dv{m_1}(m_1!)$
\item Use real-value continuation of factorial: Gamma function: $\Gamma(n) = (n-1)!$
\item $\Gamma(z) = \int_{0}^{\infty} \dd{x} x^{z-1} \exp(-x)$
\item Hence:
\begin{align}
	f'
&=
	n_1! \dv{m_1}\qty[ \big. \Gamma(m_1 + 1) ] 
	\qquad\text{(renaming } m_1 \thus z \text{ in the next lines for convenience)}
	\\
&=
	n_1! \dv{z}
	\int_{0}^{\infty} \dd{x} 
		x^{z} \exp(-x)
	\\
&=
	n_1!
	\int_{0}^{\infty} \dd{x}
		x^{z} \log(x) \exp(-x)
\end{align}	
\item Following this (or rather, using Wolfram Alpha: \\
	\url{https://www.wolframalpha.com/input/?i=d%2Fdx+x%21})	 \\
	we get
\begin{align}
	\dv{z}(z!) = \Gamma(z + 1) \Psi(z + 1)
\end{align}
	where $\Psi$ is the \emph{digamma function}\\
	(cf. \url{https://mathworld.wolfram.com/DigammaFunction.html} and \\
	     \url{https://mathworld.wolfram.com/PolygammaFunction.html}):
\begin{align}
	\Psi(n)
&=
	H_{n-1} - \gamma
\end{align}
where
\begin{align}
	H_{n}
&=
	\sum_{k=1}^{n}
		\frac{1}{k}
	\qfor{} n \in \setNaturals
&
	\text{\enquote{harmonic numbers}}
\\
	H_0
&=
	0
\\
	\gamma
&=
	\lim_{n \to \infty} \qty(
		-\ln(n)
		+ \sum_{k=0}^{n}
			\frac{1}{k}
	)
\approx
	0.577215
&
	\text{\enquote{Euler–Mascheroni constant}}
\end{align}

\item maybe useful later: alternate form:
\begin{align}
	\Psi(z)
&=
	\sum_{n=0}^{\infty}
		\frac{1}{n+1}
	\sum_{k=0}^{n}
		\qty(-1)^{k}
		\begin{pmatrix}
			n \\ k
		\end{pmatrix}
		\ln(z + k)
\end{align}
\item get
\begin{align}
	f'
&=
	n_1 ! \;
	\Gamma(m_1 + 1) \,
	\Psi  (m_1 + 1) \\
&=
	n_1 ! \; m_1 ! \;
	\qty( H_{m_1} - \gamma )
\end{align}
\end{itemize}

\subsubsection{Applicability of $g$}
Derivative $g'$: need to exist, and be nonzero
\begin{itemize}
\item $g(z) = [\sqrt{z} \exp( \iunit \theta_1)]^{z + 1}
			 [\sqrt{n_j} \exp(-\iunit \chi  _1)]^{n_1 + 1} = 0$
	where $m_1$ was relabelled $z$ for convenience
\item get
\begin{align}
	\underbrace{
		[\sqrt{n_j} \exp(-\iunit \chi  _1)]^{n_1 + 1}
	}_{\xi}
	\dv{z}
		[\sqrt{z} \exp( \iunit \theta_1)]^{z + 1}
&=
	\xi
	\qty( \bigg.
		\sqrt{z} \exp( \iunit \theta_1)
	)^{z+1}
	\qty(
		\frac
			{z+1}
			{2z}
		+
		\log(\sqrt{z} \exp( \iunit \theta_1) )
	) \\
&=
	\xi
	\exp[ \big. \iunit (z+1) \theta_1 ]
	\qty[
		\frac{z+1}{2}
		z^{\frac{z-1}{2}}
		+
		z^{\frac{z+1}{2}}
		\log(\sqrt{z} \exp( \iunit \theta_1) )
	]
\end{align}

\item L'Hopital could be applied again ($a \cdot b = \frac{a}{1/b}$), but this does not yield a solveable form.
\item Problem is due to the $z^z$ term, which always gives a new $\log(z)$ term
\end{itemize}		


\subsection{Evaluate $m_1 = 0$}

\end{document}