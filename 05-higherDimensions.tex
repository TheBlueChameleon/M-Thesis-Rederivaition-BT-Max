\documentclass[
	english,
	a4paper,
	fontsize=10pt,
	parskip=half,
	titlepage=true,
	DIV=12,
	final
]{scrreprt}


%==============================================================================%
% PACKAGES
%
% Standard text formatting
\usepackage[utf8]{inputenc}
\usepackage{babel}
\usepackage[T1]{fontenc}
\usepackage{lmodern}
\usepackage{microtype}
\usepackage{ragged2e}

\usepackage{csquotes}
\usepackage{xspace}

\usepackage{placeins}	% FloatBarrier.
\usepackage{url}
\usepackage[bf, format=plain]{caption}

\usepackage{hyperref}
\hypersetup{
    colorlinks,
    citecolor=black,
    filecolor=black,
    linkcolor=black,
    urlcolor=black
}

% gfx
\usepackage{wrapfig}
\usepackage{xcolor}

% tables
\usepackage{tabularx}
\usepackage{booktabs}
\usepackage{multicol}
\usepackage{multirow}
\usepackage{makecell}
\usepackage{color, colortbl}

% math
\usepackage{amsmath}
\usepackage{amssymb}
\usepackage{dsfont}
\let\olddiv\div
\usepackage[arrowdel]{physics}
\usepackage{mathtools}

% indexes, links, page format
\usepackage{scrlayer-scrpage}

% misc
\usepackage[super]{nth}
\usepackage[
	output-decimal-marker={.},
	input-symbols = {()},  			% do not treat "(" and ")" in any special way
	group-digits  = true  			% guess what.
]{siunitx}
\usepackage{minted}

%==============================================================================%
% GLOBAL MACROS
%

% Document properties
\newcommand{\myName}{Stefan Hartinger\xspace}
\newcommand{\myTitle}{Master Thesis Notes: Dealing with more than two modes\xspace}

\addtokomafont{labelinglabel}{\sffamily}

% Text abbreviations
\newcommand*{\ie}{i.\,e.\xspace}
\newcommand*{\eg}{e.\,g.\xspace}

% Misc Symbols
\newcommand*{\thus}{\ensuremath{\rightarrow}\xspace}
\newcommand*{\Thus}{\ensuremath{\Rightarrow}\xspace}

% Tables
\newcommand*{\tabcrlf}{\\ \hline}			% actually still allows for optional argument

% Math
\newcommand*{\numberthis}{\addtocounter{equation}{1}\tag{\theequation}}

\newcommand*{\smallfrac}  [2]{\ensuremath{{}^        {#1} \!/_        {#2}}}
\newcommand*{\smallfracrm}[2]{\ensuremath{{}^{\mathrm{#1}}\!/_{\mathrm{#2}}}}

\newcommand*{\transp}{\ensuremath{^\intercal}}

\newcommand*{\iunit}{\ensuremath{\mathrm{i}}}

\newcommand*{\setNaturals} {\ensuremath{\mathds{N}}}
\newcommand*{\setIntegers} {\ensuremath{\mathds{Z}}}
\newcommand*{\setReals}    {\ensuremath{\mathds{R}}}
\newcommand*{\setRationals}{\ensuremath{\mathds{Q}}}
\newcommand*{\setComplex}  {\ensuremath{\mathds{C}}}

\newcommand*{\Lag}{\ensuremath{\mathcal{L}}\xspace}
\newcommand*{\Ham}{\ensuremath{\mathcal{H}}\xspace}

%\newcommand*{\Poisson}[2]{\ensuremath{\left\{ {#1}, {#2} \right\}}}
% physics has \pb which is poisson bracket
% also use alias acom: anticommutator, which is exactly the same.

\newcommand*{\equalCond}{  \mathop{=}\limits^!  }

\DeclareMathOperator{\arsinh}{arsinh}
\DeclareMathOperator{\diag}{diag}

\newcommand*{\DD}[1]{\ensuremath{\text{D}\vec{#1}\;}}
\newcommand*{\set}[1]{\ensuremath{\{#1\}}}

%==============================================================================%
% GLOBAL PARAMTERS
%

\title{\myTitle}
\author{\myName}
\date{\today}

% header, footer
\clearpairofpagestyles
	\cfoot
		[\pagemark]
		{\pagemark}
	\ohead
		[\myTitle, \myName]
		{\myTitle, \myName}
\pagestyle{scrheadings}

%==============================================================================%
% THE REAL STUFF
%	
\begin{document}
\tableofcontents
\newpage

\chapter{Phase Retrieval as Eigenvalue Problem}
In Max paper, the Fourier link between incoming and outgoing phases was derived (cf. eqns. 26 and 27 in his work):
\begin{align}
	\sqrt{m_p} \exp(-\iunit \theta_p) &= \sum_q \sqrt{n_q} u_{pq} \exp(-\iunit \chi  _q) \\
	\sqrt{n_q} \exp(-\iunit \chi  _q) &= \sum_p \sqrt{m_p} u_{pq} \exp(+\iunit \theta_p)
\end{align}

Using definitions:
\begin{align}
	\mathbb{M} &= \sqrt{\diag \vec{m}} \\
	\mathbb{N} &= \sqrt{\diag \vec{n}}
\end{align}
this can be stated as:
\begin{align}
	\vec{\nu}
&=
	\underbrace{
		\mathbb{N}^{-1} \mathbb{M} \mathbb{U}
	}_{\coloneqq \mathbb{C}}
	\vec{\mu}
\end{align}
in which:
\begin{align}
	\mu_p = \exp(+\iunit \theta_p) \\
	\nu_q = \exp(-\iunit \chi  _q)
\end{align}

If there exists an invertible matrix $\mathbb{Q} = \sqrt{\mathbb{C}}$, \ie $\mathbb{Q}^2 = \mathbb{C}$, then:

\begin{align}
	\vec{\nu}
&=
	\mathbb{Q}^2 \vec{\mu}
\\
	\mathbb{Q}^{-1} \vec{\nu}
&=
	\mathbb{Q} \vec{\mu}
\end{align}

---


Using the fact that each component of $\vec{\mu}, \vec{\nu}$ is a phase, we can multiply with its complex conjugate:

\begin{align}
	N
&=
	\vec{\nu}^* \cdot \vec{\nu} \\
&=
	(\mathbb{C} \vec{\mu})^\dagger (\mathbb{C} \vec{\mu}) \\
&=
	\vec{\mu}^\dagger \mathbb{U}^\dagger \mathbb{M} \mathbb{N}^{-1}
	\mathbb{N}^{-1} \mathbb{M} \mathbb{U} \vec{\mu}	
\end{align}

If for symmetric matrices $\mathbb{A}, \mathbb{B}$, $\mathbb{AB}$ is symmetric, then so is $\mathbb{BA}$. (\url{https://math.stackexchange.com/questions/1697991/do-diagonal-matrices-always-commute}). $\mathbb{M}, \mathbb{N}$ are diagonal, and so are their products. Hence:

\begin{align}
	N
&=
	\vec{\mu}^\dagger
	\mathbb{U}^\dagger 
	\qty(\mathbb{M} \mathbb{N}^{-1})^{2}
	\mathbb{U} 
	\vec{\mu}	
\end{align}

(... this ends in $N=N$ ...)

\begin{align}
	N
&=
	(\mathbb{C} \vec{\mu})^\dagger (\mathbb{C} \vec{\mu}) \\
&=
	(\mathbb{C} \vec{\mu})^\dagger  \vec{\nu} \\
&=
	\vec{\mu}^\dagger \mathbb{U}^\dagger \mathbb{M} \mathbb{N}^{-1} \vec{\nu}
\end{align}

%À priori, this only works where $\forall j,k: m_j, n_k \neq 0$, but previous works suggest this should be possible for generic occupation numbers. For development purposes, this allows a very compact method of creating mock data to be fed into the SPA, as Eigenvalue problems can be solved in $\mathcal{O}(K^3)$ where $\mathbb{C} \in \setComplex^{K \times K}$. There's plenty of routines that can do the job, such as Eigen3 (freeware).
\end{document}
