\documentclass[
	english,
	a4paper,
	fontsize=10pt,
	parskip=half,
	titlepage=true,
	DIV=12,
	final
]{scrreprt}


%==============================================================================%
% PACKAGES
%
% Standard text formatting
\usepackage[utf8]{inputenc}
\usepackage{babel}
\usepackage[T1]{fontenc}
\usepackage{lmodern}
\usepackage{microtype}
\usepackage{ragged2e}

\usepackage{csquotes}
\usepackage{xspace}

\usepackage{placeins}	% FloatBarrier.
\usepackage{url}
\usepackage[bf, format=plain]{caption}

\usepackage{hyperref}
\hypersetup{
    colorlinks,
    citecolor=black,
    filecolor=black,
    linkcolor=black,
    urlcolor=black
}

% gfx
\usepackage{wrapfig}

% tables
\usepackage{tabularx}
\usepackage{booktabs}
\usepackage{multicol}
\usepackage{multirow}
\usepackage{makecell}
\usepackage{color, colortbl}

% math
\usepackage{amsmath}
\usepackage{amssymb}
\usepackage{dsfont}
\let\olddiv\div
\usepackage[arrowdel]{physics}
\usepackage{mathtools}

% indexes, links, page format
\usepackage{scrlayer-scrpage}

% misc
\usepackage[super]{nth}
\usepackage[
	output-decimal-marker={.},
	input-symbols = {()},  			% do not treat "(" and ")" in any special way
	group-digits  = true  			% guess what.
]{siunitx}
\usepackage{minted}

%==============================================================================%
% GLOBAL MACROS
%

% Document properties
\newcommand{\myName}{Stefan Hartinger\xspace}
\newcommand{\myTitle}{Master Thesis Notes: Deriving a Zero Occupancy Formula\xspace}

\addtokomafont{labelinglabel}{\sffamily}

% Text abbreviations
\newcommand*{\ie}{i.\,e.\xspace}
\newcommand*{\eg}{e.\,g.\xspace}

% Misc Symbols
\newcommand*{\thus}{\ensuremath{\rightarrow}\xspace}
\newcommand*{\Thus}{\ensuremath{\Rightarrow}\xspace}

% Tables
\newcommand*{\tabcrlf}{\\ \hline}			% actually still allows for optional argument

% Math
\newcommand*{\numberthis}{\addtocounter{equation}{1}\tag{\theequation}}

\newcommand*{\smallfrac}  [2]{\ensuremath{{}^        {#1} \!/_        {#2}}}
\newcommand*{\smallfracrm}[2]{\ensuremath{{}^{\mathrm{#1}}\!/_{\mathrm{#2}}}}

\newcommand*{\transp}{\ensuremath{^\intercal}}

\newcommand*{\iunit}{\ensuremath{\mathrm{i}}}

\newcommand*{\setNaturals} {\ensuremath{\mathbb{N}}}
\newcommand*{\setIntegers} {\ensuremath{\mathbb{Z}}}
\newcommand*{\setReals}    {\ensuremath{\mathbb{R}}}
\newcommand*{\setRationals}{\ensuremath{\mathbb{Q}}}
\newcommand*{\setComplex}  {\ensuremath{\mathbb{C}}}

\newcommand*{\Lag}{\ensuremath{\mathcal{L}}\xspace}
\newcommand*{\Ham}{\ensuremath{\mathcal{H}}\xspace}

%\newcommand*{\Poisson}[2]{\ensuremath{\left\{ {#1}, {#2} \right\}}}
% physics has \pb which is poisson bracket
% also use alias acom: anticommutator, which is exactly the same.

\newcommand*{\equalCond}{  \mathop{=}\limits^!  }

\DeclareMathOperator{\arsinh}{arsinh}
\DeclareMathOperator{\diag}{diag}

%==============================================================================%
% GLOBAL PARAMTERS
%

\title{\myTitle}
\author{\myName}
\date{\today}

% header, footer
\clearpairofpagestyles
	\cfoot
		[\pagemark]
		{\pagemark}
	\ohead
		[\myTitle, \myName]
		{\myTitle, \myName}
\pagestyle{scrheadings}

%==============================================================================%
% THE REAL STUFF
%	
\begin{document}
\tableofcontents
\newpage

\chapter{Bunched Incoming Zero-Occupancy Treatment}
\section{Assumptions, Terminology}
For starters, we will assume that only $m_K = 0$. This shall be denoted as \emph{single zero occupancy} or \emph{SZO}.

From there on, the \emph{bunched zero occupancy} (\emph{BZO}) can be treated:
\begin{align}
	\exists b^{*} : \forall j > b^{*} : m_j = 0 \\
	\qquad \forall j \leq b^* : m_j > 0
\end{align}
This means, the \emph{incoming modes} $j = 1 \ldots b^{*}$ are occupied at least once while all higher modes are not occupied. The case $b^* = K$ corresponds to all modes occupied.

In both, SZO and BZO, we will further assume that the outgoing modes are still all occupied:
\begin{align}
	\forall j : n_j > 0
\end{align}

So, SZO translates to $b^* = K - 1$.

From there on, the generic case follows:
\begin{itemize}
\item The scenario is assumed to be symmetric in $m, n$ -- verify!
\item non-bunched configurations can be obtained by re-ordering coordinates -- verify!
\end{itemize}

\subsection{Notes on L'Hopital}
\begin{itemize}
\item It should be possible to obtain BZO solutions from SZO. Regard only one degree of freedom, treat the other ZO variables as fixed parameters.
\item After obtaining a const factor pertaining the terms depending on the SZO-zero, proceed with the next DoF.
\end{itemize}


\section{Starting Point}
Statement:
\begin{align}
	A_F(\vec{m}, \vec{n})
&=
	\eval{\qty(
			\prod_{j}
			\frac
				{1}
				{\sqrt{m_j! \; n_j!}}
			\pdv[m_j]{x_j}
			\pdv[n_j]{y_j}
		) \exp(\vec{x} \, \mathbb{U} \, \vec{y})
	}_{\vec{x} = \vec{y} = \vec{0}}
\label{eqn:BSA_raw}
\end{align}

Notes:
\begin{itemize}
\item was given
\end{itemize}

Split into BZO form:
\begin{align}
	A_F(\vec{m}, \vec{n})
&=
	\eval{
		\qty(
			\prod_{j=1}^{b^*}
			\frac
				{1}
				{\sqrt{m_j! \; n_j!}}
			\pdv[m_j]{x_j}
			\pdv[n_j]{y_j}
		)
		\qty(
			\prod_{j=b^* + 1}^{K}
			\frac
				{1}
				{\sqrt{n_j!}}
			\pdv[n_j]{y_j}
		)
		\exp(\vec{x} \, \mathbb{U} \, \vec{y})
	}_{\vec{x} = \vec{y} = \vec{0}}
\label{eqn:BSA_raw_BZO}
\end{align}

Holds because:
\begin{itemize}
\item $0! = 1$
\item $\pdv[0]{z} = 1$
\end{itemize}

\section{BSA in integral form}
Equation:
\begin{align}
	A_F(\vec{m}, \vec{n})
&=
	\qty(
		\prod_{j=1}^{b^*}
			\frac
				{\sqrt{m_j! \; n_j!}}
				{-4\pi^{2}}
			\oint_{\gamma}
				\frac
					{\dd{x_j} \dd{y_j}}
					{x_j^{m_j+1}  y_j^{n_j+1}}
	)
	\qty(
		\prod_{j=b^* + 1}^{K}
			\frac
				{\sqrt{n_j !}}
				{2\pi\iunit}
			\oint_{\gamma}
				\frac
					{\dd{y_j}}
					{y_j^{n_j + 1}}
	)
	\exp( \Big. \vec{x} \, \mathbb{U} \, \vec{y}) \\
&=
	\qty(
		\frac{1}{2\pi\iunit}
	)^{K + b^*}
		\prod_{j=1}^{b^*}
			\sqrt{m_j! \; n_j!}
			\oint_{\gamma}
				\frac
					{\dd{x_j} \dd{y_j}}
					{x_j^{m_j+1}  y_j^{n_j+1}}
		\prod_{j=b^* + 1}^{K}
			\sqrt{n_j !}
			\oint_{\gamma}
				\frac
					{\dd{y_j}}
					{y_j^{n_j + 1}}
	\exp( \Big. \vec{x} \, \mathbb{U} \, \vec{y}) \\
&=
	\qty(
		\frac{1}{2\pi\iunit}
	)^{K + b^*}
	\prod_{k=1}^{K}
	\prod_{j=1}^{b^*}
		\sqrt{m_j! \; n_k!}
		\oint_{\gamma}
			\frac
				{\dd{x_j} \dd{y_k}}
				{x_j^{m_j+1}  y_k^{n_k+1}}
	\exp( \Big. \vec{x} \, \mathbb{U} \, \vec{y})
	\label{eqn:BSA_Int_BZO}
\end{align}

\section{Change of Variables}
Using polar coordinates:
\begin{align}
	\begin{cases}
	x_j &= \sqrt{m_j} \exp( \iunit \theta_j) \\
	y_j &= \sqrt{n_j} \exp(-\iunit \chi  _j)
	\end{cases}
	\label{eqn:DefXY}
\end{align}

in eqn. (\ref{eqn:BSA_Int_BZO}) gives:
\begin{multline}
	A_F(\vec{m}, \vec{n})
=
	\qty(
		\frac{1}{2\pi\iunit}
	)^{K + b^*}
	\prod_{k=1}^{K}
	\prod_{j=1}^{b^*}
		\sqrt{m_j! \; n_k!}
		\oint_{\gamma}
			\frac
				{ (\iunit \sqrt{m_j}) \; (\iunit \sqrt{n_k}) \; \dd{\theta_j} \dd{\chi_k}}
				{\sqrt{m_j}^{m_j+1}  \sqrt{n_j}^{n_k+1}}
			\exp( \Big. \iunit(\theta_j - \chi_k))
\\
	\exp(
		\Big.
		-\iunit (m_j + 1) \theta_j
		+\iunit (n_k + 1) \chi  _k
	)
	\exp( \Big. \vec{x} \, \mathbb{U} \, \vec{y})
\end{multline}
\begin{multline}
	\qquad\qquad
=
	\qty(
		\frac{\iunit}{2\pi\iunit}
	)^{K + b^*}
	\prod_{k=1}^{K}
	\prod_{j=1}^{b^*}
		\sqrt{m_j! \; n_k!}
		\oint_{\gamma}
			\frac
				{\dd{\theta_j} \dd{\chi_k}}
				{\sqrt{m_j}^{m_j}  \sqrt{n_j}^{n_k}}
\\
	\exp(
		\Big.
		-\iunit m_j \theta_j
		+\iunit n_k \chi  _k
	)
	\exp( \Big. \vec{x} \, \mathbb{U} \, \vec{y})
\end{multline}
\begin{multline}
	\qquad\qquad
=
	\qty(
		\frac{\iunit}{2\pi\iunit}
	)^{K + b^*}
	\prod_{k=1}^{K}
	\prod_{j=1}^{b^*}
		\sqrt{\frac
			{m_j! \; n_k!}
			{m_j^{m_j}  n_j^{n_k}}
		}
		\oint_{\gamma}
			\dd{\theta_j} \dd{\chi_k}
	\exp(
		\Big.
		-\iunit m_j \theta_j
		+\iunit n_k \chi  _k
	)
	\exp( \Big. \vec{x} \, \mathbb{U} \, \vec{y})
\end{multline}

using
\begin{align}
	C
&=
	\qty(
		\frac{\iunit}{2\pi\iunit}
	)^{K + b^*}
	\prod_{k=1}^{K}
	\prod_{j=1}^{b^*}
		\sqrt{\frac
			{m_j! \; n_k!}
			{m_j^{m_j}  n_j^{n_k}}
		}
\end{align}
({\color{red}keep in mind that this definition of $C$ differs from the fully-occupied scenario!})
gives
\begin{align}
	A_F(\vec{m}, \vec{n})
&=
	C
	\prod_{k=1}^{K}
	\prod_{j=1}^{b^*}
		\oint_{\gamma}
			\dd{\theta_j} \dd{\chi_k}
	\exp(
		\Big.
		-\iunit m_j \theta_j
		+\iunit n_k \chi  _k
	)
	\exp( \Big. \vec{x} \, \mathbb{U} \, \vec{y}) \\
&=
	C
	\prod_{k=1}^{K}
	\prod_{j=1}^{b^*}
		\oint_{\gamma}
			\dd{\theta_j} \dd{\chi_k}
	\exp(
		\Big.
		-\iunit m_j \theta_j
		+\iunit n_k \chi  _k
	)
	\exp(
		\sum_{k,l=1}^{K}
		\sqrt{m_k, n_l}
		u_{k,l}
		\exp(\Big.
			\iunit(\theta_k - \chi_l)
		)
	)
	\label{eqn:BSA_Int_BZO_Polar}
\end{align}
Note
\begin{itemize}
\item although there are zeros in $\vec{m}$, this could still be written in terms of a dot product
	$\vec{n} \cdot \vec{\chi} - \vec{m} \cdot \vec{\theta}$
	, as contributions from $j > b^*$ simply vanish.
\end{itemize}

\section{Integrating out the $\theta_1$ DoF}
split eqn. (\ref{eqn:BSA_Int_BZO_Polar}) into $\theta_1$ dependent terms and rest. Use temporary variables:
\begin{align}
	\vec{\alpha} &= \vec{\theta} - (\theta_1, \ldots, \theta_1)\transp
	&
	\vec{\theta} &= \vec{\alpha} + (\theta_1, \ldots, \theta_1)\transp
\\
	\vec{\beta } &= \vec{\chi  } - (\theta_1, \ldots, \theta_1)\transp
	&
	\vec{\chi  } &= \vec{\beta } + (\theta_1, \ldots, \theta_1)\transp
\end{align}
and get
\begin{multline}
	A_F(\vec{m}, \vec{n})
=
	C
	\prod_{k=1}^{K}
	\prod_{{\color{blue} j=2}}^{b^*}
		\oint_{\gamma}
			\dd{\alpha_j} \dd{\beta_k}
		\int_{0}^{2\pi} {\color{blue} \dd{\theta_1}}
			\exp(
				\Big.
				n_k \beta _k   -
				m_j \alpha_j
			)
\\
	\exp(
		\iunit \qty(
			\sum_{p=1}^{K}   n_p   -
			\sum_{o=1}^{b^*} m_o
		)
		{\color{blue} \theta_1}
	)
	\exp(
		\sum_{o,p=1}^{K}
		\sqrt{m_o, n_p} \;
		u_{o,p}
		\exp(\Big.
			\iunit(\alpha_o - \beta_p)
		)
	)
\end{multline}

As in Max' paper, get a Kronecker Delta from the $\theta_1$ exponential:
\begin{multline}	
	A_F(\vec{m}, \vec{n})
=
	2\pi \delta_{M,N} C
	\prod_{k=1}^{K}
	\prod_{j=2}^{b^*}
		\oint_{\gamma}
			\dd{\alpha_j} \dd{\beta_k}
\\
	\exp(
		\Big.
		n_k \beta _k   -
		m_j \alpha_j
	)
	\exp(
		\sum_{o,p=1}^{K}
		\sqrt{m_o, n_p} \;
		u_{o,p}
		\exp(\Big.
			\iunit(\alpha_o - \beta_p)
		)
	)
\end{multline}
\end{document}