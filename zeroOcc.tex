\documentclass[
	english,
	a4paper,
	fontsize=10pt,
	parskip=half,
	titlepage=true,
	DIV=12,
	final
]{scrreprt}


%==============================================================================%
% PACKAGES
%
% Standard text formatting
\usepackage[utf8]{inputenc}
\usepackage{babel}
\usepackage[T1]{fontenc}
\usepackage{lmodern}
\usepackage{microtype}
\usepackage{ragged2e}

\usepackage{csquotes}
\usepackage{xspace}

\usepackage{placeins}	% FloatBarrier.
\usepackage{url}
\usepackage[bf, format=plain]{caption}

\usepackage{hyperref}
\hypersetup{
    colorlinks,
    citecolor=black,
    filecolor=black,
    linkcolor=black,
    urlcolor=black
}

% gfx
\usepackage{wrapfig}

% tables
\usepackage{tabularx}
\usepackage{booktabs}
\usepackage{multicol}
\usepackage{multirow}
\usepackage{makecell}
\usepackage{color, colortbl}

% math
\usepackage{amsmath}
\usepackage{amssymb}
\usepackage{dsfont}
\let\olddiv\div
\usepackage[arrowdel]{physics}
\usepackage{mathtools}

% indexes, links, page format
\usepackage{scrlayer-scrpage}

% misc
\usepackage[super]{nth}
\usepackage[
	output-decimal-marker={.},
	input-symbols = {()},  			% do not treat "(" and ")" in any special way
	group-digits  = true  			% guess what.
]{siunitx}
\usepackage{minted}

%==============================================================================%
% GLOBAL MACROS
%

% Document properties
\newcommand{\myName}{Stefan Hartinger\xspace}
\newcommand{\myTitle}{Master Thesis Notes: Deriving a Zero Occupancy Formula\xspace}

\addtokomafont{labelinglabel}{\sffamily}

% Text abbreviations
\newcommand*{\ie}{i.\,e.\xspace}
\newcommand*{\eg}{e.\,g.\xspace}

% Misc Symbols
\newcommand*{\thus}{\ensuremath{\rightarrow}\xspace}
\newcommand*{\Thus}{\ensuremath{\Rightarrow}\xspace}

% Tables
\newcommand*{\tabcrlf}{\\ \hline}			% actually still allows for optional argument

% Math
\newcommand*{\numberthis}{\addtocounter{equation}{1}\tag{\theequation}}

\newcommand*{\smallfrac}  [2]{\ensuremath{{}^        {#1} \!/_        {#2}}}
\newcommand*{\smallfracrm}[2]{\ensuremath{{}^{\mathrm{#1}}\!/_{\mathrm{#2}}}}

\newcommand*{\transp}{\ensuremath{^\intercal}}

\newcommand*{\iunit}{\ensuremath{\mathrm{i}}}

\newcommand*{\setNaturals} {\ensuremath{\mathbb{N}}}
\newcommand*{\setIntegers} {\ensuremath{\mathbb{Z}}}
\newcommand*{\setReals}    {\ensuremath{\mathbb{R}}}
\newcommand*{\setRationals}{\ensuremath{\mathbb{Q}}}
\newcommand*{\setComplex}  {\ensuremath{\mathbb{C}}}

\newcommand*{\Lag}{\ensuremath{\mathcal{L}}\xspace}
\newcommand*{\Ham}{\ensuremath{\mathcal{H}}\xspace}

%\newcommand*{\Poisson}[2]{\ensuremath{\left\{ {#1}, {#2} \right\}}}
% physics has \pb which is poisson bracket
% also use alias acom: anticommutator, which is exactly the same.

\newcommand*{\equalCond}{  \mathop{=}\limits^!  }

\DeclareMathOperator{\arsinh}{arsinh}
\DeclareMathOperator{\diag}{diag}

\newcommand*{\DD}[1]{\ensuremath{\text{D}\vec{#1}\;}}

%==============================================================================%
% GLOBAL PARAMTERS
%

\title{\myTitle}
\author{\myName}
\date{\today}

% header, footer
\clearpairofpagestyles
	\cfoot
		[\pagemark]
		{\pagemark}
	\ohead
		[\myTitle, \myName]
		{\myTitle, \myName}
\pagestyle{scrheadings}

%==============================================================================%
% THE REAL STUFF
%	
\begin{document}
\tableofcontents
\newpage

\chapter{Treatment of Incoming Bunched Zero-Occupancy Scenarios}
\section{Assumptions, Terminology}
Initially, we will assume \emph{bunched zero occupancy} (\emph{BZO}):
\begin{align}
	\exists b^{*} : \forall j > b^{*} : m_j = 0 \\
	\qquad \forall j \leq b^* : m_j > 0
\end{align}
This means, the \emph{incoming modes} $j = 1 \ldots b^{*}$ are occupied at least once while all higher modes are not occupied. The case $b^* = K$ corresponds to all modes occupied.

We will further assume that the outgoing modes are still all occupied:
\begin{align}
	\forall j : n_j > 0
\end{align}

From there on, the generic case follows: The non-bunched configurations can be obtained by re-ordering coordinates.

\section{Starting Point}
Statement:
\begin{align}
	A_F(\vec{m}, \vec{n})
&=
	\eval{\qty(
			\prod_{j=1}^{K}
			\frac
				{1}
				{\sqrt{m_j! \; n_j!}}
			\pdv[m_j]{x_j}
			\pdv[n_j]{y_j}
		) \exp(\vec{x}\transp \, \mathbb{U} \, \vec{y})
	}_{\vec{x} = \vec{y} = \vec{0}}
\label{eqn:BSA_raw}
\end{align}

Notes:
\begin{itemize}
\item was given
\end{itemize}

Split into BZO form:
\begin{align}
	A_F(\vec{m}, \vec{n})
&=
	\eval{
		\qty(
			\prod_{j=1}^{b^*}
			\frac {1} {\sqrt{m_j!}}
			\pdv[m_j]{x_j}
		)
		\qty(
			\prod_{k=1}^{K}
			\frac
				{1}
				{\sqrt{n_k!}}
			\pdv[n_k]{y_k}
		)
		\exp(\vec{x}\transp \, \mathbb{U} \, \vec{y})
	}_{\vec{x} = \vec{y} = \vec{0}}
\label{eqn:BSA_raw_BZO}
\end{align}

Holds because:
\begin{itemize}
\item $0! = 1$
\item $\pdv[0]{z} = 1$
\end{itemize}

\section{BSA in Integral Form}
Equation:
\begin{align}
	A_F(\vec{m}, \vec{n})
&=
	\qty(
		\frac{1}{2\pi\iunit}
	)^{K + b^*}
	\oint_{\gamma}
		\qty( \prod_{j=1}^{b^*}
			\frac
				{\sqrt{m_j!} \dd{x_j}}
				{x_j^{m_j+1}}
		)
		\qty( \prod_{k=1}^{K}
			\frac
				{\sqrt{n_k!} \dd{y_k}}
				{ y_k^{n_k+1} }
		)
		\exp( \Big. \vec{x}\transp \, \mathbb{U} \, \vec{y})
	\label{eqn:BSA_Int_BZO}
\end{align}

\section{Change of Variables}
Using polar coordinates:
\begin{align}
	\begin{cases}
	x_j &= \sqrt{m_j} \exp( \iunit \theta_j) \\
	y_j &= \sqrt{n_j} \exp(-\iunit \chi  _j)
	\end{cases}
	\label{eqn:DefXY}
\end{align}

in eqn. (\ref{eqn:BSA_Int_BZO}) gives:
\begin{multline}
	A_F(\vec{m}, \vec{n})
=
	\qty(
		\frac{1}{2\pi\iunit}
	)^{K + b^*}
	\int_{0}^{2\pi}
		\qty( \prod_{k=1}^{K}
			\sqrt{n_k!}
			\frac
				{ {\color{blue} \iunit} \sqrt{n_k}}
				{ \sqrt{n_k}^{n_k + 1} }
			\dd{\chi_k}
		)
		\qty( \prod_{j=1}^{b^*}
			\sqrt{m_j!}
			\frac
				{ {\color{blue} \iunit} \sqrt{m_j}}
				{ \sqrt{m_j}^{m_j + 1} }
			\dd{\theta_j}
		)
\\
	\exp[ \iunit \qty(
		{\textstyle \sum_{j=1}^{b^*} \theta_j} - 
		{\textstyle \sum_{k=1}^{K}   \chi  _k}
	)]
	\exp[
		\iunit \qty(
			-{\textstyle \sum_{j=1}^{b^*} (m_j + 1) \theta_j}
			+{\textstyle \sum_{k=1}^{K}   (n_k + 1) \chi  _k}
	)]
	\exp( \Big. \vec{x}\transp \, \mathbb{U} \, \vec{y})
\end{multline}
where the imaginary units come from substitution $(\vec{x}, \vec{y}) \to (\vec{\theta}, \vec{\chi})$:
\begin{align}
	\dd{x_j}
&=
	\dv{x_j}{\theta_j} \dd{\theta_j}
=
	\iunit \, \sqrt{m_j} \, \exp(\iunit \theta_j) \dd{\theta_j}
\end{align}
and likewise for $y_k \to \chi_k$.

With this, get:
\begin{multline}
	A_F(\vec{m}, \vec{n})
=
	\qty(
		\frac{{\color{blue}\iunit}}{2\pi\iunit}
	)^{K + b^*}
	\int_{0}^{2\pi}
		\qty( \prod_{k=1}^{K}
			\sqrt{\frac
				{ n_k! }
				{ n_k^{n_k} }
			}\dd{\chi_k}
		)
		\qty( \prod_{j=1}^{b^*}
			\sqrt{\frac
				{ m_j! }
				{ m_j^{m_j} }
			}\dd{\theta_j}
		)
\\
			\exp[
				\iunit \qty(
				-{\textstyle \sum_{j=1}^{b^*} m_j \theta_j}
				+{\textstyle \sum_{k=1}^{K}   n_k \chi  _k}
			)]
			\exp( \Big. \vec{x}\transp \, \mathbb{U} \, \vec{y})
\end{multline}
\begin{multline}
	\qquad\qquad
=
	\qty(
		\frac{1}{2\pi}
	)^{K + b^*}
	\int_{0}^{2\pi}
		\qty( \prod_{k=1}^{K}
			\sqrt{\frac
				{ n_k! }
				{ n_k^{n_k} }
			} \dd{\chi_k}
		)
		\qty( \prod_{j=1}^{b^*}
			\sqrt{\frac
				{ m_j! }
				{ m_j^{m_j} }
			} \dd{\theta_j}
		)
\\
		\exp[
			\iunit \qty(
			-\vec{m} \cdot \vec{\theta}
			+\vec{n} \cdot \vec{\chi}
		)]
		\exp( \Big. \vec{x}\transp \, \mathbb{U} \, \vec{y})
\end{multline}

using
\begin{align}
	C
&=
	\qty( \prod_{k=1}^{K}
		\sqrt{\frac
			{ n_k! }
			{ n_k^{n_k} }
		}
	)
	\qty( \prod_{j=1}^{b^*}
		\sqrt{\frac
			{ m_j! }
			{ m_j^{m_j} }
		}
	)
\label{eqn:DefC}
\end{align}
({\color{red}keep in mind that this definition of $C$ differs from the fully-occupied scenario!})
gives
\begin{align}
	A_F(\vec{m}, \vec{n})
&=
	C
	\int_{0}^{2\pi}
		{\color{blue} \dd{\theta_1}}
		\DD{\theta}
		\DD{\chi  }
			\exp[
				\iunit \qty(
				-\vec{m} \cdot \vec{\theta}
				+\vec{n} \cdot \vec{\chi}
			)]
			\exp( \Big. \vec{x}\transp \, \mathbb{U} \, \vec{y})
	\label{eqn:BSA_Int_BZO_Polar}
\end{align}
where
\begin{align}
	\DD{\theta}
&=
	\prod_{{\color{blue} j=2}}^{b^*} \dd{\theta_j}
&
	\DD{\chi}
&=
	\prod_{k=1}^{K}   \dd{\chi  _k}
\end{align}

Note
\begin{itemize}
\item although there are zeros in $\vec{m}$, this can still be written in terms of a dot product
	$\vec{m} \cdot \vec{\theta} - \vec{n} \cdot \vec{\chi}$
	, as contributions from $j > b^*$ simply vanish.
\item because of this, no special treatment of the matrix multiplication $\vec{x} \, \mathbb{U} \, \vec{y}$ is needed.
\end{itemize}

\section{Integrating out the $\theta_1$ DoF}
split eqn. (\ref{eqn:BSA_Int_BZO_Polar}) into $\theta_1$ dependent terms and rest. Use temporary variables:
\begin{align}
	\vec{\alpha} &= \vec{\theta} - (\theta_1, \ldots, \theta_1)\transp
	&
	\vec{\theta} &= \vec{\alpha} + (\theta_1, \ldots, \theta_1)\transp
\\
	\vec{\beta } &= \vec{\chi  } - (\theta_1, \ldots, \theta_1)\transp
	&
	\vec{\chi  } &= \vec{\beta } + (\theta_1, \ldots, \theta_1)\transp
\end{align}
and get
\begin{multline}
	A_F(\vec{m}, \vec{n})
=
	C
	\int_{0}^{2\pi} {\color{blue} \dd{\theta_1}}
	\int_{0}^{2\pi}
		\qty( \prod_{k=1}               ^{K}   \dd{\beta_k} )
		\qty( \prod_{{\color{blue} j=2}}^{b^*} \dd{\alpha_j} )
			\exp[
				\iunit \qty(
				{\textstyle \sum_{s=1}^{K}   n_s \beta _s}   -
				{\textstyle \sum_{t=1}^{b^*} m_t \alpha_t}
			)]
\\
	\exp[
		\iunit \qty(
			{\textstyle \sum_{s=1}^{K}   n_s}   -
			{\textstyle \sum_{t=1}^{b^*} m_t}
		)
		{\color{blue} \theta_1}
	]
	\underbrace{
	\exp[
		{\textstyle \sum_{s,t=1}^{K} }
			\sqrt{m_s, n_t} \;
			u_{s,t}
			\exp(\Big. \iunit(\alpha_s - \beta_t) )
	]}_{\exp( \vec{x}\transp \, \mathbb{U} \, \vec{y})}
\end{multline}

As in Max' paper, get a Kronecker Delta from the $\theta_1$ exponential:
\begin{multline}	
	A_F(\vec{m}, \vec{n})
=
	2\pi \delta_{M,N} C
	\int_{0}^{2\pi}
		\qty( \prod_{k=1}^{K}   \dd{\beta _k} )
		\qty( \prod_{j=2}^{b^*} \dd{\alpha_j} )
	\exp[
		\iunit \qty(
		{\textstyle \sum_{s=1}^{K}   n_s \beta _s}   -
		{\textstyle \sum_{t=1}^{b^*} m_t \alpha_t}
	)]
\\
	\exp[
		{\textstyle \sum_{s,t=1}^{K} }
			\sqrt{m_s, n_t} \;
			u_{s,t}
			\exp(\Big. \iunit(\alpha_s - \beta_t) )
	]
\end{multline}
and resubstitute $(\vec{\alpha}, \vec{\beta}) \to (\vec{\theta}, \vec{\chi})$:
\begin{multline}	
	A_F(\vec{m}, \vec{n})
=
	2\pi \; \delta_{M,N} \; C
	\int_{0}^{2\pi}
		\DD{\theta} \DD{\chi}
	\exp[
		\iunit \qty(
		{\textstyle \sum_{s=1}^{K}   n_s \chi  _s}   -
		{\textstyle \sum_{t=1}^{b^*} m_t \theta_t}
	)]
\\
	\exp[
		{\textstyle \sum_{s,t=1}^{K} }
			\sqrt{m_s, n_t} \;
			u_{s,t}
			\exp(\Big. \iunit(\theta_s - \chi_t) )
	]
\end{multline}
with arbitrary def: $\theta_1 = 0$

\section{Applying SPA}
\subsection{Definitions}
Using
\begin{align}
	\lambda
&=
	\qty( \prod_{k=1}^{K}   n_k )
	\qty( \prod_{j=1}^{b^*} m_j )
\label{eqn:DefLambda}
\\
	f(
		\underbrace{ \vec{\theta}, \vec{\chi} }_{\vec{z}}
	)
&=
	\iunit
	\lambda^{-1}
	\qty[
		\qty(
			\sum_{j=1}^{b^*}
				m_j \theta_j   -
			\sum_{k=1}^{K}
				n_k \chi  _k
		)
	-
		\qty(
			\sum_{s,t=1}^{K}
			\sqrt{m_s, n_t} \;
			u_{s,t}
			\exp(\Big.
				\iunit(\theta_s - \chi_t)
			)
		)
	]
\\
	\oint_{\gamma} \dd[n]{\vec{z}}
		\exp[-\lambda f(\vec{z})]
&=
	\qty( \frac
		{2\pi}
		{\lambda}
	)^{\frac{n}{2}}
	\sum_{r}
		\frac
		{\exp(-\lambda f(\vec{z}_r))}
		{\sqrt{\det S_r}}
\label{eqn:Def_SPA}
\end{align}
where $\vec{z}_r$ are the local extrema of $f$: $\eval{\dv{\vec{z}}f}_{\vec{z} = \vec{z}_r} = 0$\\
and $S_r$ is the Hessian of $f$ evaluated in $\vec{z}_r$\\
{\color{red} note that $\lambda$ differs from the fully occupied form.}

\subsection{Saddle Points}
\label{sec:saddles}
Require:
\begin{align}
	\pdv{f}{\theta_p} &\equalCond 0
&
	\pdv{f}{\chi  _q} &\equalCond 0 
\end{align}
with $p \in \{2, \ldots, b^*\}$ and $q \in \{1, \ldots, K\}$

Find:
\begin{gather}
	\pdv{f}{\theta_p}
=
	\iunit \lambda^{-1}
	\qty[
		m_p
		-
		\sum_{l=1}^{K}
			\sqrt{m_p n_l} \; u_{p,l} \; \exp[ \Big. \iunit(\theta_p - \chi_l)]
	]
	\label{eqn:Jacobiantheta}
\equalCond
	0 \\
\Thus
	m_p
\equalCond
	\sum_{l=1}^{K}
		\sqrt{m_p n_l} \; u_{p,l} \; \exp[ \Big. \iunit(\theta_p - \chi_l) ] \\
\Thus
	\sqrt{m_p} \exp(-\iunit \theta_p)
\equalCond
	\sum_{l=1}^{K} \sqrt{n_l} \; u_{p,l} \; \exp(-\iunit\chi_l)
	\label{eqn:FourierLinkForward}
\end{gather}
with $p \in \{2, \ldots, K\}$ where $\theta_1 = 0$. Implicitly also holds for $p=1$ {\color{red}(Does it?)}, but the derivatives only start at $p=2$.

Likewise:
\begin{gather}
	\pdv{f}{\chi_q}
=
	\iunit\lambda^{-1}
	\qty[
		- n_q
		+
		\sum_{k=1}^{b^*}
			\sqrt{m_k n_q} \; u_{k,q} \; \exp[ \Big. \iunit(\theta_k - \chi_q) ]
	]
	\label{eqn:JacobianChi}
\equalCond
	0 \\
\Thus
	\sqrt{n_q} \exp(+\iunit \chi_q)
\equalCond
	\sum_{k=1}^{b^*}
	\sqrt{m_k} \; u_{k,q} \; \exp(+\iunit\theta_k)
	\label{eqn:FourierLinkBackward}
\end{gather}
with $q \in \{1, \ldots, K\}$ where again $\theta_1 = 0$. Summation may also run up to $K$, since higher contributions are weighted with $\sqrt{m_j} = 0$.

Compactly:
\begin{align}
	\vec{x}^{*} &= \mathbb{U} \vec{y}
&
	\vec{y}^{*} &= \mathbb{U}\transp \vec{x}
	\label{eqn:MatrixCondition}
\end{align}

\subsection{Main Results up to here}
\begin{itemize}
\item The phase relation still holds for BZO scenarios!
\item normalization constants need to be adjusted:
	\begin{itemize}
	\item $C =
	\qty( \prod_{k=1}^{K}
		\frac
			{ \sqrt{n_k!} }
			{ \sqrt{n_k}^{n_k} }
	)
	\qty( \prod_{j=1}^{b^*}
		\frac
			{ \sqrt{m_j!} }
			{ \sqrt{m_j}^{m_j} }
	)$
	\item $\lambda =
	\qty( \prod_{k=1}^{K}   n_k )
	\qty( \prod_{j=1}^{b^*} m_j )$
	\end{itemize}
\item The formulae up to here should not be applied to ZO modes, albeit behaviour for such $m_p$
	implies that they are even valid there. I.\,e. do not attempt to use eqn.
	(\ref{eqn:FourierLinkForward}) for $p > b^*$, or double-check if it is valid use if you do.
\item strictly speaking, the matrix sum over $k$ should be adjusted such that no $\theta_k$ with
	$k > b^*$ are referenced. However, all instances where this happens are weighted with a
	$\sqrt{m_k} = 0$.
\end{itemize}

\section{Hessian}
Recall: We start from
\begin{align}
	S_r
=
	\eval{
		\pdv{f(\vec{z})}%
			{z_\alpha}{z_\beta}
	}_{\vec{z} = \vec{z}_r}
\end{align}
but only true DoFs are relevant. Hence, regard $m_j, n_j$ and $\theta_1$ as const. \\
\Thus $(K + b^* - 1)^{2}$ derivatives wrt. $\theta_2, \ldots \theta_{b^*}, \chi_1, \ldots, \chi_K$

Compactly: Use $\vec{z}_j = \vec{\theta} \oplus \vec{\chi}$, \ie the entire Fourier pair. Re-use eqns (\ref{eqn:Jacobiantheta}) and (\ref{eqn:JacobianChi}) from \ref{sec:saddles} by introducing (remembering the constraint $\theta_1 = 0$):

\begin{align}
	\mathbb{J} = \qty( \grad_{\vec{z}_j} f )
\end{align}
\begin{align}
	J_{\theta, p} 
&= 
	\pdv{f}{\theta_p}
=
	\iunit \lambda^{-1}
	\qty[
		+m_p
		-
		\sum_{l=1}^{K}
			\sqrt{m_p n_l} \; u_{p,l} \; \exp[\Big. \iunit(\theta_p - \chi_l)]
	]
\qfor{} p = 2, \ldots, b^*
\\
	J_{\chi, q} 
&= 
	\pdv{f}{\chi_q}
=
	\iunit \lambda^{-1}
	\qty[
		-n_q
		+
		\sum_{k=1}^{K}
			\sqrt{m_k n_q} \; u_{k,q} \; \exp[\Big. \iunit(\theta_k - \chi_q)]
	]
\qfor{} q = 1, \ldots, K
\end{align}
Understand $\mathbb{J}$ as a \emph{column vector}. In spite of it having two indices here, there is only one dimension. $\theta, \chi$ refer to subblocks in the vector rather than degrees of freedom.

Total dimension of $\mathbb{J}$ is $(K + b^* - 1) \times 1$ due to constraint $\theta_1 = 0$

Get Hessian matrix elements:
\begin{align}
	\qty( S_j )
&=
	\begin{pmatrix}
		\pdv{J_{\theta}}{\vec{\theta}} &
		\pdv{J_{\theta}}{\vec{\chi  }} 
		\\
		\pdv{J_{\chi  }}{\vec{\theta}} &
		\pdv{J_{\chi  }}{\vec{\chi  }} 
	\end{pmatrix}
\end{align}
\begin{align}
	\pdv{J_{\theta, p}}{\theta_{\alpha}}
&=
	+
	\lambda^{-1}
	\delta_{p,\alpha}
	\sum_{l=1}^{K}
		\sqrt{m_p n_l} \; u_{p,l} \; \exp[\Big. \iunit(\theta_p - \chi_l)]
&\text{for } \alpha = 2, \ldots, b^*
\\
	\pdv{J_{\theta, p}}{\chi_{\alpha}}
&=
	-
	\lambda^{-1}
	\sqrt{m_p n_\alpha} \; u_{p,\alpha} \; \exp[\Big. \iunit(\theta_p - \chi_\alpha)]
&\text{for } \alpha = 1, \ldots, K
\\
	\pdv{J_{\chi, q}}{\theta_{\alpha}}
&=
	-
	\lambda^{-1}
	\sqrt{m_\alpha n_q} \; u_{\alpha,q} \; \exp[\Big. \iunit(\theta_\alpha - \chi_q)]
&\text{for } \alpha = 2, \ldots, b^*
\\
	\pdv{J_{\chi, q}}{\chi_{\alpha}}
&=
	+
	\lambda^{-1}
	\delta_{q,\alpha}
	\sum_{k=1}^{K}
		\sqrt{m_k n_q} \; u_{k,q} \; \exp[\Big. \iunit(\theta_k - \chi_q)]
&\text{for } \alpha = 1, \ldots, K
\end{align}
Dimensions:
\begin{align}
	\dim S_j
&=
	\begin{pmatrix}
		(b^* - 1) \times (b^* - 1)	& (b^* - 1) \times K \\
		K \times (b^* - 1)			& K \times K
	\end{pmatrix}
=
	(K + b^* - 1) \times (K + b^* - 1)
\end{align}

The Kronecker Delta is due to the fact that the derived-for variable is present only once at the stage of Jacobian $\mathbb{J}$.

Using the eqns (\ref{eqn:FourierLinkForward}) and (\ref{eqn:FourierLinkBackward}), we can further collapse the diagonal elements:
\begin{align}
	\pdv{J_{\theta, p}}{\theta_{\alpha}}
&=
	\lambda^{-1}
	\delta_{p,\alpha}
	m_p
&
	p \in \{2, \ldots, b^*\}
\\
	\pdv{J_{\chi, q}}{\chi_{\alpha}}
&=
	\lambda^{-1}
	\delta_{q,\alpha}
	n_q
&
	q \in \{1, \ldots, K\}
\end{align}

Thus:
\begin{align}
	(S_j)
&=
	\lambda^{-1}
	\diag(\vec{m}', \vec{n})
	+
	\begin{pmatrix}
		\mathds{O}_{b^*-1} & 
		\pdv{J_{\theta, p}}{\chi_{\alpha}}
		\\
		\pdv{J_{\chi, q}}{\theta_{\alpha}} &
		\mathds{O}_{K}
	\end{pmatrix}
=
	\lambda^{-1}
	\begin{pmatrix}
		\diag(\vec{m}') & F 			\\
		F\transp & \diag(\vec{n})
	\end{pmatrix}
\end{align}
where
\begin{align}
	F
&:=
	\lambda \qty( \pdv{J_{\theta}}{\vec{\chi}} ) 
	= \lambda \qty( \pdv{J_{\chi}}{\vec{\theta}} )\transp
	= \qty( -\sqrt{m_p n_q} \; u_{p,q} \; \exp[\Big. \iunit(\theta_p - \chi_q)] )_{
		\substack{p = 2, \ldots, b^* \\ q = 1, \ldots, K}
	}
\\
	\mathds{O}_{D}
&=	
	\left.
	\mqty(
		0		& \ldots		& 0 \\
		\vdots	& \ddots 	& \vdots \\
		0		& \ldots		& 0
	)
	\right\} D \text{ rows}, D \text{ columns}
\\
	\vec{m}'
&=
	(m_2, \ldots, m_{b^*})\transp
\end{align}
The implied symmetry of the off-diagonal blocks is guaranteed by this being a Hessian matrix and can be read from the off-diagonal blocks by comparing the position of indices.

Note:
\begin{itemize}
\item $(S_j)$ is of dimension $(K + b^* - 1)^{2}$.
\item $F$ is of dimension $(b^* - 1) \times K$
\end{itemize}

\section{Determinant of the Hessian}
\subsection{Determinant-Preserving Expansion}
\subsubsection{Theorem: Determinant-perserving expansion by unity block}
Let there be a matrix $M \in \setReals^{D \times D}$, as well as a matrix $M'$:
\begin{align}
	M'
&=
	\begin{pmatrix}
		\mathds{1}_{E} & \mathds{O}_{E \times D} \\
		\mathds{O}_{D \times E} & M
	\end{pmatrix}
\end{align}
where $\mathds{1}_{E}$ is the unit matrix of dimension $E$ and $\mathds{O}_{E \times D}$ is a null matrix of dimension $E \times D$.

Then,
\begin{equation}
	\forall E \in \setNaturals : \det M = \det M'
\end{equation}

\textbf{Proof:}\\
Regard the Matrix $M^{(n)}$:
\begin{align}
	M^{(n)}
&=
	\begin{pmatrix}
		\mathds{1}_{n} & \mathds{O}_{n \times D} \\
		\mathds{O}_{D \times n} & M
	\end{pmatrix}
\end{align}
\ie $M^{(E)} = M'$. The determinant of $M^{(1)}$ can be easily found from the Laplace expansion:

For an $n \times n$ matrix $B = (b_{i,j})$, and an arbitrary, fixed $r = 1, \ldots, n$, the determinant is given by:
\begin{align}
	\det B
&=
	\sum_{j=1}^{n}
		b_{rj} C_{rj}
\end{align}
where $C_{ij}$ is the cofactor to the $i^{\text{th}}$ row and $j^{\text{th}}$ column, \ie the determinant of the submatrix with the indexed row and column removed, times $(-1)^{i+j}$.

So for the given $M^{(1)}$, and choosing $r = 1$, the determinant reads:
\begin{align}
	\det M^{(1)}
&=
	1 \det M^{(0)} + 0 \\
&=
	\det M
\end{align}

Further, it can easily be seen that
\begin{align}
	\det M^{(n)} = \det M^{(n-1)} = \det M
\end{align}
and from this follows the claim.
\begin{flushright}
	\emph{q.e.d.}
\end{flushright}

\subsubsection{Theorem: Determinant-perserving expansion by unity infix}
Let $M$ be as above, with the blocks:
\begin{align}
	M
&=
	\begin{pmatrix}
		A & B \\ C & D
	\end{pmatrix}
\end{align}
and
\begin{align}
	M'
&=
	\begin{pmatrix}
		A & \mathds{O} & B \\
		\mathds{O} & \mathds{1}_{E} & \mathds{O} \\
		C & \mathds{O} & D
	\end{pmatrix}
\end{align}
where $\mathds{O}$ are null matrices of dimensions such that they match the blocks implicitly defined by $A, B, C, D, \mathds{1}_{E}$.

Then
\[ \forall E \in \setNaturals : \det M = \det M' \]

\textbf{Proof:}\\
From the same arguments as in the last theorem, but choosing $r$ such that it denotes the row above $D$.
\begin{flushright}
	\emph{q.e.d.}
\end{flushright}


\subsection{Definitions: $2K \times 2K$ Form}
Let $\tilde{S}$ be the expansion of $S$ to dimension $2K \times 2K$ of the same determinant:
\begin{align}
	\tilde{S}
&=
	\lambda^{-1}
	\begin{pmatrix}
		\mqty{	1 & \mathds{O}_{1 \times (b^* - 1)} \\ 
				\mathds{O}_{(b^* - 1) \times 1} & \diag(\vec{m'})} &
		\mathds{O}_{b^* \times (K - b^*)}	& 
		\mqty{\mathds{O}_{1 \times K} \\ F}
	\\
		\mathds{O}_{(K - b^*) \times b^*}  &  \mathds{1}_{K - b^*}  &  \mathds{O}_{(K - b^*) \times K}
	\\
		\mqty{\mathds{O}_{K \times 1} & F\transp} & 
		\mathds{O}_{K \times (K - b^*)} & \diag(\vec{n})
	\end{pmatrix}
\end{align}
(overview form without dimensions):
\begin{align}
	\tilde{S}
&=
	\lambda^{-1}
	\begin{pmatrix}
		1			& \mathds{O} 		& \mathds{O}		& \mathds{O} \\
		\mathds{O}	& \diag(\vec{m'})	& \mathds{O}		& F          \\
		\mathds{O}	& \mathds{O}			& \mathds{1}		& \mathds{O} \\
		\mathds{O}	& F\transp			& \mathds{O}		& \diag(\vec{n})
	\end{pmatrix}
\end{align}

where still:
\begin{align}
	F
&:=
	\lambda \qty( \pdv{J_{\theta}}{\vec{\chi}} ) 
	= \lambda \qty( \pdv{J_{\chi}}{\vec{\theta}} )\transp
	= \qty( -\sqrt{m_p n_q} \; u_{p,q} \; \exp[\Big. \iunit(\theta_p - \chi_q)] )_{
		\substack{p = 2, \ldots, b^* \\ q = 1, \ldots, K}
	}
\label{eqn:DefDetOffDiagonalF}
\\
	\vec{m}'
&=
	(m_2, \ldots, m_{b^*})\transp
\end{align}

With this, implicitly using the infix theorem:
\begin{align}
	\det S
&=
	\lambda^{-K - b^* + 1}
	\underbrace{\mqty|
		1			& \mathds{O} 		& \mathds{O}		& \mathds{O} \\
		\mathds{O}	& \diag(\vec{m'})	& \mathds{O}		& F          \\
		\mathds{O}	& \mathds{O}			& \mathds{1}		& \mathds{O} \\
		\mathds{O}	& F\transp			& \mathds{O}		& \diag(\vec{n})
	|}_{= S'}
=
	\lambda^{-K - b^* + 1}
	\mqty|
		\diag(\tilde{\vec{m}})	& \tilde{F}			\\
		\tilde{F}\transp			& \diag(\vec{n})
	|
\label{eqn:detS_protoform}
\end{align}

where:
\begin{itemize}
\item $\tilde{\vec{m}} = (
	1,
	\underbrace{m_2, \ldots, m_{b^*}}_{b^* -1 \text{ items}},
	\underbrace{1, \ldots, 1}_{K - b^* \text{ times}}
	~)\transp$ is of dimension $K$
\item $\tilde{F} = \begin{pmatrix}
		\mathds{O}_{1 \times K} \\ 
		F \\ 
		\mathds{O}_{K - b^* \times K}
	\end{pmatrix}$
	is of dimension $K \times K$
\end{itemize}

\subsection{Alternative Form}
Statement:
\begin{align}
	\det S'
&=
	\mqty|
		\underbrace{\begin{pmatrix}
			\diag(\tilde{\vec{m}})	& \mathds{O}			\\
			\tilde{F}\transp			& \mathds{1}
		\end{pmatrix}}_{2K \times 2K}
		\cdot
		\underbrace{\begin{pmatrix}
			\mathds{1}	&	\qty(\diag(\tilde{\vec{m}}))^{-1} \tilde{F}		\\
			\mathds{O}	&	\diag(\vec{n}) - \tilde{F}\transp \qty(\diag(\tilde{\vec{m}}))^{-1}\tilde{F}
		\end{pmatrix}}_{2K \times 2K}
	|
	\label{eqn:detSProto}
\end{align}
Holds because:
\begin{itemize}
\item Multiplication yields identity
\end{itemize}

\subsection{Partial Evaluation of the Determinant}
Statement
\begin{align}
	\det S'
&=
	\underbrace{
		\qty( \prod_{k=1}^{K}   n_k )
		\qty( \prod_{j=2}^{b^*} m_j )
	}_{= \mathcal{N} }
	\det[
		\mathds{1} -
		\tilde{F}\transp \cdot \qty( \diag(\tilde{\vec{m}}) )^{-1}
		\cdot
		\tilde{F} \cdot \qty( \diag(\vec{n}) )^{-1}
	]
\label{eqn:detSPrimePartial}
\end{align}
Note: $\lambda \neq \mathcal{N}$, but $\lambda = m_1 \mathcal{N}$ {\color{red} and $\lambda$ is still different from the fully occupied form}.

Keep definition:
\begin{align}
	\mathcal{N}
&=
	\qty( \prod_{k=1}^{K}   n_k )
	\qty( \prod_{j=2}^{b^*} m_j )
\label{eqn:DefN}
\end{align}

Holds because
\begin{itemize}
\item directly taken from rederivation.
\end{itemize}

\subsection{Reforming $\tilde{F}$}
Statement:
\begin{align}
	\tilde{F}
&=
	\diag\qty[ \sqrt{ \tilde{\vec{m}}} * \exp(+\iunit \vec{\theta}) ]
	\cdot \tilde{\mathbb{U}} \cdot
	\diag\qty[ \sqrt{        \vec{n} } * \exp(-\iunit \vec{\chi  }) ]
\end{align}
(where $*$ denotes component-wise multiplication. Likewise, $\sqrt{.}$ and $\exp(.)$ are understood to act on the components of the vectors individually and yield a vector.)

Where:
\begin{equation}
	\tilde{u}_{i,j} = \begin{cases}
		-u_{i,j}		&\qq*{for } {\color{red} 2 \leq i \leq b^*} \\
		0			&\qotherwise*
	\end{cases}
\end{equation}
{\color{red} This differs from Max' definition of $\tilde{\mathbb{U}}$}.

Note that this introduction of zero rows suppresses the necessity of the condition $\theta_1 = 0$. Likewise, the phases of unoccupied modes ($i > b^*$) are ignored.

Holds because:
\begin{itemize}
\item see definition of F in eqn. (\ref{eqn:DefDetOffDiagonalF})
\end{itemize}

\subsection{Reformed $\tilde{F}$ in the Determinant}
Statement:
\begin{align}
	\det S'
&=
	\mathcal{N}
	\qty( \prod_{j=1}^{K} \exp(-2\iunit \chi_j) )
	\det\qty[\bigg.
		\diag\qty(\exp(\Big.2\iunit\vec{\chi}))
		-
		\tilde{\mathbb{U}}\transp
		\diag\qty(\exp(2\iunit\vec{\theta}))
		\tilde{\mathbb{U}}
	]
\end{align}
where, again, $\exp(.)$ is understood to act on the individual components of a vector.

Holds because:
\begin{itemize}
\item taken from rederivation
\item tested superficially for compatibility with new vector $\tilde{\vec{m}}$
\end{itemize}

\subsection{Complete Det of the Hessian}
Statement:
\begin{align}
	\det S
&=
	\frac{ \lambda^{-K - b^* + 2} }{ m_1 }
	\qty( \prod_{j=1}^{K} \exp(-2\iunit \chi_j) )
	\det\qty[\bigg.
		\diag\qty(\exp(\Big.2\iunit\vec{\chi}))
		-
		\tilde{\mathbb{U}}\transp
		\diag\qty(\exp(2\iunit\vec{\theta}))
		\tilde{\mathbb{U}}
	]
\label{eqn:detComplete}
\end{align}

Holds because:
\begin{itemize}
\item Phase term as found in rederivation
\item magnitude:
	\begin{itemize}
	\item Start from eqn. (\ref{eqn:detS_protoform}) : $\det S = \lambda^{-K - b^* + 1} \det S'$
	\item $
	\frac
		{\mathcal{N}}
		{\lambda^{K + b^* - 1}}
=
	\frac
		{\lambda / m_1}
		{\lambda^{K + b^* - 1}}
=
	\frac{\lambda^{- K - b^* + 2}}{m_1}
$	
	\end{itemize}
\end{itemize}

Note:
\begin{itemize}
\item in generalized versions of this, use only such $\theta_j, \chi_j$ that are properly defined.
\end{itemize}

\section{Complete Amplitude Formula}
Statement:
\begin{multline}
	A_F(\vec{m}, \vec{n})
=
	2\pi \, \delta_{N,M} \, \qty(
		\frac{1}{2\pi\lambda}
	)^{\frac{K + b^*}{2}}
	\exp(N)
	\qty( \prod_{k=1}^{K}
		\sqrt{\frac
			{ n_k! }
			{ n_k^{n_k} }
	})
	\qty( \prod_{j=1}^{b^*}
		\sqrt{\frac
			{ m_j! }
			{ m_j^{m_j} }
	})
\\
	\sum_r
		\frac
		{\exp[
			\qty(- \sum_{k=2}^{b^*} \iunit m_k \theta_k) + 
			\qty(  \sum_{k=1}^{K}   \iunit n_k \chi  _k)
		]}
		{\sqrt{\det S_r}}
\end{multline}


Holds because:
\begin{align}
	A_F(\vec{m}, \vec{n})
&=
	2\pi \delta_{N,M}
	\underbrace{
		\qty(
			\frac{1}{2\pi}
		)^{K + b^*}
		\qty( \prod_{k=1}^{K}
			\sqrt{\frac
				{ n_k! }
				{ n_k^{n_k} }
			}
		)
		\qty( \prod_{j=1}^{b^*}
			\sqrt{\frac
				{ m_j! }
				{ m_j^{m_j} }
			}
		)
	}_{C}
	\qty(
		\frac{2\pi}{\lambda}
	)^{\frac{K + b^*}{2}}
	\sum_r
		\frac
			{\exp[-\lambda f(\vec{z}_r)]}
			{\sqrt{\det S_r}} \\
&=
	2\pi \delta_{N,M} \qty(
		\frac{1}{2\pi\lambda}
	)^{\frac{K + b^*}{2}}
	\qty( \prod_{k=1}^{K}
		\sqrt{\frac
			{ n_k! }
			{ n_k^{n_k} }
		}
	)
	\qty( \prod_{j=1}^{b^*}
		\sqrt{\frac
			{ m_j! }
			{ m_j^{m_j} }
		}
	)
	\sum_r
		\frac
		{\overbrace{
			\exp[-\lambda f(\vec{z}_r)]
		}^{\mathcal{E}}}
		{\sqrt{\det S_r}}
\end{align}
from SPA formula eqn. (\ref{eqn:Def_SPA}) with
\begin{align}
	\mathcal{E}
&=
	\exp[ 
		\qty(- \sum_{k=2}^{b^*} \iunit m_k \theta_k) + 
		\qty(  \sum_{k=1}^{K}   \iunit n_k \chi  _k)   +
		\vec{x}\transp \, \mathbb{U} \, \vec{y}
	]
=
	\exp[ 
		\qty(- \sum_{k=2}^{b^*} \iunit m_k \theta_k) + 
		\qty(  \sum_{k=1}^{K}   \iunit n_k \chi  _k)   +
		N
	]
\end{align}
where $\vec{x}\transp \, \mathbb{U} \, \vec{y} = N$ follows from applying the definition in eqn. (\ref{eqn:DefXY})


\section{Non-BZO scenarios With Zero Occupancy}
The above derivation assumes BZO with the unoccupied modes at the end of the list.
Here I will consider whether it is possible to re-label modes such that the BZO scenario is obtained, and which changes are necessary, if any. The lines in column \emph{Operation} correspond to sections in this document.
\begin{tabularx}
	{\linewidth}
	{p{.30\linewidth}
	 p{.65\linewidth}}
	\textbf{Operation} & \textbf{Comment} \tabcrlf
	
	BSA in Integral Form &
		change definition of $\mathbb{U}, \vec{x}$ according to re-labelling.
		For $\mathbb{U}$ this means swapping rows. \newline
		Otherwise okay.
	\tabcrlf
	Change of Variables &
		change accordance rules for $\vec{x}$, as already implied in the last step. This will affect 
		$\vec{\theta}$. Also update the definition of $C$ and $\DD{\theta}$. \newline
		Otherwise okay.
	\tabcrlf
	Integrating out the $\theta_1$ DoF &
		Update definition of $\vec{\alpha}$. The re-labelling has already selected an occupied mode as
		$j=1$ mode \newline
		Otherwise okay.
	\tabcrlf
	Applying SPA: Definitions &
		Update definition of $\lambda$. \newline
		Otherwise okay.
	\tabcrlf
	Applying SPA: Saddle Points &
		Recall update of $\mathbb{U}$ \newline
		Otherwise okay.
	\tabcrlf
	Hessian &
		Update $\mathbb{J}, \vec{m}', F$ \newline
		Otherwise okay.
	\tabcrlf
	Determinant of the Hessian: Definitions: $2K \times 2K$ Form &
		Update $\tilde{S}, \tilde{\vec{m}}, \tilde{F}$ \newline
		Otherwise okay.
	\tabcrlf
	Determinant of the Hessian: Alternative Form &
		Okay.
	\tabcrlf
	Determinant of the Hessian: Partial Evaluation of the Determinant &
		Update $\mathcal{N}$ \newline
		Otherwise okay.
	\tabcrlf
	Determinant of the Hessian: Reforming $\tilde{F}$ &
		Okay.
	\tabcrlf
	Determinant of the Hessian: Reformed $\tilde{F}$ in the Determinant &
		Okay.
\end{tabularx}

\subsection{Generalized Determinant}
Following above considerations we turn now to the determinant of a non-BZO scenario.

In this, we understand $b^*$ to be the \emph{number of occupied incoming modes}, rather than an upper index. The reordering of $\tilde{\mathbb{U}}, \vec{\theta}$ will implicitly be reversed. That means, the index $j$ represents the $j^{\text{th}}$ incoming mode and $\vec{\theta}$ is given in order of the modes. This is possible since all of the above operations were only reordering the system of equations but did not change any quality thereof.

Instead of integrating out the $\theta_1$ DoF, any mode with nonzero occupancy can be chosen to be the reference mode. This mode shall be denoted $j^*$

With this definition we get:
\begin{align}
	\lambda
&=
	\qty( \prod_{k=1}^{K}   n_k )
	\qty( \prod_{j : m_j \neq 0} m_j )
\label{eqn:DefLambdaGeneric}
\\
	\tilde{u}_{i,j} 
&=
	\begin{cases}
		-u_{i,j}		&\qq*{for } i \neq j^* \land m_i \neq 0 \\
		0			&\qotherwise*
	\end{cases}
\label{eqn:DefReducedDFTGeneric}
\end{align}

And with this the determinant:
\begin{align}
	\det S
&=
	\frac{ \lambda^{-K - b^* + 2} }{ m_{j^*} }
	\qty( \prod_{j=1}^{K} \exp(-2\iunit \chi_j) )
	\det\qty[\bigg.
		\diag\qty(\exp(\Big.2\iunit\vec{\chi}))
		-
		\tilde{\mathbb{U}}\transp
		\diag\qty(\exp(2\iunit\vec{\theta}))
		\tilde{\mathbb{U}}
	]
\label{eqn:detSGeneralized}
\end{align}

\subsection{Generalized Amplitude}
Definition of $b^*, j^*, \lambda$ as above.

\begin{multline}
	A_F(\vec{m}, \vec{n})
=
	2\pi \, \delta_{N,M} \, \qty(
		\frac{1}{2\pi\lambda}
	)^{\frac{K + b^*}{2}}
	\exp(N)
	\qty( \prod_{k=1}^{K}
		\sqrt{\frac
			{ n_k! }
			{ n_k^{n_k} }
	})
	\qty( \prod_{j : m_j \neq 0}
		\sqrt{\frac
			{ m_j! }
			{ m_j^{m_j} }
	})
\\
	\sum_r
		\frac
		{\exp[
			\qty(- \sum_{j\neq j^* \land m_j \neq 0} \iunit m_j \theta_j) + 
			\qty(  \sum_{k=1}^{K}                    \iunit n_k \chi  _k)
		]}
		{\sqrt{\det S_r}}
\end{multline}


\section{Specialization: 2x2}
Using Definitions from eqn. (\ref{eqn:DefLambdaGeneric}) for $\lambda$ and eqn. (\ref{eqn:DefReducedDFTGeneric}) for $\tilde{\mathbb{U}}$.

Recall eqn. (\ref{eqn:DefLambda}), which is equivalent to eqn. (\ref{eqn:DefLambdaGeneric}):
\begin{align*}
	\lambda
&=
	\qty( \prod_{k=1}^{K}   n_k )
	\qty( \prod_{j : m_j \neq 0} m_j )
=
	m_{j^*} \, n_1 \, n_2
\end{align*}

\subsection{Unoccupied $m_1$ mode}
\subsubsection{Determinant}
\begin{align}
	\det S
&=
	\frac{ \lambda^{-2 - 1 + 2} }{ m_2 }
	\qty( \prod_{j=1}^{2} \exp(-2\iunit \chi_j) )
	\det\qty[\bigg.
		\diag\qty(\exp(\Big.2\iunit\vec{\chi}))
		-
		\tilde{\mathbb{U}}\transp
		\diag\qty(\exp(2\iunit\vec{\theta}))
		\tilde{\mathbb{U}}
	]
\end{align}
for this specific case, $\tilde{\mathbb{U}}$ is actually a null matrix (we have to pick $j=2$ as reference). Hence, the 
$\exp(\pm 2\iunit \chi_j)$ terms cancel out:
\begin{align}
	\det S
&=
	\frac{1}{m_2^{2}}
	\frac{1}{n_1 n_2}
\end{align}

\subsubsection{Amplitude}
\begin{align}
	A_F(m_2, n_1, n_2)
&=
	2\pi \, \delta_{N,M} \, \qty(
		\frac{1}{2\pi\lambda}
	)^{\frac{2 + 1}{2}}
	\exp(N)
	\qty( \prod_{k=1}^{2}
		\sqrt{\frac
			{ n_k! }
			{ n_k^{n_k} }
	})
		\sqrt{\frac
			{ m_2! }
			{ m_2^{m_2} }
	}
	\sum_r
		\frac
		{\exp[ \sum_{k=1}^{2} \iunit n_k \chi_k]}
		{\sqrt{\det S_r}} \\
&=
	2\pi \, \delta_{N,M} \, \qty(
		\frac{1}{2\pi\lambda}
	)^{\smallfrac{3}{2}}
	\exp(N)
		\sqrt{\frac
			{ n_1! \; n_2! \; m_2! }
			{ n_1^{n_1} \; n_2^{n_2} \; m_2^{m_2}}
	}
	\sum_r
		\frac
		{\exp[
			\iunit ( n_1 \chi_1 + n_2 \chi_2 )
		]}
		{\sqrt{\det S_r}}
	\\
&=
	2\pi \, \delta_{N,M} \, \qty(
		\frac{1}{2\pi \; m_2 n_1 n_2}
	)^{\smallfrac{3}{2}}
	(n_1 n_2)^{\smallfrac{1}{2}} m_2
	\exp(N)
	\sqrt{\frac
		{ n_1! \; n_2! \; m_2! }
		{ n_1^{n_1} \; n_2^{n_2} \; m_2^{m_2}}
	}
	\sum_r
		\exp[
			\iunit ( n_1 \chi_1 + n_2 \chi_2 )
		] \\
&=
	\frac
		{ \delta_{N,M} \exp(N) }
		{ \sqrt{2\pi m_2} \; n_1 n_2 }
	\sqrt{\frac
		{ m_2! \; n_1! \; n_2! }
		{ m_2^{m_2} n_1^{n_1} n_2^{n_2} }
	}
	\sum_r
		\exp[ \Big.
			\iunit \qty(  
			n_1 \chi_1 + n_2 \chi_2)
		]
\end{align}

\subsection{Unoccupied $m_2$ mode}
These can be found directly from the BZO scenario.

\subsubsection{Determinant}
\begin{align}
	\det S
&=
	\frac{ \lambda^{-2 - 1 + 2} }{ m_1 }
	\qty( \prod_{j=1}^{2} \exp(-2\iunit \chi_j) )
	\det\qty[\bigg.
		\diag\qty(\exp(\Big.2\iunit\vec{\chi}))
		-
		\tilde{\mathbb{U}}\transp
		\diag\qty(\exp(2\iunit\vec{\theta}))
		\tilde{\mathbb{U}}
	] \\
&=
	\frac{1}{n_1 n_2}
	\frac{1}{m_1^{2}}
	\qty( \prod_{j=1}^{2} \exp(-2\iunit \chi_j) )
	\det\qty[\bigg.
		\diag\qty(\exp(\Big.2\iunit\vec{\chi}))
		-
		\tilde{\mathbb{U}}\transp
		\diag\qty(\exp(2\iunit\vec{\theta}))
		\tilde{\mathbb{U}}
	]
\end{align}
for this specific case, $\tilde{\mathbb{U}}$ is actually a null matrix. Hence, the 
$\exp(\pm 2\iunit \chi_j)$ terms cancel out:
\begin{align}
	\det S
&=
	\frac{1}{n_1 n_2}
	\frac{1}{m_1^{2}}
\end{align}

\subsubsection{Amplitude}
\begin{multline}
	A_F(m_1, n_1, n_2)
=
	2\pi \, \delta_{N,M} \, \qty(
		\frac{1}{2\pi \; m_1 n_1 n_2}
	)^{\frac{2 +1}{2}}
	\exp(N)
	\sqrt{\frac
		{ m_1! \; n_1! \; n_2! }
		{ m_1^{m_1} n_1^{n_1} n_2^{n_2} }
	}
\\
	\sum_r
		\frac
		{\exp[ \Big.
			\iunit \qty( -0 + 
			n_1 \chi_1 + n_2 \chi_2)
		]}
		{\sqrt{\det S_r}}
\end{multline}
\begin{align}
\nonumber
&=
	2\pi \, \delta_{N,M} \, \qty(
		\frac{1}{2\pi \; m_1 n_1 n_2}
	)^{\smallfrac{3}{2}}
	\underbrace{
		(n_1 \; n_2)^{\smallfrac{1}{2}} m_1
	}_{\smallfrac{1}{\det S_r}} \;
	\exp(N)
	\\ & \qquad\qquad
	\sqrt{\frac
		{ m_1! \; n_1! \; n_2! }
		{ m_1^{m_1} n_1^{n_1} n_2^{n_2} }
	}
	\sum_r
		\exp[ \Big.
			\iunit \qty(  
			n_1 \chi_1 + n_2 \chi_2)
		]
	\\
&=
	2\pi \, \delta_{N,M} \, \qty(
		\frac{1}{2\pi \; n_1 n_2}
	)
	\sqrt{\frac{1}{2\pi m_1}}
	\exp(N)
	\sqrt{\frac
		{ m_1! \; n_1! \; n_2! }
		{ m_1^{m_1} n_1^{n_1} n_2^{n_2} }
	}
	\sum_r
		\exp[ \Big.
			\iunit \qty(  
			n_1 \chi_1 + n_2 \chi_2)
		]
\\
	A_F(m_1, n_1, n_2)
&=
	\frac
		{ \delta_{N,M} \exp(N) }
		{ \sqrt{2\pi m_1} \; n_1 n_2 }
	\sqrt{\frac
		{ m_1! \; n_1! \; n_2! }
		{ m_1^{m_1} n_1^{n_1} n_2^{n_2} }
	}
	\sum_r
		\exp[ \Big.
			\iunit \qty(  
			n_1 \chi_1 + n_2 \chi_2)
		]
\end{align}


\subsection{Comparision: Fully occupied}
\subsubsection{Determinant}
\begin{multline*}
	\det S
=
	\frac{1}{m_1}
	\qty( \prod_{j=1}^{2} m_j n_j)^{-2}
	\qty( \prod_{j=1}^{2} \exp(-2\iunit \chi_j) )
\\
	\qty[
		\exp( \Big. 2\iunit (\chi_1 + \chi_2) )
		-\frac{1}{2} \qty[ \Big.
			\exp[2\iunit (\theta_2 + \chi_1)] +
			\exp[2\iunit (\theta_2 + \chi_2)]
		]
		-
		\frac{1}{4}\exp( \Big. 4\iunit \theta_2 )
	]
\end{multline*}

\subsubsection{Amplitude}
\begin{multline}
	A_F(\vec{m}, \vec{n})
\approx
	\qty(\frac
		{2\pi}
		{\prod_{j=1}^{K} m_j n_j}
	)^{3/2}
	\qty(
		\prod_{j=1}^{2}
		\frac
			{1}
			{-4\pi^{2}}
		\sqrt{\frac
			{m_j!      \; n_j!}
			{m_j^{m_j} \; n_j^{n_j} }
		}
	)
	2\pi \, \delta_{M,N}
	\sum_{j}
		\qty(\bigg. \mu_j^*)^{m_2}
	\ldots
\\
	\ldots
		\qty(\sqrt{\frac{m_1}{n_1}} u_{1,1}  +  \sqrt{\frac{m_2}{n_1}} u_{2,1} \mu_{j})^{n_1}
		\qty(\sqrt{\frac{m_1}{n_2}} u_{1,2}  +  \sqrt{\frac{m_2}{n_2}} u_{2,2} \mu_{j})^{n_2}
	\frac
		{\exp(N)}
		{\sqrt{\det S_j}}
\end{multline}

\chapter{Treatment of Outgoing Zero-Occupancy Scenarios}
In the previous chapter, it was assumed that one or more \emph{incoming modes} had zero occupancy, while all the \emph{outgoing} modes were occupied at least once.

Following the same scheme of derivation I will attempt here to find analogous formulae for scenarios of the form:
\begin{align}
	\mqty{
						& \forall j : m_j > 0 \\
		\exists b^* :	& \forall j \leq b^* : n_j > 0,\\
						& \forall j   >  b^* : n_j = 0
	}
\end{align}
\ie we start from a BZO case and deduce the generic case thereof.

\section{Validity table}
Following the sections in chapter one, this tells which adaptions to the shown scheme are necessary to find formulae for outgoing zero occupancies.

\begin{tabularx}
	{\linewidth}
	{p{.30\linewidth}
	 p{.65\linewidth}}
	\textbf{Operation} & \textbf{Comment} \tabcrlf
	
	Starting Point &
		Flip roles of $\vec{m}, \vec{n}$ (product boundaries) \newline
		Otherwise Okay.
	\tabcrlf
	BSA in Integral Form &
		Flip roles of $\vec{m}, \vec{n}$ (product boundaries) \newline
		Otherwise okay.
	\tabcrlf
	Change of Variables &
		Flip roles of $\vec{m}, \vec{n}$ (product boundaries) \newline
		Get different $C, \DD{\chi}, \DD{\theta}$.\newline
		Otherwise okay.
	\tabcrlf
	Integrating out the $\theta_1$ DoF &
		Flip roles of $\vec{\alpha}, \vec{\beta}$ \newline
		Otherwise Okay.
	\tabcrlf
	Applying SPA: Definitions &
		Update definition of $\lambda$. \newline
		Otherwise okay.
	\tabcrlf
	Applying SPA: Saddle Points &
		Get new valid index ranges: $p = 2, \ldots, K; q = 1, \ldots, b^*$ \newline
		Get new summation boundary in correspondence. \newline
		Otherwise okay.
	\tabcrlf
	Hessian &
		Changed ranges for Jacobian, Hessian \newline
		Treat $\vec{m}'$ like in Max' paper, but introduce $\vec{n}'$, covering only the nonzero modes
		Otherwise okay.
	\tabcrlf
\end{tabularx}

\begin{tabularx}
	{\linewidth}
	{p{.30\linewidth}
	 p{.65\linewidth}}
	\textbf{Operation} & \textbf{Comment} \tabcrlf
	
	Determinant of the Hessian: Definitions: $2K \times 2K$ Form &
		Use Max' def of $\tilde{\vec{m}}$ together with 
		$\tilde{\vec{n}} = (n_1, \ldots, n_{b^*}, 1, \ldots, 1)\transp$ of lenght $K$. Lacks proof that
		this expansion does preserve determinant, but Alex said that it is okay.\newline
		Update $\tilde{F}$ to include a block of null \emph{columns} (see section below).
		Update $\mathcal{N}$: product upper index for $k$ is $b^*$, upper index for $j$ is $K$.
		Otherwise okay.
	\tabcrlf
	Determinant of the Hessian: Alternative Form &
		Okay.
	\tabcrlf
	Determinant of the Hessian: Partial Evaluation of the Determinant &
		Substitute $\vec{n} \to \tilde{\vec{n}}$. Recall updated $\mathcal{N}$ \newline
		Otherwise okay.
	\tabcrlf
	Determinant of the Hessian: Reforming $\tilde{F}$ &
		Update $\tilde{\mathbb{U}}$:
	\tabcrlf
	Determinant of the Hessian: Reformed $\tilde{F}$ in the Determinant &
		Okay.
\end{tabularx}

\section{Changed Constants and Expressions}
\begin{align}
	C
&=
	\qty( \prod_{k=1}^{b^*}
		\sqrt{\frac
			{ n_k! }
			{ n_k^{n_k} }
		}
	)
	\qty( \prod_{j=1}^{N}
		\sqrt{\frac
			{ m_j! }
			{ m_j^{m_j} }
		}
	)
\label{eqn:DefC_n}
\\
	\lambda
&=
	\qty( \prod_{k=1}^{b^*} n_k )
	\qty( \prod_{j=1}^{K}   m_j )
\label{eqn:DefLambda_n}
\\
	\sqrt{m_p} \exp(-\iunit \theta_p)
&\equalCond
	\sum_{l=1}^{b^*} \sqrt{n_l} \; u_{p,l} \; \exp(-\iunit\chi_l)
\label{eqn:FourierLinkForward_n}
\\
	\sqrt{n_q} \exp(+\iunit \chi_q)
&\equalCond
	\sum_{k=1}^{K}
	\sqrt{m_k} \; u_{k,q} \; \exp(+\iunit\theta_k)
\label{eqn:FourierLinkBackward_}
\\
	\tilde{F}
&=
	\begin{pmatrix}
		0		& \ldots & 0				& 0		& \ldots & 0 \\
		F_{2,1}	& \ldots & F_{2, b^*}	& 0		& \ldots & 0 \\
		\vdots	& \ddots	 & \vdots		& \vdots	& \ddots & \vdots \\
		F_{K,1} & \ldots & F_{K, b^*}	& 0		& \ldots & 0
	\end{pmatrix}
\label{eqn:DefFTilda_n}
\\
	u_{i,j}
&=
	\begin{cases}
		0			& \text{for } i = 1 \\
		0			& \text{for } j > b^* \\
		-u_{i,j}		& \text{otherwise}
	\end{cases}
\label{eqn:DefUTilda_n}
\end{align}

\section{Generalized Formulae}
Following the arguments from the last chapter it is possible to allow \emph{any} outgoing modes to have zero occupancy, \ie to lift the BZO condition

In this section: Using symbols:
\begin{itemize}
\item $K$: Total number of incoming/outgoing modes
\item $b^*$: number of occupied outgoing modes
\item $j^*$: reference mode ID
\end{itemize}

\begin{align}
	\lambda
&=
	\qty( \prod_{k : n_k \neq 0} n_k )
	\qty( \prod_{j = 1}^{K}      m_j )
\label{eqn:DefLambdaGeneric_n}
\end{align}

\subsection{Generalized Determinant}
\begin{align}
	\det S
&=
	\frac{ \lambda^{-K - b^* + 2} }{ m_{j^*} }
	\qty( \prod_{j: n_j \neq 0} \exp(-2\iunit \chi_j) )
	\det\qty[\bigg.
		\diag\qty(\exp(\Big.2\iunit\vec{\chi}))
		-
		\tilde{\mathbb{U}}\transp
		\diag\qty(\exp(2\iunit\vec{\theta}))
		\tilde{\mathbb{U}}
	]
\label{eqn:detSGeneralized_n}
\end{align}
Assume $n_j = 0 \Thus \chi_j = 0$.\\

Note that for $n_j = 0$, this produces contributions of $1$ to the det, \ie this does not change the det. Both matrices are diagonal, and the RHS of the difference has zeros in the columns that correspond to ZOs in $\vec{n}$.

\subsection{Generalized Amplitude}
\begin{multline}
	A_F(\vec{m}, \vec{n})
=
	2\pi \, \delta_{N,M} \, \qty(
		\frac{1}{2\pi\lambda}
	)^{\frac{K + b^*}{2}}
	\exp(N)
	\qty( \prod_{k : n_k \neq 0}
		\sqrt{\frac
			{ n_k! }
			{ n_k^{n_k} }
	})
	\qty( \prod_{j = 1}^{K}
		\sqrt{\frac
			{ m_j! }
			{ m_j^{m_j} }
	})
\\
	\sum_r
		\frac
		{\exp[
			\qty(- \sum_{j\neq j^*}      \iunit m_j \theta_j) + 
			\qty(  \sum_{k : n_k \neq 0} \iunit n_k \chi  _k)
		]}
		{\sqrt{\det S_r}}
\end{multline}

\section{Specialization: 2x2}
\subsection{Unoccupied $n_1$ mode}
\begin{align}
	\lambda
&=
	m_1 \, m_2 \, n_2
\end{align}

\subsubsection{Determinant}
Regard the form of $\tilde{U}$:
\begin{gather}
	\tilde{U}
=
	\begin{pmatrix}
		0 & 0 \\ 0 & -\smallfrac{1}{\sqrt{2}}
	\end{pmatrix}
=
	\tilde{U}\transp
\\
	\Thus
	\tilde{\mathbb{U}}\transp
	\diag\qty(\exp(2\iunit\vec{\theta}))
	\tilde{\mathbb{U}}
=
	\frac{1}{2} \exp(2\iunit \theta_2)
	\begin{pmatrix}
		0 & 0 \\ 0 & 1
	\end{pmatrix}
\\
	\Thus
	\diag\qty( \exp(\Big.2\iunit\vec{\chi}) )
	-
	\tilde{\mathbb{U}}\transp
	\diag\qty(\exp(2\iunit\vec{\theta}))
	\tilde{\mathbb{U}}
=
	\begin{pmatrix}
		1 & 0 \\
		0 & \exp(2\iunit \chi_2) - \frac{1}{2} \exp(2\iunit \theta_2)
	\end{pmatrix}	
\end{gather}

\begin{align}
	\det S
&=
	\frac{ \lambda^{-2 - 1 + 2} }{ m_1 }
	\exp(-2\iunit \chi_2)
	\qty[ 
		\exp(2\iunit \chi_2) - \frac{1}{2} \exp(2\iunit \theta_2)
	] \\
&=
	\frac
		{1 - \frac{1}{2} \exp[\Big. 2\iunit (\theta_2 - \chi_2)]}
		{m_1^2 \, m_2 \, n_2}
\end{align}

\subsubsection{Amplitude}
\begin{align}
	A_F(m_1, m_2, n_2)
&=
	2\pi \, \delta_{N,M} \, \qty(
		\frac{1}{2\pi\lambda}
	)^{\frac{2 + 1}{2}}
	\exp(N)
	\qty(
		\sqrt{\frac
			{ m_1!      \; m_2!     \; n_2! }
			{ m_1^{m_1} \;m_2^{m_2} \; n_2^{n_2} }
	})
	\sum_r
		\frac
		{\exp[
			\qty(- \iunit m_2 \theta_2) + 
			\qty(  \iunit n_2 \chi  _2)
		]}
		{\sqrt{\det S_r}}
	\\
&=
\nonumber
	2\pi \, \delta_{N,M} \, \qty(
		\frac{1}{2\pi m_1 \, m_2 \, n_2}
	)^{\smallfrac{3}{2}}
	\exp(N)
	\sqrt{\frac
		{ m_1!      \; m_2!     \; n_2! }
		{ m_1^{m_1} \;m_2^{m_2} \; n_2^{n_2} }
	}
	\sqrt{m_1^2 \, m_2 \, n_2}
\\ &\qquad
	\sum_r
		\frac
		{\exp[
			\qty(- \iunit m_2 \theta_2) + 
			\qty(  \iunit n_2 \chi  _2)
		]}
		{\sqrt{1 - \frac{1}{2} \exp[\Big. 2\iunit (\theta_2 - \chi_2)]}}
	\\
&=
	\frac
		{\delta_{N,M} \exp(N)}
		{\sqrt{2\pi m_1} m_2 n_2}
	\sqrt{\frac
		{ m_1!      \; m_2!     \; n_2! }
		{ m_1^{m_1} \;m_2^{m_2} \; n_2^{n_2} }
	}
	\sum_r \frac
		{\exp[ \Big. \iunit
			\qty( n_2 \chi_2 - m_2 \theta_2 )
		]}
		{\sqrt{1 - \frac{1}{2} \exp[\Big. 2\iunit (\theta_2 - \chi_2)]}}
\end{align}

\subsection{Unoccupied $n_2$ mode}
\begin{align}
	\lambda
&=
	m_1 \, m_2 \, n_1
\end{align}

\subsubsection{Determinant}
Regard the form of $\tilde{U}$:
\begin{align}
	\tilde{U}
&=
	\begin{pmatrix}
		0 & 0 \\ +\smallfrac{1}{\sqrt{2}} & 0
	\end{pmatrix}
&
	\tilde{U}\transp
&=
	\begin{pmatrix}
		0 & +\smallfrac{1}{\sqrt{2}} \\ 0 & 0
	\end{pmatrix}
\end{align}
\begin{gather}
	\Thus
	\tilde{\mathbb{U}}\transp
	\diag\qty(\exp(2\iunit\vec{\theta}))
	\tilde{\mathbb{U}}
=
	\frac{1}{2} \exp(2\iunit \theta_2)
	\begin{pmatrix}
		1 & 0 \\ 0 & 0
	\end{pmatrix}
\\
	\Thus
	\diag\qty( \exp(\Big.2\iunit\vec{\chi}) )
	-
	\tilde{\mathbb{U}}\transp
	\diag\qty(\exp(2\iunit\vec{\theta}))
	\tilde{\mathbb{U}}
=
	\begin{pmatrix}
		\exp(2\iunit \chi_1) - \frac{1}{2} \exp(2\iunit \theta_2) & 0 \\
		0 & 1
	\end{pmatrix}	
\end{gather}
\begin{align}
	\det S
&=
	\frac{ \lambda^{-2 - 1 + 2} }{ m_1 }
	\exp(-2\iunit \chi_1)
	\qty[ 
		\exp(2\iunit \chi_1) - \frac{1}{2} \exp(2\iunit \theta_2)
	] \\
&=
	\frac
		{1 - \frac{1}{2} \exp[\Big. 2\iunit (\theta_2 - \chi_1)]}
		{m_1^2 \, m_2 \, n_1}
\end{align}

\subsubsection{Amplitude}
\begin{align}
	A_F(m_1, m_2, n_1)
&=
	2\pi \, \delta_{N,M} \, \qty(
		\frac{1}{2\pi\lambda}
	)^{\frac{2 + 1}{2}}
	\exp(N)
	\qty(
		\sqrt{\frac
			{ m_1!      \; m_2!     \; n_1! }
			{ m_1^{m_1} \;m_2^{m_2} \; n_1^{n_1} }
	})
	\sum_r
		\frac
		{\exp[
			\qty(- \iunit m_2 \theta_2) + 
			\qty(  \iunit n_1 \chi  _1)
		]}
		{\sqrt{\det S_r}}
	\\
&=
\nonumber
	2\pi \, \delta_{N,M} \, \qty(
		\frac{1}{2\pi m_1 \, m_2 \, n_1}
	)^{\smallfrac{3}{2}}
	\exp(N)
	\sqrt{\frac
		{ m_1!      \; m_2!     \; n_1! }
		{ m_1^{m_1} \;m_2^{m_2} \; n_1^{n_1} }
	}
	\sqrt{m_1^2 \, m_2 \, n_1}
\\ &\qquad
	\sum_r
		\frac
		{\exp[
			\qty(- \iunit m_2 \theta_2) + 
			\qty(  \iunit n_1 \chi  _1)
		]}
		{\sqrt{1 - \frac{1}{2} \exp[\Big. 2\iunit (\theta_2 - \chi_1)]}}
	\\
&=
	\frac
		{\delta_{N,M} \exp(N)}
		{\sqrt{2\pi m_1} m_2 n_2}
	\sqrt{\frac
		{ m_1!      \; m_2!     \; n_1! }
		{ m_1^{m_1} \;m_2^{m_2} \; n_1^{n_1} }
	}
	\sum_r \frac
		{\exp[ \Big. \iunit
			\qty( n_1 \chi_1 - m_2 \theta_2 )
		]}
		{\sqrt{1 - \frac{1}{2} \exp[\Big. 2\iunit (\theta_2 - \chi_2)]}}
\end{align}

\chapter{Treatment of Zero-Occupancy Scenarios in Both Ends}
Regarding the results of the last two chapters it is assumed that combining the results is valid and does not impose new rules.

Again, $j^*$ shall denote the reference incoming mode. The number of incoming occupied modes is 
$b_i^* < K$ and that of outgoing occupied modes is $b_o^* < K$.

\section{Generalized Formulae}
\begin{align}
	\lambda
&=
	\qty( \prod_{k : n_k \neq 0} n_k )
	\qty( \prod_{j : m_j \neq 0} m_j )
\label{eqn:DefLambdaGeneric_mn}
\\
	\tilde{u}_{j,k} 
&=
	\begin{cases}
		0			&\qq*{for } j = j^* \\
		0			&\qq*{for } m_j = 0 \\
		0			&\qq*{for } n_k = 0 \\
		-u_{j,k}		&\qotherwise*
	\end{cases}
\label{eqn:DefReducedDFTGeneric_mn}
\end{align}

\subsection{Generalized Determinant}
\begin{align}
	\det S
&=
	\frac{ \lambda^{-b_i^* - b_o^* + 2} }{ m_{j^*} }
	\qty( \prod_{k: n_k \neq 0} \exp(-2\iunit \chi_k) )
	\det\qty[\bigg.
		\diag\qty(\exp(\Big.2\iunit\vec{\chi}))
		-
		\tilde{\mathbb{U}}\transp
		\diag\qty(\exp(2\iunit\vec{\theta}))
		\tilde{\mathbb{U}}
	]
\label{eqn:detSGeneralized_mn}
\end{align}

\subsection{Generalized Amplitude}
\begin{multline}
	A_F(\vec{m}, \vec{n})
=
	2\pi \, \delta_{N,M} \, \qty(
		\frac{1}{2\pi\lambda}
	)^{\frac{b_i^* + b_o^*}{2}}
	\exp(N)
	\qty( \prod_{k : n_k \neq 0}
		\sqrt{\frac
			{ n_k! }
			{ n_k^{n_k} }
	})
	\qty( \prod_{j : m_j \neq 0}
		\sqrt{\frac
			{ m_j! }
			{ m_j^{m_j} }
	})
\\
	\sum_r
		\frac
		{\exp[
			\qty(- \sum_{j\neq j^* \land m_j \neq 0} \iunit m_j \theta_j) + 
			\qty(  \sum_{k : n_k \neq 0}             \iunit n_k \chi  _k)
		]}
		{\sqrt{\det S_r}}
\end{multline}

\section{Specialization: 2x2}
\begin{align}
	\tilde{\mathbb{U}}
&=
	\begin{pmatrix}
		0 & 0 \\ 0 & 0
	\end{pmatrix}
&
	\det S
&=
	\frac{ \lambda^{-b_i^* - b_o^* + 2} }{ m_{j^*} }
=
	\frac{1}{m_{j^*}}
\end{align}

\subsection{Unoccupied $m_1$ and $n_1$ mode}
\begin{align}
	\lambda &= m_2 \, n_2
&
	j^* &= 2
\end{align}
\begin{align}
	A_F(m_2, n_2)
&=
	2\pi \, \delta_{m_2,n_2} \, \qty(
		\frac{1}{2\pi \, m_2 \, n_2}
	)^{\frac{1 + 1}{2}}
	\exp(N)
	\sqrt{\frac
		{ m_2!      \; n_2! }
		{ m_2^{m_2} \; n_2^{n_2} }
	}
	\sqrt{m_2}
	\sum_r
		\exp( \iunit n_2 \chi_2 )
	\\
&=
	\frac
		{\exp(N) \; \delta_{m_2,n_2}}
		{\sqrt{m_2} \, n_2}
	\sqrt{\frac
		{ m_2!      \; n_2! }
		{ m_2^{m_2} \; n_2^{n_2} }
	}
	\sum_r
		\exp( \iunit n_2 \chi_2 )
\end{align}

\subsection{Unoccupied $m_1$ and $n_2$ mode}
\begin{align}
	\lambda &= m_2 n_1
&
	j^* &= 2
\end{align}
\begin{align}
	A_F(m_2, n_1)
&=
	\frac
		{\exp(N) \; \delta_{m_2,n_1}}
		{\sqrt{m_2} \, n_1}
	\sqrt{\frac
		{ m_2!      \; n_1! }
		{ m_2^{m_2} \; n_1^{n_1} }
	}
	\sum_r
		\exp( \iunit n_1 \chi_1 )
\end{align}

\subsection{Unoccupied $m_2$ and $n_1$ mode}
\begin{align}
	\lambda &= m_1 n_2
&
	j^* &= 1
\end{align}
\begin{align}
	A_F(m_1, n_2)
&=
	\frac
		{\exp(N) \; \delta_{m_1,n_2}}
		{\sqrt{m_1} \, n_2}
	\sqrt{\frac
		{ m_1!      \; n_2! }
		{ m_1^{m_1} \; n_2^{n_2} }
	}
	\sum_r
		\exp( \iunit n_2 \chi_2 )
\end{align}

\subsection{Unoccupied $m_2$ and $n_2$ mode}
\begin{align}
	\lambda &= m_1 n_1
&
	j^* &= 1
\end{align}
\begin{align}
	A_F(m_1, n_1)
&=
	\frac
		{\exp(N) \; \delta_{m_1,n_1}}
		{\sqrt{m_1} \, n_1}
	\sqrt{\frac
		{ m_1!      \; n_1! }
		{ m_1^{m_1} \; n_1^{n_1} }
	}
	\sum_r
		\exp( \iunit n_1 \chi_1 )
\end{align}

\section{Recovery of the fully occupied formula}
\emph{NZO} denotes \emph{No-Zero-Occupancy}.
\begin{itemize}
\item $\lambda$ in the generalized form yields the NZO form.
\item $\tilde{\mathbb{U}}$ in the generalized form yields the NZO form.
\item $\det S$ in the generalized form yields the NZO form.
\item $A_F$ in the generalized form yields the NZO form.
\item[\Thus] This is a true generalization of the known NZO form.
\end{itemize}

\end{document}